\documentclass[12pt,]{article}
\usepackage{lmodern}
\usepackage{amssymb,amsmath}
\usepackage{ifxetex,ifluatex}
\usepackage{fixltx2e} % provides \textsubscript
\ifnum 0\ifxetex 1\fi\ifluatex 1\fi=0 % if pdftex
  \usepackage[T1]{fontenc}
  \usepackage[utf8]{inputenc}
\else % if luatex or xelatex
  \ifxetex
    \usepackage{mathspec}
  \else
    \usepackage{fontspec}
  \fi
  \defaultfontfeatures{Ligatures=TeX,Scale=MatchLowercase}
    \setmainfont[]{Times New Roman}
\fi
% use upquote if available, for straight quotes in verbatim environments
\IfFileExists{upquote.sty}{\usepackage{upquote}}{}
% use microtype if available
\IfFileExists{microtype.sty}{%
\usepackage{microtype}
\UseMicrotypeSet[protrusion]{basicmath} % disable protrusion for tt fonts
}{}
\usepackage[margin=1in]{geometry}
\usepackage{hyperref}
\hypersetup{unicode=true,
            pdftitle={Reading reflections},
            pdfauthor={Shen Qu},
            pdfborder={0 0 0},
            breaklinks=true}
\urlstyle{same}  % don't use monospace font for urls
\usepackage{graphicx,grffile}
\makeatletter
\def\maxwidth{\ifdim\Gin@nat@width>\linewidth\linewidth\else\Gin@nat@width\fi}
\def\maxheight{\ifdim\Gin@nat@height>\textheight\textheight\else\Gin@nat@height\fi}
\makeatother
% Scale images if necessary, so that they will not overflow the page
% margins by default, and it is still possible to overwrite the defaults
% using explicit options in \includegraphics[width, height, ...]{}
\setkeys{Gin}{width=\maxwidth,height=\maxheight,keepaspectratio}
\IfFileExists{parskip.sty}{%
\usepackage{parskip}
}{% else
\setlength{\parindent}{0pt}
\setlength{\parskip}{6pt plus 2pt minus 1pt}
}
\setlength{\emergencystretch}{3em}  % prevent overfull lines
\providecommand{\tightlist}{%
  \setlength{\itemsep}{0pt}\setlength{\parskip}{0pt}}
\setcounter{secnumdepth}{0}
% Redefines (sub)paragraphs to behave more like sections
\ifx\paragraph\undefined\else
\let\oldparagraph\paragraph
\renewcommand{\paragraph}[1]{\oldparagraph{#1}\mbox{}}
\fi
\ifx\subparagraph\undefined\else
\let\oldsubparagraph\subparagraph
\renewcommand{\subparagraph}[1]{\oldsubparagraph{#1}\mbox{}}
\fi

%%% Use protect on footnotes to avoid problems with footnotes in titles
\let\rmarkdownfootnote\footnote%
\def\footnote{\protect\rmarkdownfootnote}

%%% Change title format to be more compact
\usepackage{titling}

% Create subtitle command for use in maketitle
\providecommand{\subtitle}[1]{
  \posttitle{
    \begin{center}\large#1\end{center}
    }
}

\setlength{\droptitle}{-2em}

  \title{Reading reflections}
    \pretitle{\vspace{\droptitle}\centering\huge}
  \posttitle{\par}
  \subtitle{USP 570}
  \author{Shen Qu}
    \preauthor{\centering\large\emph}
  \postauthor{\par}
      \predate{\centering\large\emph}
  \postdate{\par}
    \date{Week 5}


\begin{document}
\maketitle

\begin{itemize}
\tightlist
\item
  Virtuous circle or vicious circle in mode split
\end{itemize}

Levinson and Krizek (2018 Chapter.5) introduce some concepts about
individual demand, modal competition, and network effects explain the
mode choice decisions. The discipline of psychology defines travel
behavior as habitual behavior which means ``learned sequences of acts
that have become automatic responses to specific cues, and are
functional in obtaining certain goals or end-states.''

Travelers are generally individually rational. They select the
automobile for transit's weakness in terms of door-to-door travel time.
\textbf{Wardrop's Principle of User Equilibrium} states that users are
minimizing their own time rather than reducing society's overall travel
time. \textbf{arms race} is another reason of consumer prefer SUVs than
compact cars, choosing car instead of bicycle in some case. The
competition between users leads to overuse ``common pool'' resources
that have limited supply and free access, bids up the cost for everyone.
The competition between modes may result in socially sub-optimal results
and lead to a vicious circle.

To avoid the \textbf{Prisoner's Dilemma}, a mechanism should let players
consider their effects on others, care about the future payoffs, and
play a Pareto efficient strategy. The investment and subsidies in
transit show collective rationality for lower total social travel costs
by a high transit mode share. The \textbf{Mohring Effect} states that
when bus frequency increases on a given route, users benefit from
reduced waiting times. An increasing returns property of networks leads
to a positive feedback loop.

Transit supply and demand have two stable states: One at high wait time
yields zero ridership, which returns high wait time. Another at low wait
time produces high ridership, which returns low wait times. The interim
states are not stable and need large subsidies to prop them up. Thus,
when we choose a study case - a city, or a light rail line, we need to
make an instrumental judgment: Where is the point located and what
direction is it moving? And then to find out the current and potential
influent factors.

\begin{itemize}
\tightlist
\item
  Describing people's travel
\end{itemize}

In Chapter.6, Levinson and Krizek (2018) describe and explain people's
time spent on activities and transportation. The destination
characteristics strongly influence trip characteristics. These
descriptions only let us realize the complexity of travel behaviors.
It's still hard to understand and measure the whole process of trips.
The analysts use several strategies to understand the time spent in
travel. Work versus non-work trips, simple versus complex tours, there
are many dimensions of travel. The common methods include calculating
the distance, number of trips, and the mode of travel. Measures can be
averaged at different levels from a single individual to a household, to
transportation analysis zones, or even to entire metropolitan regions.
Hagerstrand's Space-Time Prism provides a powerful method for
representing the relationships among activities, individual trips, and
total travel.

``Travel is a derived demand, and therefore we only do it when we want
to get somewhere else or when necessary.'' In other words, how complex
people's activities determine how complex people's travels are. The
concoction of behaviors is difficult to predict. Hamilton suggests that
people's actual commutes were eight times longer than model-predicted
values for shortest commutes. But I don't agree that ``even the most
robust models that predict travel distance rarely explain more than 30
percent of the observed variation.'' (Levinson and Krizek 2018, 107) The
`observed variation' might just represent 30 percent of the population
of the transportation system. A large amount of travel, activities, and
choices are not observable now. Many collected data are not random and
dependent. It like someone imagines an elephant by observing its ears.
The problem is not his/her judgment, is the limited information.
Improving data collection is often harder and more expensive than
changing modeling. But the substantial improvement for predicting travel
behavior rely more on better data.

\hypertarget{references}{%
\section*{References}\label{references}}
\addcontentsline{toc}{section}{References}

\hypertarget{refs}{}
\leavevmode\hypertarget{ref-levinson2018metropolitan}{}%
Levinson, David M, and Kevin J Krizek. 2018. \emph{Metropolitan Land Use
and Transport: Planning for Place and Plexus}. Routledge.
\url{https://doi.org/10.4324/9781315684482}.


\end{document}
