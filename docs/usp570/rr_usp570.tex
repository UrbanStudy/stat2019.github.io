\documentclass[12pt,]{article}
\usepackage{lmodern}
\usepackage{amssymb,amsmath}
\usepackage{ifxetex,ifluatex}
\usepackage{fixltx2e} % provides \textsubscript
\ifnum 0\ifxetex 1\fi\ifluatex 1\fi=0 % if pdftex
  \usepackage[T1]{fontenc}
  \usepackage[utf8]{inputenc}
\else % if luatex or xelatex
  \ifxetex
    \usepackage{mathspec}
  \else
    \usepackage{fontspec}
  \fi
  \defaultfontfeatures{Ligatures=TeX,Scale=MatchLowercase}
    \setmainfont[]{Times New Roman}
\fi
% use upquote if available, for straight quotes in verbatim environments
\IfFileExists{upquote.sty}{\usepackage{upquote}}{}
% use microtype if available
\IfFileExists{microtype.sty}{%
\usepackage{microtype}
\UseMicrotypeSet[protrusion]{basicmath} % disable protrusion for tt fonts
}{}
\usepackage[margin=1in]{geometry}
\usepackage{hyperref}
\hypersetup{unicode=true,
            pdftitle={Reading reflections},
            pdfauthor={Shen Qu},
            pdfborder={0 0 0},
            breaklinks=true}
\urlstyle{same}  % don't use monospace font for urls
\usepackage{graphicx,grffile}
\makeatletter
\def\maxwidth{\ifdim\Gin@nat@width>\linewidth\linewidth\else\Gin@nat@width\fi}
\def\maxheight{\ifdim\Gin@nat@height>\textheight\textheight\else\Gin@nat@height\fi}
\makeatother
% Scale images if necessary, so that they will not overflow the page
% margins by default, and it is still possible to overwrite the defaults
% using explicit options in \includegraphics[width, height, ...]{}
\setkeys{Gin}{width=\maxwidth,height=\maxheight,keepaspectratio}
\IfFileExists{parskip.sty}{%
\usepackage{parskip}
}{% else
\setlength{\parindent}{0pt}
\setlength{\parskip}{6pt plus 2pt minus 1pt}
}
\setlength{\emergencystretch}{3em}  % prevent overfull lines
\providecommand{\tightlist}{%
  \setlength{\itemsep}{0pt}\setlength{\parskip}{0pt}}
\setcounter{secnumdepth}{0}
% Redefines (sub)paragraphs to behave more like sections
\ifx\paragraph\undefined\else
\let\oldparagraph\paragraph
\renewcommand{\paragraph}[1]{\oldparagraph{#1}\mbox{}}
\fi
\ifx\subparagraph\undefined\else
\let\oldsubparagraph\subparagraph
\renewcommand{\subparagraph}[1]{\oldsubparagraph{#1}\mbox{}}
\fi

%%% Use protect on footnotes to avoid problems with footnotes in titles
\let\rmarkdownfootnote\footnote%
\def\footnote{\protect\rmarkdownfootnote}

%%% Change title format to be more compact
\usepackage{titling}

% Create subtitle command for use in maketitle
\providecommand{\subtitle}[1]{
  \posttitle{
    \begin{center}\large#1\end{center}
    }
}

\setlength{\droptitle}{-2em}

  \title{Reading reflections}
    \pretitle{\vspace{\droptitle}\centering\huge}
  \posttitle{\par}
  \subtitle{USP 570}
  \author{Shen Qu}
    \preauthor{\centering\large\emph}
  \postauthor{\par}
      \predate{\centering\large\emph}
  \postdate{\par}
    \date{Week 7}


\begin{document}
\maketitle

\begin{itemize}
\tightlist
\item
  The `Diamond of Design'
\end{itemize}

Levinson and Krizek (2018 Chapter.11) introduced a new `Diamond', a
structure of four key design tenents-hierarchy, morphology, layers, and
architectural content. Here author define the term `design' as ``how
elements of place and plexus arrange their parts into a whole on a
variety of scales, from the neighborhood to the metropolis.'' This
framework even coveres the cities system as a high-level hierarchy
place. This perspective is more from engineering or computer science.
The author compare the layers of place and plexus with the OSI (Open
Systems Interconnetion) Model, a computer networking framwork. This view
point is different with the planner who have design background but it
provide a powerful tools to explain the complexity of urban land use and
transportation.

\begin{itemize}
\tightlist
\item
  Objectively measure subjective qualities
\end{itemize}

The study by Ewing et al. (2006) attempts to quantitatively measure five
urban design qualities in terms of physical characteristics of street
for walkability. The research team selected nine from 51 perceptual
qualities, 48 from more than 200 video clips, and 10 urban design and
planning experts. The expert panel are invloved in defining the
`operational' and give the walkability ratings for each clips and assign
a score for each quality on a scale from 1 (low) to 5 (high). This study
used the fractional factorial design, the crossed multilevel Design,
random effects models, and linear regression models to built up the
relationship among physical teatures, urban design qulities ad overall
walkability.

At the end the team dropped four qualities and choose imageability,
enclosure, human scale, and transparency as the measurement of
walkability based on five criteria including significant level,
ICC,\footnote{Intraclass Correlation Coefficient reflects the proportion
  of the variance of an observation that is the result of differences
  between treatments.} and variance components. This study found that 38
of more than 130 physical features have significant effects on one or
more perceptual qualities. In a later paper, Ewing and Handy (2009) gave
all the consensus qualitaitive definiations and operational definitions
for each qualities and identify detailed physical features associated
with each qualitty

\begin{itemize}
\tightlist
\item
  Discussion: predetermined or prior information.
\end{itemize}

Inversing input and output. The team selected 48 clips that ``best
matched the combinations of high/low values in a \(2^{8-4}\) fractional
factorial design (FFD).'' FFD is for simplify and improve
experiment.\footnote{``A major use of fractional factorials is in
  screening experiments---experiments in which many factors are
  considered and the objective is to identify those factors that have
  large effects. Screening experiments are usually performed in the
  early stages of a project when many of the factors initially
  considered likely have little or no effect on the response. The
  factors identified as important are then investigated more thoroughly
  in subsequent experiments.''} (Montgomery 2017) Through examining all
the combination of factors, the result of FFD should tell us which
factors and interactions are improtant or non-negligible. On the
contrary, this reaserch use FFD as a method of sample selection. ``The
\(2^{8-4}\) sample allowed us to capture the main effects\footnote{In
  FFD, ``significant interaction will often mask the significance of
  main effects. In the presence of significant interaction, the
  experimenter must usually examine the levels of one factor, say A,
  with levels of the other factors fixed to draw conclusions about the
  main effect of A.''} of each urban design quality on overall
walkability,\ldots{}to maximize geographic diversity.'' (Clemente and
Ewing 2005) Actually, what this step do is not sample selection, is
treatment selecetion. The 48 clips were rated by viewer in the next step
giving \(480\) observations on walkability and \(480\times9\)
observations on qualities.

The \(2^{8-4}\) FFD means k=8 factors, p=4 independent generators and 48
observations can assign each of 16 runs with 3 replication. Properly
choosing the generators make the effects of potential interest are not
aliased with each other. However, the papar don't provide the list of
generators and alias, don't clarify which factors are confounded and
which factors and interaction have significant effects.

The nonrandomized `random effects models'. At the beginning of research
design, research team define their purpose are ``trying to
operationalize design concepts, not assess public preferences''. They
believe ``average person'' cannot rate streetscapes as to their
``legibility,'' ``transparency,'' and so on. The expert define the
criteria and rate the clips being selceted to match some designated
qualities. Both of the viewers and the scenes are not randomly selected.
These fixed treatments don't satisfy the assumptions of random effects
models. It is not strange that ``eight of the nine qualities were
collinear. Tolerance values were unacceptably low when all variables
were included in a regression at once.'' In this situation, we cannont
extend the conclusions to all treatments in the population. Technically,
this reaserch selected 5 from 9 urban design qualities are more amenable
and would be defined operationally in the field survey instrument. The
relationship about qualities and walkability are predetermined.

It is also not enough to examine the relationship between walkability
and physical features by significant level in linear regression. We can
look this study as an iteration of the Bayesian paradigm. The prior
information \(\pi(\theta)\) (qualities) combined with the sample
information \(f(\vec x|\theta)\) (viewers and scenes) to obtain the
posterior distribution \(\pi(\theta|\vec x)\). (Casella and Berger 2002)
Collecting more and more scenes and viewers to train the models and draw
the inference.

\hypertarget{references}{%
\section*{References}\label{references}}
\addcontentsline{toc}{section}{References}

\hypertarget{refs}{}
\leavevmode\hypertarget{ref-casella2002statistical}{}%
Casella, George, and Roger L Berger. 2002. \emph{Statistical Inference}.
Vol. 2. Duxbury Pacific Grove, CA.

\leavevmode\hypertarget{ref-clemente2005identifying}{}%
Clemente, Otto, and Reid Ewing. 2005. ``Final Report-Identifying and
Measuring Urban Design Qualities Related to Walkability.''

\leavevmode\hypertarget{ref-ewing2009measuring}{}%
Ewing, Reid, and Susan Handy. 2009. ``Measuring the Unmeasurable: Urban
Design Qualities Related to Walkability.'' \emph{Journal of Urban
Design} 14 (1). Taylor \& Francis: 65--84.

\leavevmode\hypertarget{ref-ewing2006identifying}{}%
Ewing, Reid, Susan Handy, Ross C Brownson, Otto Clemente, and Emily
Winston. 2006. ``Identifying and Measuring Urban Design Qualities
Related to Walkability.'' \emph{Journal of Physical Activity and Health}
3 (s1): S223--S240.

\leavevmode\hypertarget{ref-levinson2018metropolitan}{}%
Levinson, David M, and Kevin J Krizek. 2018. \emph{Metropolitan Land Use
and Transport: Planning for Place and Plexus}. Routledge.
\url{https://doi.org/10.4324/9781315684482}.

\leavevmode\hypertarget{ref-montgomery2017design}{}%
Montgomery, Douglas C. 2017. \emph{Design and Analysis of Experiments}.
John wiley \& sons.


\end{document}
