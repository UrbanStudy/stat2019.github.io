\documentclass[12pt,]{article}
\usepackage{lmodern}
\usepackage{amssymb,amsmath}
\usepackage{ifxetex,ifluatex}
\usepackage{fixltx2e} % provides \textsubscript
\ifnum 0\ifxetex 1\fi\ifluatex 1\fi=0 % if pdftex
  \usepackage[T1]{fontenc}
  \usepackage[utf8]{inputenc}
\else % if luatex or xelatex
  \ifxetex
    \usepackage{mathspec}
  \else
    \usepackage{fontspec}
  \fi
  \defaultfontfeatures{Ligatures=TeX,Scale=MatchLowercase}
    \setmainfont[]{Times New Roman}
\fi
% use upquote if available, for straight quotes in verbatim environments
\IfFileExists{upquote.sty}{\usepackage{upquote}}{}
% use microtype if available
\IfFileExists{microtype.sty}{%
\usepackage{microtype}
\UseMicrotypeSet[protrusion]{basicmath} % disable protrusion for tt fonts
}{}
\usepackage[margin=1in]{geometry}
\usepackage{hyperref}
\hypersetup{unicode=true,
            pdftitle={Policy Memo},
            pdfborder={0 0 0},
            breaklinks=true}
\urlstyle{same}  % don't use monospace font for urls
\usepackage{graphicx,grffile}
\makeatletter
\def\maxwidth{\ifdim\Gin@nat@width>\linewidth\linewidth\else\Gin@nat@width\fi}
\def\maxheight{\ifdim\Gin@nat@height>\textheight\textheight\else\Gin@nat@height\fi}
\makeatother
% Scale images if necessary, so that they will not overflow the page
% margins by default, and it is still possible to overwrite the defaults
% using explicit options in \includegraphics[width, height, ...]{}
\setkeys{Gin}{width=\maxwidth,height=\maxheight,keepaspectratio}
\IfFileExists{parskip.sty}{%
\usepackage{parskip}
}{% else
\setlength{\parindent}{0pt}
\setlength{\parskip}{6pt plus 2pt minus 1pt}
}
\setlength{\emergencystretch}{3em}  % prevent overfull lines
\providecommand{\tightlist}{%
  \setlength{\itemsep}{0pt}\setlength{\parskip}{0pt}}
\setcounter{secnumdepth}{0}
% Redefines (sub)paragraphs to behave more like sections
\ifx\paragraph\undefined\else
\let\oldparagraph\paragraph
\renewcommand{\paragraph}[1]{\oldparagraph{#1}\mbox{}}
\fi
\ifx\subparagraph\undefined\else
\let\oldsubparagraph\subparagraph
\renewcommand{\subparagraph}[1]{\oldsubparagraph{#1}\mbox{}}
\fi

%%% Use protect on footnotes to avoid problems with footnotes in titles
\let\rmarkdownfootnote\footnote%
\def\footnote{\protect\rmarkdownfootnote}

%%% Change title format to be more compact
\usepackage{titling}

% Create subtitle command for use in maketitle
\providecommand{\subtitle}[1]{
  \posttitle{
    \begin{center}\large#1\end{center}
    }
}

\setlength{\droptitle}{-2em}

  \title{Policy Memo}
    \pretitle{\vspace{\droptitle}\centering\huge}
  \posttitle{\par}
  \subtitle{Autonomous vehicles and land use}
  \author{}
    \preauthor{}\postauthor{}
      \predate{\centering\large\emph}
  \postdate{\par}
    \date{June 11}


\begin{document}
\maketitle

To: Scott Haggerty, chair of governing Commission of Bay Area
Metropolitan Transportation Commission

From: Shen Qu, Policy Advisor

Date: 6/11/2019

RE: How Bay Area should be planning for autonomous vehicles?

\hypertarget{summary}{%
\section{Summary}\label{summary}}

This memo is one of the series of policy analysis about autonomous
vehicles for Bay Area. It trys to answer how the autonomous vehicles
will affect urban land use, and What the MPO and cities should be
planning to seize this opportunity and address the challenges.

\hypertarget{background-the-current-and-projected-status-of-avs.}{%
\section{Background: the current and projected status of
AVs.}\label{background-the-current-and-projected-status-of-avs.}}

\emph{Technology}: The history of exploring autonomous vehicles can date
back to late 1950s (Milakis 2019). Since DARPA ran the Grand Challenge
in 2004, the autonomous vehicle technologies entered a ``critical
juncture.'' (Docherty, Marsden, and Anable 2018) In the fields of
automation control systems, some critical hardwares like processors and
sensors are having the ability of undertaking more complex tasks. The
improving softwares and algorithm are becoming more fledged. According
the third version of definition and taxonomy by SAE (2018), the existing
technologies are achiving from the Level 3 - Conditional Driving
Automation to Level 4 - High Driving Automation.\footnote{SAE defines
  the concept of AV as ADS-DV (ADS-Dedicated Vehicle), ``A vehicle
  designed to be operated exclusively by a level 4 or level 5 ADS for
  all trips within its given Operational Design Domain (ODD)
  limitations.'' ADS means ``The hardware and software that are
  collectively capable of performing the entire Dynamic Driving Task
  (DDT) on a sustained basis, regardless of whether it is limited to a
  specific ODD; this term is used specifically to describe a level 3, 4,
  or 5 driving automation system.''}

\emph{Industry}: Since 2009, Google had conducted a series of tests for
AVs over 10 million miles on real-world roads in California, Texas, and
other states. Waymo, a company founded by Google, hold the only testing
permit for driverless testing by California DMV and committed to
providing a ride-hailing services in Arizona in 2018. In 2019, Waymo
announced their Level 4 AV will be assembled in Detroit. Almost all big
automakers such as Ford, General Motors, Volkswagen, etc., are investing
heavily in this field.(Crute et al. 2018) The leading transportation
network companys (TNC) like Uber and Lyft are making up a term of
Mobility-as-a-service (Maas) to change current travel modes by AVs.

\emph{Academia}: Many scholars start working on the reaserch of AVs with
lots of energy. Gandia et al. (2019)`s reserch found 10580 published
papers in this field from 1945 to 2018. Since 2012, the number of
articles have an exponential growth with 39\% growth rate while 8-9\%
average growth rate in science. Although a large amount of the research
are from the perspective of systems control, computer science, robotics,
engineering, there are more and more articls that start to focus on the
AVs' impact on transportation and Land use.

\emph{Governavce}: In 2011, the Nevada Department of Motor Vehicles
issued a first license to google's experimental AVs. Currently, ``33
states have passed legislation related to AV.'' ``15 states enacted 18
AV related bills.'' (NCSL 2019) From 2016 to 2018 the U.S. Department of
Transportation (U.S.DOT) and the National Highway and Transportation
Safety Administration (NHTSA) published three federal guidences for
Automated Driving Systems (ADS is the definition by SAE). The guidances
advocate industry, state and local government to support the AVs'
development. (NHTSA 2019) California is playing a leading role in this
field. Until January 2019, California DMV has issued AV Testing Permits
(with a driver) to 62 companies on public roadways.

Unveiling the future: Muller (2017) introduces the four stages in the
spatial evolution of the American metropolitan. Form Walking-Horsecar
Ear to Electric Streetcar Era, Recreational Auto Era, and Freeway Era.
The four-stage model shows that each ``break through in movement
technology'' had reshaped the previous dominated urban form and launched
a new era with a ``distinctive spatial structure.'' There are some
indications that a new era might be dawning. We try to answer how AVs
could influence demand for transportation and land use. This memo will
focus on the crutial response for land use. The topic of safety,
liability, and other issues will discussed in other memos.

\hypertarget{the-framworks-of-changes}{%
\section{The Framworks of changes}\label{the-framworks-of-changes}}

While industry, scholars and governments realized the high impact of AV,
some research frameworks also imply it is a highly uncertainly evolution
rising up some complex issues such as coupling, resonance, or agitation.
Milakis, Van Arem, and Van Wee (2017) arrange many substantial
implications of AVs by a structure of {[}\emph{ripple effect}{]}, whcih
reflected a sequantially spreading process. Land use is placed in the
second-order that is affected by the factors in the first-order
including travel cost, travel time, vehicle use, capacity, trvael modes,
and etc. The flaw of this structure is that the ripple effects model
emphasizes the diffusion characteristic of the AV technology and cannot
describe the feedback effects. The changes of urban form and land use
will influence the travel behavior and traffic in the first-order too.

The \emph{Diamond of Assembly} (Levinson and Krizek 2018 Chapter.12) and
the \emph{feedback cylce} (Wegener and Fürst 2004; adaped by
Soteropoulos, Berger, and Ciari 2019) are two helpful complements. These
figures can present the relationship between transportation and land use
in a more clear manner. AV technology as a exogenous variable will
influence travel behaviors in the first ripple, and then is reflected in
the change of accessiblity. The dynamic of accesessibility will interact
with land use and forms a new cycle.

The history of `the four stages' (Muller 2017) tells us, if AVs being a
breakthrough force, many previous models, methods, and arguments may be
different or even fail. Under the two kinds of the framework above, we
will review the classical theories about transportation and land use,
following the sequence of the first order, second order, and feedback
cycle.

\hypertarget{the-first-order}{%
\section{The first order}\label{the-first-order}}

\emph{The utility maximization problem}: As the core of consumer theory,
the supply and demand model explaines the relationship between the price
(travel cost) and the quantity (VMT). When price changed or the curve
shifted, the market-clearing equilibrium point will reach a new one. In
estimating the impact of AVs on travel consumption, most of current
research agree that AVs will provide more options for travelers with
less travel time and larger capacity. According the results of 37
modelling studies from 2013 to 2018, The application of AVs means the
lower price and higher service qualities, which will induce more and
longer trips, produce more total VMT (Soteropoulos, Berger, and Ciari
2019). At the same time, these research noice that more available
services by AVs may encourage low-occupied vehile, reduce transit use,
and invrease VMT. (Taiebat et al. 2018) The tradeoff between higher
effecient and higher demand makes the trends of congestion and emission
being uncertaint. Overall, these are equalising and incremental changes
in quantities.

However, as a breakthrough technology, the impacts of AVs may not follow
some divergent curves and show the polarising characteristics. For
example, the \emph{prospect theory} of value and gains shows a logistic
curve. (Levinson and Krizek 2018, 6) When coming to the relationship of
bus ridership to wait time, the curve shows the initial point is not
stable. (Levinson and Krizek 2018, 83) A slightly change will let it
slipe to two stabel equilibria. But the equilibrious points represent
two direction, a vicious circle (low ridership and high wait time) or a
virtuous circle (high ridership and low wait time).

To explain this phenomenon, we need understand the components of travel
cost and current options on travel modes. The opportunity cost and
transaction cost are the significant part of travel cost. Being late for
work, a meeting, or a flight means much higher opportunity loss, whcih
can explain why people hat uncertaint and wait time. Similarly, hailing
a taxi for each trip or planning carpool everyday means the transaction
cost are unacceptable expensive. Meanwhile, the available travel modes
options are limited. Once people choosing transit, their trips often
lost the reliability and flexibility. They also have few options on the
transit quality, such as speed, headway, and routes. Thus, people choose
to own a private car to control the high opportunity cost and
transaction cost. Now we know why the initial point is not stable.

The practices of mobility-as-a-services (Maas) proved by TNC introduced
a potential solution. For the connected AV's `perfect information' and
automation, Mass could largely reduce the opportunity cost and
transaction cost, then could reduce car ownership. Moreover, It provdes
a series of options by monetized the values of reliability and
flexibility. A hurried passenger may will pay \$100 for arriving the
airport as soon as possible. A worker may will get up one hour earlier
for a \$10 off-peak pass per month.

This is a business mode transition in many ways. Many research show that
the ridersharing level is a critical variable for AVs application. The
preseting value of ridershare could drastically change the simulating
results of the models. What's more, travel time could become a positive
factor in transportation models. The value of time as a large part of
travel cost may decrease and even could turn to a productive gain
(Singleton 2019). Although scholars are discussing how significance the
time cost will change, that would like a suddenly switching lanes rather
than a gradual process. We should realize the impacts of AVs may keep
accelerating until all the energy released and should reset the prvious
equations with new parameters and even new distributions.

\hypertarget{the-second-order}{%
\section{The second order}\label{the-second-order}}

As shown in Milakis, Van Arem, and Van Wee (2017)'s ripple effect, the
first-order changes triggered by AVs are not about land use. But these
forces will push the land use changes forward.

An essential reason is that the travel cost is positively correlated
with distance. From the technical perspective, \emph{the gravity model}
for measuring accessibility considers the mass of place and distance
(travel time) as the determinative factors. From the economic
perspective, all the activities want to minimize the cost and compete
for the land close to city center. \emph{The bid-rent theroy} oriented
form Thünen (1826)`s model, which derived from the utility maximization
through introducing the spatial variables. From the 'concentric zone
model' by Burgess (2008), to Christaller (1933)'s central place theory,
and to Alonso and others (1964)'s land market model, this economic
theory is the basis of many land use models. The relevant research
conclude that wider urban sprawl might be an output of AVs application
(Soteropoulos, Berger, and Ciari 2019). In the same way, the research
about parking say that the CBD could be more dense for less demand of
parking lot {[}{]}. However, these inferences build on the theories of
spatial economic equilibrium. The underline assumptions are the
functions remain the same and the incremental changes happen on values
and parameters. There are less research from the theories of spatial
polarisation (Pred 1966).

Hagerstrand (1970)'s \emph{action-space theory} proposes ``action spaces
are limited by three types of constraints: capacity constraints,
coupling constraints, and institutional constraints.'' While the growth
of capacity is still incremental, the matching cost might be
negligitable in AVs era. The \emph{theory of time and travel budgets} by
Zahavi, Beckmann, and Golob (1981) and Downes and Emmerson (1985) agree
that ``longer trips make more dispersed locations.'' But existing
literature don't talk a lot about what is ``a higher degree of spatial
division of labour possible'' in AVs era.

The theory of \emph{network society} by Castells (2011) and other theory
of the information society imply another possible perspect of land use
pattern in the AVs era. If the travel demands and supplies can match in
realtime and generate as many as possible options for customers, the
`space of places' also could become the `space of flows.' The
traditional forms, including cores, clusters, and corridors will be
disintegrated. In the long run, The network may control the travel and
assign people to a place. People only choose their activities and
surrounding environment but don't care about the actual locations.

\hypertarget{the-feedback-cylce}{%
\section{The feedback cylce}\label{the-feedback-cylce}}

In this section, we try to explore what changes may happen on land use
feedback cycle brought by the AVs technology. From the temporal
dimension, the theory of long waves by (Kondratieff 1926) and
(Schumpeter 1939) discribed the cycle of a ``earlier technologies went
from invention through take-off and rapid growth to saturation and was
eventually superseded by a more advanced technology.'' From the spatial
dimension, the changes triggered by AVs may like a ``cyclical sequence
of agglomeration and deglomeration phases'' (Van den Berg et al. 1982).

Although many funding tools try to combine the transportation
investiment and land value return together to incentive virtuous circle,
there always some drawbacks (Vadali et al. 2018). From the perspective
of equalty, charging an unmeasurable service is unfair. From the
perspective of market, an unmeasurable trade is inevitablelly
inefficient. The feedback cylce shows a decay process for free rider
problem, or for delay of demand response. Pricing by demand-supply
mechanism becomes infeasible.

\emph{Tiebout model} contributes a non-political solution to optimal
public goods provision (Tiebout 1956). The ``foot voting'' by the mobile
residents motivates the competition between communities. The primary
assumptions Tiebout model relied on are that conusmers can freely choose
where they live. It assumes that there are engough commuinties available
and commuting cost is negligible. there are not externalities or
spillover of public goods across towns. Many scholars critique Tiebout
model for its unpractical assumptions. For the same reasons, Tiebout
model is more success in suburban areas. The roads in suburban
communities like a club goods. There is no free rider problem. All the
cost and externalities internalized in property tax.

Entering the AV era, Tiebout model might apply to whole urban area. Maas
suppored by connected AV allows us to measure the road use for each
trip. Moreover, Maas provide the oppotuinity of internalizing all the
positive/negative externalities and redefine the boundary of public
good. The theory of local governments competetion and beneficiary pays
principle can be fully applied on the new feedback cycle. Under this
scenario, transportation become a excludable and rivalrous goods. The
cities or communities compete by investing and improving infrastucture;
Transportation suppliers bid for road license and provide as many as
possible options for consumers; Consumers choose the travel services
based on their ability and willingness to pay money or change their
itineraries. The dark side is the social segregation in Schelling's
model (Schelling 1971) may be more significant and invisible. But, as
Levinson and Krizek (2018) say, ``prices create choices, and choices are
fair.'' (p.248)

Behavior and land use (Soteropoulos, Berger, and Ciari 2019)

(Hawkins and Nurul Habib 2019)

long-term effects (Milakis 2019)

(Fagnant and Kockelman 2015)

Zmud et al. (2018)

\hypertarget{the-policy-options}{%
\section{The policy options}\label{the-policy-options}}

As Crute et al. (2018) suggested, In response to the changes by AVs
technology, the policy makers should ``bolster transportation demand
management, reconsider the right-of-way, and continue to develop
transit.'' These types of method are aimed at the first-order changes.
``Rethinking the parking standard and requirements'' could help better
redevelop the urban parking lots saved by AVs. Urabn Groth Boundary is
also a ready-made tool for controling sprawl. In the face of those
gradual changes, existing theories and policy tools are sufficient to
cope with them.

The true challenges are how to response the selfreinforcement changes,
the new distributions, and the networks never seen before? And how to
let AVs leverage the virtuous circle rather than vicious? The solution
of Maas provides a valuable space for policy-makers to guide the
transportation system in the direction we want. Responding to the
promotion of Maas by industry, government should advocate ``a simpler,
smarter and fairer system of road user charging''.(Barrett, Wedderburn,
and Belcher 2019) The London government is launching a new scheme of
``charging drivers on a per-mile basis based on distanced travelled,
vehicle emissions, local levels of congestion and pollution, and
availability of public transport alternatives.'' Hand (2016)'s suggests
to start some pilot projects. SDOT (2017)'s policy want to ``establish a
city-owned transportation network company digital platform to incubate
smaller shared AV fleet businesses'' in Los Angele.

Other MPAs are on the move. This memo cannot cover all the AVs' issues
but want to urges MTC to realize the AVs' substantial effects and
support the relevant research for the future of Bay Area.

mitigating

the change on road capacities, parking lots, curve space.

Above analysis are inference base on exsiting theories and research.

It's hard to conduct empirical studies of Land-use Tansport interaction
under the AV technology.

The stated preference, reveled preference, simulation by models

\begin{itemize}
\tightlist
\item
  The response for these incremental changes:
\end{itemize}

The ripple effects

cut off labor cost, the change of parking, affected by travel demand and
behavior

Bid for Mobility. pircing policy

such scheme would reflect the true impact of individual vehicle
journeys. be integrated with London's wider transport system and be
accessible via an app and digital platform. This would allow Londoners
to compare, plan and pay for all journeys in one place. Ultimately it
would encourage drivers to leave their cars at home when possible, by
providing them with alternative travel options.

\begin{itemize}
\tightlist
\item
  The response for these shifting changes:
\end{itemize}

Heavy property and light property.

feedback cycles produce

housing, urban design,

\begin{itemize}
\tightlist
\item
  The recommended initiative changes:
\end{itemize}

Research: Identifies the benefits and costs of these possible outcomes.

round-the-clock services.

full ridesharing by realtime matching

the impact on land use,

The long-term influences include the reconstructure of urban forms and
spatial distributions.

use cost and transaction costs - full match

deals fail

adjusting, adapting, guiding

(Legacy et al. 2019)

Presents policy and planning options for mitigating or otherwise
addressing the possible land use effects.

designating pilot area

Discusses how the MPO and cities may need alter the tools and analyses
they use to consider AVs.

Zoning, Division, and partion, not uniform

\hypertarget{conclusion}{%
\section{Conclusion}\label{conclusion}}

overestimated and under estimate

from link to node

CA should play a leading role. responsibility

\hypertarget{notes}{%
\section{Notes}\label{notes}}

The concept of \emph{Public good} helps to understand the cyclical
process. Transprotation infrastructure is the semi-public good. Except
toll road, this system is non-excludable. But the congested sections in
the peak hour are rivalrous. Fuel taxs partly reflect the amount of use
of road, but can not adjust the spatial distribution. Even the
congestion charging like London, is still a binary intervention (Yes or
no). Some research conclude that the success of pricing policy depend on
political support. {[}{]} But the research don't realize that the
current charging plan cannot effectively and accurately reflect the use
of transprotation infrastructure.

\hypertarget{references}{%
\section*{References}\label{references}}
\addcontentsline{toc}{section}{References}

\hypertarget{refs}{}
\leavevmode\hypertarget{ref-alonso1964location}{}%
Alonso, William, and others. 1964. ``Location and Land Use. Toward a
General Theory of Land Rent.'' \emph{Location and Land Use. Toward a
General Theory of Land Rent.} Cambridge, Mass.: Harvard Univ. Pr.

\leavevmode\hypertarget{ref-barrett2019green}{}%
Barrett, S, M Wedderburn, and E Belcher. 2019. ``Green Light: Next
Generation Road User Charging for a Healthier, More Liveable, London.''
\url{www.centreforlondon.org/publication/road-user-charging/}.

\leavevmode\hypertarget{ref-burgess2008growth}{}%
Burgess, Ernest W. 2008. ``The Growth of the City: An Introduction to a
Research Project.'' In \emph{Urban Ecology}, 71--78. Springer.

\leavevmode\hypertarget{ref-castells2011rise}{}%
Castells, Manuel. 2011. \emph{The Rise of the Network Society}. Vol. 12.
John wiley \& sons.

\leavevmode\hypertarget{ref-christaller1933zentralen}{}%
Christaller, Walter. 1933. \emph{Die Zentralen Orte in Süddeutschland:
Eine ökonomisch-Geographische Untersuchung über Die Gesetzmässigkeit Der
Verbreitung Und Entwicklung Der Siedlungen Mit Städtischen Funktionen}.
University Microfilms.

\leavevmode\hypertarget{ref-APA2018autonomous}{}%
Crute, Jeremy, William Riggs, Timothy Stewart Chapin, and Lindsay
Stevens. 2018. ``Planning for Autonomous Mobility.'' PAS Report 592.
American Planning Association.

\leavevmode\hypertarget{ref-docherty2018governance}{}%
Docherty, Iain, Greg Marsden, and Jillian Anable. 2018. ``The Governance
of Smart Mobility.'' \emph{Transportation Research Part A: Policy and
Practice} 115. Elsevier: 114--25.
\url{https://doi.org/10.1016/j.tra.2017.09.012}.

\leavevmode\hypertarget{ref-downes1985urban}{}%
Downes, JD, and P Emmerson. 1985. ``Urban Travel Modelling with Flexible
Travel Budgets.'' \emph{Crowthorne: Transport and Road Research
Laboratory}.

\leavevmode\hypertarget{ref-fagnant2015preparing}{}%
Fagnant, Daniel J, and Kara Kockelman. 2015. ``Preparing a Nation for
Autonomous Vehicles: Opportunities, Barriers and Policy
Recommendations.'' \emph{Transportation Research Part A: Policy and
Practice} 77. Elsevier: 167--81.

\leavevmode\hypertarget{ref-gandia2019autonomous}{}%
Gandia, Rodrigo Marçal, Fabio Antonialli, Bruna Habib Cavazza, Arthur
Miranda Neto, Danilo Alves de Lima, Joel Yutaka Sugano, Isabelle
Nicolai, and Andre Luiz Zambalde. 2019. ``Autonomous Vehicles:
Scientometric and Bibliometric Review.'' \emph{Transport Reviews} 39
(1). Taylor \& Francis: 9--28.
\url{https://doi.org/10.1080/01441647.2018.1518937}.

\leavevmode\hypertarget{ref-hagerstrand1970people}{}%
Hagerstrand, T. 1970. ``What About People in Spatial Science.''
\emph{Regional Science Association} 24: 7--21.

\leavevmode\hypertarget{ref-hand2016mobility_LA}{}%
Hand, Ashley Z. 2016. ``Urban Mobility in a Digital Age - a
Transportation Technology Strategy for Los Angeles.''
\url{www.urbanmobilityla.com/}.

\leavevmode\hypertarget{ref-hawkins2019integrated}{}%
Hawkins, Jason, and Khandker Nurul Habib. 2019. ``Integrated Models of
Land Use and Transportation for the Autonomous Vehicle Revolution.''
\emph{Transport Reviews} 39 (1). Taylor \& Francis: 66--83.
\url{https://doi.org/10.1080/01441647.2018.1449033}.

\leavevmode\hypertarget{ref-kondratieff1926langen}{}%
Kondratieff, ND. 1926. ``Die Langen Wellen Der Konjunktur. Archiv Fuer
Sozialwissenschaft Und Sozialpolitik.'' \emph{Tuebingen} 56 (3): 537.

\leavevmode\hypertarget{ref-legacy2019planning}{}%
Legacy, Crystal, David Ashmore, Jan Scheurer, John Stone, and Carey
Curtis. 2019. ``Planning the Driverless City.'' \emph{Transport Reviews}
39 (1). Taylor \& Francis: 84--102.
\url{https://doi.org/10.1080/01441647.2018.1466835}.

\leavevmode\hypertarget{ref-levinson2018metropolitan}{}%
Levinson, David M, and Kevin J Krizek. 2018. \emph{Metropolitan Land Use
and Transport: Planning for Place and Plexus}. Routledge.
\url{https://doi.org/10.4324/9781315684482}.

\leavevmode\hypertarget{ref-milakis2019long}{}%
Milakis, Dimitris. 2019. ``Long-Term Implications of Automated Vehicles:
An Introduction.'' Taylor \& Francis.
\url{https://doi.org/10.1080/01441647.2019.1545286}.

\leavevmode\hypertarget{ref-milakis2017policy}{}%
Milakis, Dimitris, Bart Van Arem, and Bert Van Wee. 2017. ``Policy and
Society Related Implications of Automated Driving: A Review of
Literature and Directions for Future Research.'' \emph{Journal of
Intelligent Transportation Systems} 21 (4). Taylor \& Francis: 324--48.
\url{https://doi.org/10.1080/15472450.2017.1291351}.

\leavevmode\hypertarget{ref-Muller2017transportation}{}%
Muller, Peter O. 2017. ``Transportation and Urban Form.'' In \emph{The
Geography of Urban Transportation, Fourth Edition}, edited by G.
Giuliano and S. Hanson, 57--85. Guilford Publications.
\url{https://books.google.com/books?id=J3GnDQAAQBAJ}.

\leavevmode\hypertarget{ref-NCSL2019AV}{}%
NCSL. 2019. ``Autonomous Vehicles \textbar{} Self-Driving Vehicles
Enacted Legislation.'' National Conference of State Legislatures. 2019.
\url{http://www.ncsl.org/research/transportation/autonomous-vehicles-self-driving-vehicles-enacted-legislation.aspx}.

\leavevmode\hypertarget{ref-NHTSA2019ADS}{}%
NHTSA. 2019. ``Automated Driving Systems.'' National Highway;
Transportation Safety Administration. 2019.
\url{https://www.nhtsa.gov/vehicle-manufacturers/automated-driving-systems}.

\leavevmode\hypertarget{ref-pred1966spatial}{}%
Pred, Allan Richard. 1966. \emph{The Spatial Dynamics of Us
Urban-Industrial Growth, 1800-1914: Interpretive and Theoretical
Essays}. MIT press.

\leavevmode\hypertarget{ref-sae2018taxonomy}{}%
SAE. 2018. ``Taxonomy and Definitions for Terms Related to Driving
Automation Systems for on-Road Motor Vehicles.'' J3016. SAE
International.

\leavevmode\hypertarget{ref-schelling1971dynamic}{}%
Schelling, Thomas C. 1971. ``Dynamic Models of Segregation.''
\emph{Journal of Mathematical Sociology} 1 (2). Taylor \& Francis:
143--86.

\leavevmode\hypertarget{ref-schumpeter1939business}{}%
Schumpeter, Joseph Alois. 1939. \emph{Business Cycles: A Theoretical,
Historical, and Statistical Analysis of the Capitalist Process}.
McGraw-Hill New York.

\leavevmode\hypertarget{ref-SDOT2017mobility_Sea}{}%
SDOT. 2017. ``New Mobility Playbook.'' Seattle Department of
Transportation. \url{newmobilityseattle.info/}.

\leavevmode\hypertarget{ref-singleton2019discussing}{}%
Singleton, Patrick A. 2019. ``Discussing the `Positive Utilities' of
Autonomous Vehicles: Will Travellers Really Use Their Time
Productively?'' \emph{Transport Reviews} 39 (1). Taylor \& Francis:
50--65. \url{https://doi.org/10.1080/01441647.2018.1470584}.

\leavevmode\hypertarget{ref-soteropoulos2019impacts}{}%
Soteropoulos, Aggelos, Martin Berger, and Francesco Ciari. 2019.
``Impacts of Automated Vehicles on Travel Behaviour and Land Use: An
International Review of Modelling Studies.'' \emph{Transport Reviews} 39
(1). Taylor \& Francis: 29--49.
\url{https://doi.org/10.1080/01441647.2018.1523253}.

\leavevmode\hypertarget{ref-taiebat2018review}{}%
Taiebat, Morteza, Austin L Brown, Hannah R Safford, Shen Qu, and Ming
Xu. 2018. ``A Review on Energy, Environmental, and Sustainability
Implications of Connected and Automated Vehicles.'' \emph{Environmental
Science \& Technology} 52 (20). ACS Publications: 11449--65.
\url{https://doi.org/10.1021/acs.est.8b00127}.

\leavevmode\hypertarget{ref-thunen1826isolierte}{}%
Thünen, JH von. 1826. ``Der Isolierte Staat.'' \emph{Beziehung Auf
Landwirtschaft Und Nationalökonomie}.

\leavevmode\hypertarget{ref-tiebout1956pure}{}%
Tiebout, Charles M. 1956. ``A Pure Theory of Local Expenditures.''
\emph{Journal of Political Economy} 64 (5). The University Press of
Chicago: 416--24. \url{https://doi.org/10.1086/257839}.

\leavevmode\hypertarget{ref-vadali2018guidebook}{}%
Vadali, Sharada, Johanna Zmud, Todd Carlson, Karin DeMoors, Rick Rybeck,
Steven Fitzroy, Naomi Stein, and Mark Sieber. 2018. \emph{Guidebook to
Funding Transportation Through Land Value Return and Recycling}. Project
19-13. \url{https://doi.org/10.17226/25110}.

\leavevmode\hypertarget{ref-van1982urban}{}%
Van den Berg, Leo, Roy Drewett, Leo H Klaasen, Angelo Rossi, and
Cornelis HT Vijverberg. 1982. ``Urban Europe: A Study of Growth and
Decline.'' Elmsford NY/Oxford England Pergamon Press 1982.

\leavevmode\hypertarget{ref-wegener2004land}{}%
Wegener, Michael, and Franz Fürst. 2004. ``Land-Use Transport
Interaction: State of the Art.'' \emph{Available at SSRN 1434678}.
\url{https://doi.org/10.2139/ssrn.1434678}.

\leavevmode\hypertarget{ref-zahavi1981umot}{}%
Zahavi, Yacov, Martin J Beckmann, and Thomas F Golob. 1981. ``The
Umot/Urban Interactions.''

\leavevmode\hypertarget{ref-zmud2018updating}{}%
Zmud, Johanna, Tom Williams, Maren Outwater, Mark Bradley, Nidhi Kalra,
Shelley Row, National Cooperative Highway Research Program,
Transportation Research Board, and National Academies of Sciences,
Engineering, and Medicine. 2018. \emph{Updating Regional Transportation
Planning and Modeling Tools to Address Impacts of Connected and
Automated Vehicles, Volume 2: Guidance}. Washington, D.C.:
Transportation Research Board. \url{https://doi.org/10.17226/25332}.


\end{document}
