\documentclass[12pt,]{article}
\usepackage{lmodern}
\usepackage{amssymb,amsmath}
\usepackage{ifxetex,ifluatex}
\usepackage{fixltx2e} % provides \textsubscript
\ifnum 0\ifxetex 1\fi\ifluatex 1\fi=0 % if pdftex
  \usepackage[T1]{fontenc}
  \usepackage[utf8]{inputenc}
\else % if luatex or xelatex
  \ifxetex
    \usepackage{mathspec}
  \else
    \usepackage{fontspec}
  \fi
  \defaultfontfeatures{Ligatures=TeX,Scale=MatchLowercase}
    \setmainfont[]{Times New Roman}
\fi
% use upquote if available, for straight quotes in verbatim environments
\IfFileExists{upquote.sty}{\usepackage{upquote}}{}
% use microtype if available
\IfFileExists{microtype.sty}{%
\usepackage{microtype}
\UseMicrotypeSet[protrusion]{basicmath} % disable protrusion for tt fonts
}{}
\usepackage[margin=1in]{geometry}
\usepackage{hyperref}
\hypersetup{unicode=true,
            pdftitle={Policy Memo},
            pdfborder={0 0 0},
            breaklinks=true}
\urlstyle{same}  % don't use monospace font for urls
\usepackage{graphicx,grffile}
\makeatletter
\def\maxwidth{\ifdim\Gin@nat@width>\linewidth\linewidth\else\Gin@nat@width\fi}
\def\maxheight{\ifdim\Gin@nat@height>\textheight\textheight\else\Gin@nat@height\fi}
\makeatother
% Scale images if necessary, so that they will not overflow the page
% margins by default, and it is still possible to overwrite the defaults
% using explicit options in \includegraphics[width, height, ...]{}
\setkeys{Gin}{width=\maxwidth,height=\maxheight,keepaspectratio}
\IfFileExists{parskip.sty}{%
\usepackage{parskip}
}{% else
\setlength{\parindent}{0pt}
\setlength{\parskip}{6pt plus 2pt minus 1pt}
}
\setlength{\emergencystretch}{3em}  % prevent overfull lines
\providecommand{\tightlist}{%
  \setlength{\itemsep}{0pt}\setlength{\parskip}{0pt}}
\setcounter{secnumdepth}{0}
% Redefines (sub)paragraphs to behave more like sections
\ifx\paragraph\undefined\else
\let\oldparagraph\paragraph
\renewcommand{\paragraph}[1]{\oldparagraph{#1}\mbox{}}
\fi
\ifx\subparagraph\undefined\else
\let\oldsubparagraph\subparagraph
\renewcommand{\subparagraph}[1]{\oldsubparagraph{#1}\mbox{}}
\fi

%%% Use protect on footnotes to avoid problems with footnotes in titles
\let\rmarkdownfootnote\footnote%
\def\footnote{\protect\rmarkdownfootnote}

%%% Change title format to be more compact
\usepackage{titling}

% Create subtitle command for use in maketitle
\providecommand{\subtitle}[1]{
  \posttitle{
    \begin{center}\large#1\end{center}
    }
}

\setlength{\droptitle}{-2em}

  \title{Policy Memo}
    \pretitle{\vspace{\droptitle}\centering\huge}
  \posttitle{\par}
  \subtitle{Autonomous vehicles and land use}
  \author{}
    \preauthor{}\postauthor{}
      \predate{\centering\large\emph}
  \postdate{\par}
    \date{June 11}


\begin{document}
\maketitle

To: Scott Haggerty, chair of governing Commission of Bay Area
Metropolitan Transportation Commission

From: Shen Qu, Policy Advisor

Date: 6/11/2019

RE: How Bay Area should be planning for autonomous vehicles?

\hypertarget{summary}{%
\section{Summary}\label{summary}}

This memo is one of the series of policy analysis about autonomous
vehicles for Bay Area. It trys to answer how the autonomous vehicles
will affect urban land use, and What the MPO and cities should be
planning to seize this opportunity and address the challenges.

\hypertarget{background-explains-the-current-and-projected-status-of-avs.}{%
\section{Background: Explains the current and projected status of
AVs.}\label{background-explains-the-current-and-projected-status-of-avs.}}

\emph{Technology}: The history of exploring autonomous vehicles can date
back to late 1950s (Milakis 2019). Since DARPA ran the Grand Challenge
in 2004, the autonomous vehicle technologies entered a ``critical
juncture.'' (Docherty, Marsden, and Anable 2018) In the fields of
automation control systems, some critical hardwares like processors and
sensors are having the ability of undertaking more complex tasks. The
improving softwares and algorithm are becoming more fledged. According
the definition and taxonomy by (international 2018), the existing
technologies are achiving from the Level 3 - Conditional Driving
Automation to Level 4 - High Driving Automation.\footnote{SAE defines
  the concept of AV as ADS-DV (ADS-Dedicated Vehicle), ``A vehicle
  designed to be operated exclusively by a level 4 or level 5 ADS for
  all trips within its given Operational Design Domain (ODD)
  limitations.'' ADS means ``The hardware and software that are
  collectively capable of performing the entire Dynamic Driving Task
  (DDT) on a sustained basis, regardless of whether it is limited to a
  specific ODD; this term is used specifically to describe a level 3, 4,
  or 5 driving automation system.''}

\emph{Industry}: Since 2009, Google had conducted a series of tests for
AVs over 10 million miles on real-world roads in California, Texas, and
other states. Waymo, a company founded by Google, hold the only testing
permit for driverless testing by California DMV and committed to
providing a ride-hailing services in Arizona in 2018. In 2019, Waymo
announced their Level 4 AV will be assembled in Detroit. Almost all big
automakers such as Ford, General Motors, Volkswagen, etc., are investing
heavily in this field.{[}Fagnant \& kockelman, 2015{]} The leading
transportation network companys (TNC) like Uber and Lyft are making up a
term of transportation-as-a-service (Taas) to change current travel
modes by AVs.

\emph{Academia}: Many scholars start working on the reaserch of AVs with
lots of energy. Gandia et al. (2019)`s reserch found 10580 published
papers in this field from 1945 to 2018. Since 2012, the number of
articles have an exponential growth with 39\% growth rate while 8-9\%
average growth rate in science. Although a large amount of the research
are from the perspective of systems control, computer science, robotics,
engineering, there are more and more articls that start to focus on the
AVs' impact on transportation and Land use.

\emph{Governavce}: In 2011, the Nevada Department of Motor Vehicles
issued a first license to google's experimental AVs. Currently, ``33
states have passed legislation related to AV.'' ``15 states enacted 18
AV related bills.'' (State Legislatures n.d.) From 2016 to 2018 the U.S.
Department of Transportation (U.S.DOT) and the National Highway and
Transportation Safety Administration (NHTSA) published three federal
guidences for Automated Driving Systems (ADS is the definition by
SAE\footnote{SAE defines the concept of AV as ADS-DV (ADS-Dedicated
  Vehicle), ``A vehicle designed to be operated exclusively by a level 4
  or level 5 ADS for all trips within its given Operational Design
  Domain (ODD) limitations.'' ADS means ``The hardware and software that
  are collectively capable of performing the entire Dynamic Driving Task
  (DDT) on a sustained basis, regardless of whether it is limited to a
  specific ODD; this term is used specifically to describe a level 3, 4,
  or 5 driving automation system.''}). The guidances advocate industry,
state and local government to support the AVs' development. (Highway and
Administration n.d.) California is playing a leading role in this field.
Until January 2019, California DMV has issued AV Testing Permits (with a
driver) to 62 companies on public roadways.

What are the crutial effects? What are the crutial response for land
use?

\hypertarget{changes-discusses-how-avs-could-influence-demand-for-transportation-and-in-turn-land-use.-for-this-analysis-use-both-theory-and-research.}{%
\section{Changes: Discusses how AVs could influence demand for
transportation and, in turn land use. For this analysis, use both theory
and
research.}\label{changes-discusses-how-avs-could-influence-demand-for-transportation-and-in-turn-land-use.-for-this-analysis-use-both-theory-and-research.}}

\hypertarget{focus-on-relevant-changes}{%
\subsection{focus on relevant changes}\label{focus-on-relevant-changes}}

(Milakis, Van Arem, and Van Wee 2017) Many substantial implications of
autonomous vehicles are not considered in this memo, such as safety,
liability, and etc. {[}riple effects{]}. This analysis focus on the
impact on land use, wich have short-term and long-term influences. The
short-term influences include che change of parking, urban design,
affected by travel demand and behavior

Diamond of Assembly by(Milakis, Van Arem, and Van Wee 2017) and feedback
cylce by {[}soteropoulos2019impacts{]}

The long-term influences include the reconstructure of urban forms and
spatial distributions.

\hypertarget{essence}{%
\subsection{essence}\label{essence}}

Four stage

Theory: bid-rent theroy, utility maximize.

Research: Identifies the benefits and costs of these possible outcomes.

focus on Characteristic:

cut off labor cost,

round-the-clock services.

full ridesharing by realtime matching

Time cost (Singleton 2019)

\hypertarget{methodology}{%
\subsection{Methodology}\label{methodology}}

high risk in use sufficient principle, and likelihood principle. another
option is convariance principle.

Internet, Air Transport system, TNC

\hypertarget{inference}{%
\subsection{inference:}\label{inference}}

Behavior and land use (Soteropoulos, Berger, and Ciari 2019)

(Hawkins and Nurul Habib 2019)

previous reaserch had give many estimation of the change on road
capacities, parking lots, curve space.

use cost and transaction costs - full match

deals fail

\hypertarget{the-short-term-respose}{%
\section{The short-term Respose}\label{the-short-term-respose}}

(Legacy et al. 2019)

Presents policy and planning options for mitigating or otherwise
addressing the possible land use effects.

designating pilot area

\hypertarget{housing}{%
\subsection{housing,}\label{housing}}

\hypertarget{parking}{%
\subsection{parking,}\label{parking}}

\hypertarget{urban-design}{%
\subsection{urban design}\label{urban-design}}

\hypertarget{the-strategic-planning}{%
\section{The strategic planning}\label{the-strategic-planning}}

long-term effects (Milakis 2019)

Discusses how the MPO and cities may need alter the tools and analyses
they use to consider AVs.

Zoning, Division, and partion, not uniform

\hypertarget{conclusion}{%
\section{Conclusion}\label{conclusion}}

overestimated and under estimate

from link to node

CA should play a leading role. responsibility

\hypertarget{notes}{%
\section{Notes}\label{notes}}

\hypertarget{references}{%
\section*{References}\label{references}}
\addcontentsline{toc}{section}{References}

\hypertarget{refs}{}
\leavevmode\hypertarget{ref-docherty2018governance}{}%
Docherty, Iain, Greg Marsden, and Jillian Anable. 2018. ``The Governance
of Smart Mobility.'' \emph{Transportation Research Part A: Policy and
Practice} 115. Elsevier: 114--25.
\url{https://doi.org/10.1016/j.tra.2017.09.012}.

\leavevmode\hypertarget{ref-gandia2019autonomous}{}%
Gandia, Rodrigo Marçal, Fabio Antonialli, Bruna Habib Cavazza, Arthur
Miranda Neto, Danilo Alves de Lima, Joel Yutaka Sugano, Isabelle
Nicolai, and Andre Luiz Zambalde. 2019. ``Autonomous Vehicles:
Scientometric and Bibliometric Review.'' \emph{Transport Reviews} 39
(1). Taylor \& Francis: 9--28.
\url{https://doi.org/10.1080/01441647.2018.1518937}.

\leavevmode\hypertarget{ref-hawkins2019integrated}{}%
Hawkins, Jason, and Khandker Nurul Habib. 2019. ``Integrated Models of
Land Use and Transportation for the Autonomous Vehicle Revolution.''
\emph{Transport Reviews} 39 (1). Taylor \& Francis: 66--83.
\url{https://doi.org/10.1080/01441647.2018.1449033}.

\leavevmode\hypertarget{ref-NHTSA2019ADS}{}%
Highway, National, and Transportation Safety Administration. n.d.
``Automated Driving Systems.'' Accessed May 27, 2019.
\url{https://www.nhtsa.gov/vehicle-manufacturers/automated-driving-systems}.

\leavevmode\hypertarget{ref-sae2018taxonomy}{}%
international, SAE. 2018. ``Taxonomy and Definitions for Terms Related
to Driving Automation Systems for on-Road Motor Vehicles.'' \emph{SAE
International,(J3016)}.

\leavevmode\hypertarget{ref-legacy2019planning}{}%
Legacy, Crystal, David Ashmore, Jan Scheurer, John Stone, and Carey
Curtis. 2019. ``Planning the Driverless City.'' \emph{Transport Reviews}
39 (1). Taylor \& Francis: 84--102.
\url{https://doi.org/10.1080/01441647.2018.1466835}.

\leavevmode\hypertarget{ref-milakis2019long}{}%
Milakis, Dimitris. 2019. ``Long-Term Implications of Automated Vehicles:
An Introduction.'' Taylor \& Francis.
\url{https://doi.org/10.1080/01441647.2019.1545286}.

\leavevmode\hypertarget{ref-milakis2017policy}{}%
Milakis, Dimitris, Bart Van Arem, and Bert Van Wee. 2017. ``Policy and
Society Related Implications of Automated Driving: A Review of
Literature and Directions for Future Research.'' \emph{Journal of
Intelligent Transportation Systems} 21 (4). Taylor \& Francis: 324--48.
\url{https://doi.org/10.1080/15472450.2017.1291351}.

\leavevmode\hypertarget{ref-singleton2019discussing}{}%
Singleton, Patrick A. 2019. ``Discussing the `Positive Utilities' of
Autonomous Vehicles: Will Travellers Really Use Their Time
Productively?'' \emph{Transport Reviews} 39 (1). Taylor \& Francis:
50--65. \url{https://doi.org/10.1080/01441647.2018.1470584}.

\leavevmode\hypertarget{ref-soteropoulos2019impacts}{}%
Soteropoulos, Aggelos, Martin Berger, and Francesco Ciari. 2019.
``Impacts of Automated Vehicles on Travel Behaviour and Land Use: An
International Review of Modelling Studies.'' \emph{Transport Reviews} 39
(1). Taylor \& Francis: 29--49.
\url{https://doi.org/10.1080/01441647.2018.1523253}.

\leavevmode\hypertarget{ref-NCSL2019AV}{}%
State Legislatures, National Conference of. n.d. ``Autonomous Vehicles
\textbar{} Self-Driving Vehicles Enacted Legislation.'' Accessed March
19, 2019.
\url{http://www.ncsl.org/research/transportation/autonomous-vehicles-self-driving-vehicles-enacted-legislation.aspx}.


\end{document}
