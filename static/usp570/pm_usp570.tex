\documentclass[12pt,]{article}
\usepackage{lmodern}
\usepackage{amssymb,amsmath}
\usepackage{ifxetex,ifluatex}
\usepackage{fixltx2e} % provides \textsubscript
\ifnum 0\ifxetex 1\fi\ifluatex 1\fi=0 % if pdftex
  \usepackage[T1]{fontenc}
  \usepackage[utf8]{inputenc}
\else % if luatex or xelatex
  \ifxetex
    \usepackage{mathspec}
  \else
    \usepackage{fontspec}
  \fi
  \defaultfontfeatures{Ligatures=TeX,Scale=MatchLowercase}
    \setmainfont[]{Times New Roman}
\fi
% use upquote if available, for straight quotes in verbatim environments
\IfFileExists{upquote.sty}{\usepackage{upquote}}{}
% use microtype if available
\IfFileExists{microtype.sty}{%
\usepackage{microtype}
\UseMicrotypeSet[protrusion]{basicmath} % disable protrusion for tt fonts
}{}
\usepackage[margin=1in]{geometry}
\usepackage{hyperref}
\hypersetup{unicode=true,
            pdftitle={Policy Memo},
            pdfborder={0 0 0},
            breaklinks=true}
\urlstyle{same}  % don't use monospace font for urls
\usepackage{longtable,booktabs}
\usepackage{graphicx,grffile}
\makeatletter
\def\maxwidth{\ifdim\Gin@nat@width>\linewidth\linewidth\else\Gin@nat@width\fi}
\def\maxheight{\ifdim\Gin@nat@height>\textheight\textheight\else\Gin@nat@height\fi}
\makeatother
% Scale images if necessary, so that they will not overflow the page
% margins by default, and it is still possible to overwrite the defaults
% using explicit options in \includegraphics[width, height, ...]{}
\setkeys{Gin}{width=\maxwidth,height=\maxheight,keepaspectratio}
\IfFileExists{parskip.sty}{%
\usepackage{parskip}
}{% else
\setlength{\parindent}{0pt}
\setlength{\parskip}{6pt plus 2pt minus 1pt}
}
\setlength{\emergencystretch}{3em}  % prevent overfull lines
\providecommand{\tightlist}{%
  \setlength{\itemsep}{0pt}\setlength{\parskip}{0pt}}
\setcounter{secnumdepth}{0}
% Redefines (sub)paragraphs to behave more like sections
\ifx\paragraph\undefined\else
\let\oldparagraph\paragraph
\renewcommand{\paragraph}[1]{\oldparagraph{#1}\mbox{}}
\fi
\ifx\subparagraph\undefined\else
\let\oldsubparagraph\subparagraph
\renewcommand{\subparagraph}[1]{\oldsubparagraph{#1}\mbox{}}
\fi

%%% Use protect on footnotes to avoid problems with footnotes in titles
\let\rmarkdownfootnote\footnote%
\def\footnote{\protect\rmarkdownfootnote}

%%% Change title format to be more compact
\usepackage{titling}

% Create subtitle command for use in maketitle
\providecommand{\subtitle}[1]{
  \posttitle{
    \begin{center}\large#1\end{center}
    }
}

\setlength{\droptitle}{-2em}

  \title{Policy Memo}
    \pretitle{\vspace{\droptitle}\centering\huge}
  \posttitle{\par}
  \subtitle{Accessibility measures for regional transportation planning}
  \author{}
    \preauthor{}\postauthor{}
      \predate{\centering\large\emph}
  \postdate{\par}
    \date{May 2 - June 11}


\begin{document}
\maketitle

To: Scott Haggerty, chair of governing Commission of Bay Area
Metropolitan Transportation Commission

From: Shen Qu, Policy Advisor

Date: 5/2/2019

RE: Define and measure accessibility in Plan Bay are 2050

\hypertarget{summary}{%
\section{Summary}\label{summary}}

Accessibility, the ease of reaching destinations, is a key land use and
transportation performance measure(Boisjoly and El-Geneidy 2017b).
Including clear accessibility objectives, definitions, and indicators in
Plan Bay Area 2050 is key to promoting the use of metrics in policy and
practice.

multi-criteria analysis approaches including accessibility indicators
need to guide the decision-making process and ultimately. To meet The
Bay Area's transportation needs through 2050 while preserving the
character of its diverse communities and adapting to future population
growth. To make progress toward the Bay Area long-range transportation
and land use goals.

(Boisjoly and El-Geneidy 2017a)

equity analysis greater accessibility reducing the risks of social
exclusion for vulnerable individuals \emph{(Preston and Rajé, 2007,
Lucas, 2012)}

Accessibility goals and objectives

accessibility indicators

Access to destinations metrics

Multiple modes

Visualization tools

\hypertarget{background}{%
\section{Background}\label{background}}

\begin{itemize}
\tightlist
\item
  Plan Bay Area 2050: they are going to update their regional
  transportation plan, along with their processes for selecting projects
  for funding, at all scales.
\end{itemize}

Plan Bay Area 2050 is an integrated long-range transportation and land
use plan developed and adopt by MTC and ABAG. Work on Plan Bay Area 2050
is expected to begin in August 2019 and focused update that builds upon
the growth pattern and strategies developed in the original Plan Bay
Area 2040 (adopted in July 2017) but with updated planning assumptions
that incorporate key economic, demographic and financial trends from the
last four years.

Re-examining the prior Plan's goals and targets is one of the first
steps in updating Plan Bay Area in order to make them as meaningful as
possible in measuring the Plan's performance. In the current 13
performance targets, three goals are about `Equitable Access'. It is
also a response to the most recent federal transportation bills -- the
Fixing America's Surface Transportation Act (FAST Act) that require Plan
Bay Area to address `accessibility and mobility of people and freight'
(U.S. Department of Transportation, 2014).

Although the three targets are components of accessibility, it does not
fully reflect access to destinations. In spite of access afordable
housing, a broader range of destinations is generally included

accessibility is one of the most comprehensive performance measures of
land use and transportation systems \emph{(El-Geneidy and Levinson,
2006)}.

is associated with higher employment rates, provides residents with
greater access to a variety of opportunities\emph{(Ornati et al., 1969,
Pignatar and Falcocch, 1969, Sanchez, 1999, Blumenberg and Ong, 2001,
Sari, 2015, Tyndall, 2015)}, reducing car use greater transit use
\emph{(Chen et al., 2008, Owen and Levinson, 2015b)},and the resulting
greenhouse gas emissions \emph{(Levinson, 1998, Handy, 2002)}.

accessibility allows increaseing the chances of meeting the other
performance targets in Plan Bay Area, including Climate Protection,
Healthy and Safe Communities, Open Space and Agricultural Preservation,
Economic Vitality, Transportation System Effectiveness, and should hence
also be used to assess the overall benefits of potential investments.

The accessibility-based objective is not only to improve social
inclusion and reduce social inequities, but also to support and foster
economic development through improved employment accessibility for
deprived areas. accessibility indicators should be used as general
performance indicators.

the concept of accessibility has been a common element in the goals and
objectives of transport plans Hansen (1959).

However, the federal guidelines and prior plan doesn't define access
explicitly. The terms accessibility and mobility are used as a vague
term that does not reflect the ease of reaching various destinations and
often are conflated, ``misused'', and ``abused'' in practice (Halden,
2011). Access to destinations does not translate into indicators that
reflect accessibility in previous plans.

Explicitly defining accessibility as the end goal of the transportation
network would encourage the establishment of accessibility-based
performance indicators and help policy-making for MTC, ABAG, and local
governments.

\hypertarget{definition-why-it-should-be-used-in-transportation-decision-making}{%
\section{Definition: why it should be used in transportation decision
making}\label{definition-why-it-should-be-used-in-transportation-decision-making}}

Accessibility, defined as the ease of reaching destinations,services,
and activities; the distribution, character, and amount of activity
around a given place; the choices that the built environment offers to
travelers\emph{(Preston and Rajé, 2007; Litman, 2013; Handy (2018))},

for individual, living in an area with relatively high accessibility to
jobs is associated with shorter trips, as is working in an area of
relatively high housing accessibility.9

for firms: e.g.~access to suppliers, workforce, customers, desirable
environment (aesthetically pleasing surroundings, clean air and
water),amenities (access to recreational and other non-work
destinations), friendly government, etc. (Levinson and Krizek 2018, 111)

Accessibility is instrumental in explaining the spatial form and
function of metropolitan areas. It is the ``fundamental force for the
relationship between transport and land use in a city and their impacts
on city organization, development, and planning to achieve more
sustainable outcomes.''(Levinson and Krizek 2018, 22) It is one of the
most comprehensive measures to assess the complex performance of land
use and transportation systems in a region.(Boisjoly and El-Geneidy
2017a), to assess current conditions and proposed policies

Moreover, It's a goal that almost everyone can agree on, and it opens
doors to a host of strategies that could reduce auto dependence and
improve quality of life. (Handy 2018)

\hypertarget{evaluates-the-accessibility-measures}{%
\section{Evaluates the accessibility
measures}\label{evaluates-the-accessibility-measures}}

use accessibility measures to define regional transportation priorities

is largely contingent on the spatial distribution of destinations,
influenced by the distribution of residential, economic, cultural and
social activities.

related to the spatial distribution of opportunities. such as
households, jobs, retail stores, health and other services.

\begin{itemize}
\tightlist
\item
  cumulative-opportunity measures: the availability of opportunities
  close by
\end{itemize}

(counting exchange opportunities within a defined geographic parameter)

cumulative-opportunity measures typically accounts for the number of
opportunities that can be reached from a specific location using a
specific mode within a travel costs threshold (Handy and Niemeier,
1997). For example, the number of jobs that are within 45 min of travel
times by transit from a specific place is used to assess the access to
jobs by public transit.

cumulative-opportunity measures are easy to generate, interpret, and to
communicate. they are most commonly used by policy-makers as they
provide a comprehensive measure of the land use and transport system at
the regional level (Dodson et al., 2007). and thus better suited for
planning documents

\begin{itemize}
\tightlist
\item
  gravity-based measure: the ease and worth of travel to destinations
  far away
\end{itemize}

(pitting the importance of given exchange opportunities against travel
time impedances),

the gravity-based measure discounts opportunities based on a
distance-decay function.based on their travel costs. Accordingly,
opportunities that are located farther (by distance or time) receive
less weight than closer opportunities.

this measure better reflect travel behavior as it accounts for the
travelers' perceptions of time (Ben-Akiva and Lerman, 1979).

This measure is not directly expressed in terms of the number of
opportunities. is more complex to generate, and more difficult to
interpret and communicate (Geurs and van Wee, 2004; A. Owen and
Levinson, 2014).

Integrating both cumulative-opportunity and gravity-based measure,
including access to transport and to destinations, would provide a good
indication of transport coverage, captures the performance of the land
use and transportation systems, which better reflect the social and
economic benefits (Banister, 2008, Koenig, 1980, Wachs and Kumagai,
1973).

\begin{longtable}[]{@{}lll@{}}
\toprule
& cumulative-opportunity measures & gravity-based measure\tabularnewline
\midrule
\endhead
access to destinations & &\tabularnewline
Access to public transport & &\tabularnewline
\bottomrule
\end{longtable}

\begin{itemize}
\tightlist
\item
  the evaluation of Local accessibility.
\end{itemize}

A new transport infrastructure improvement increases the relative
accessibility of some places that can directly use the facility

Local accessibility is primarily determined by nearby activity
(approximately one-half to one mile (800 to 1,600 m) in residential
areas). include to grocery stores, schools, parks or public transport
station for example libraries, schools, grocery stores, hospitals,
public parks, educational services Many of these destinations reflect
local accessibility and are thus often associated with cycling and
walking.

walking and cycling distance thresholds are used instead of travel time
thresholds (0.5 miles for walking). These appear to be appropriate
measures of accessibility, as time is generally proportional to the
distance travelled by bicycle or foot.

Access to jobs can also be a reflection of the level of services
available around a certain location, as the delivery of services often
equates a certain number of employees.

Neighborhood accessibility policy initiatives speak more to issues of
mixing uses on a parcel or neighborhood scale, site design, and more
directly, facilitating circulation patterns that enhance walking,
bicycling, and transit use.

Analysing future development scenarios in terms of their levels of
accessibility could leverage MPOs' influence on municipal-level land-use
decisions.

Such comparisons could help MPOs provide better information about the
performance and the costs of different transportation-infrastructure and
land-use scenarios.

Better information about the tradeoffs inherent in different development
scnearios can help regions choose projects more objectively.

\hypertarget{equity-analysis}{%
\section{equity analysis}\label{equity-analysis}}

Discusses how equity could be incorporated in the accessibility
measures. In doing so, how are you defining equity?

Equity analysis based on accessibility indicators generally assess the
level of accessibility of specific vulnerable groups relatively the
general population, using detailed accessibility metrics. Proximity to
Services, Amenities, and Opportunity Areas

Promoting access to housing, jobs and transportation for these residents
helps Plan Bay Area's objective to advance equity in the region;

Equity analysis can include specific types of destinations, modes, jobs,
social groups, or Temporal fluctuations.

Time restrictions also play an important role in determining
accessibility. These include land use, transport and individual
constraints such as (opening hours, schedule of services, job starting
time)

Increasing accessibility by transit, cycling and walking can contribute
to achieving broader environmental, economic and social goals. Such as
the percentage of people or jobs that are within 0.5 mile of a public
transport station.

the accessibility to all jobs may not represent the opportunities that
are available to different groups of populations. segment the
accessibility analysis by socio-economic groups. such as income, level
of education, gender and vehicle ownership affect one's abilities and
needs to access destinations.

Measures of generalized costs (including financial and time costs).
These measures better reflect the total costs of travel as they include
both financial and time burdens. is closer to reality and can also
provide an insight on fare structures and trip affordability. address
the financial constraints that vulnerable individuals may face.

An equity analysis that assesses the distribution of benefits and
burdens on communities of concern in comparison to the rest of the
region.

accessibility metrics is about the environmental justice assessment

``Environmental justice is the fair treatment and meaningful involvement
of all people regardless of race, color, national origin, or income with
respect to the development, implementation, and enforcement of
environmental laws, regulations, and policies.''

An environmental justice analysis of investments to determine whether
there are any disproportionately high and adverse impacts on low-income
and minority populations or communities of concern;

\hypertarget{more-options}{%
\section{More options}\label{more-options}}

Accessibility maps and metrics provide an overview of the land use and
transportation network. are useful tools to communicate gaps and thus
helps decision-makers, planners and the general population to better
grasp the impacts of transportation investments.

the utility-based measures capture the economic benefits provided by
changes in the network. Utility-based measures account for most
components of accessibility and can be included in traditional
cost-benefit analysis (van Wee, 2016).

other dimensions of accessibility might currently be neglected in
metropolitan transportation plans. For example, affordability, transfer
and digital connectivity Lyons and Davidson (2016) should come up as the
aspects of accessibility objectives.

More efforts are needed to effectively implement accessibility-based
approaches.

\hypertarget{conclusion}{%
\section{Conclusion}\label{conclusion}}

Plan Bay Area 2050 sets a roadmap for future transportation investments
and will identify what it would take to accommodate expected growth. It
is neccessary to define and measure accessibility in updating Plan Bay
area 2050.

Accessibility, the ease of reaching destinations, allows capturing the
complex interactions between land use and transportation systems while
providing a social perspective on transportation planning.

Integrating RTP goals with accessibility-focused performance measures
could help MPOs make better decisions about the selection and
prioritisation of transportation infrastructure. For instance,
prioritising transit improvements to connect key origins and
destinations can increase ridership \emph{(Badoe and Miller, 2000;
Chakraborty and Mishra, 2013)}

Regional and local governments can play a key role in setting clear
accessibility requirements for transportation planning processes and
planning documents. (Boisjoly and El-Geneidy 2017b)

Clearly defined accessibility goals and objectives are included in the
plan. and translated into performance indicators,Accessibility
indicators are systematically included in the performance analyses, to
ultimately derive policy recommendations.

metrics are used to assess the general performance of the land use and
transportation system, in addition to social equity.for various modes of
transport

Overall, using clearly defined indicators, provide greater transparency
and typically foster the inclusion of accessibility aspect in the
decision-making process.

\hypertarget{distinguishing-accessibility-and-mobility}{%
\section{distinguishing Accessibility and
Mobility}\label{distinguishing-accessibility-and-mobility}}

travel is a derived demand \emph{(Goldman and Gorham, 2006; Grengs et
al., 2010; Halden, 2002; Handy and Niemeier, 1997; Levine et al.,
2012)}. people travel to places where they can meet their daily needs,
not simply to move about. the purpose of most travel is about the
destination, not the journey.

accessibility focuses on reaching destinations, the end goal of a
transportation network, focused on making it easier for individuals to
reach destinations where they can meet daily needs such as work,
recreation, socialising, shopping, and other forms of social exchange
\emph{(Martens, 2015; Miller, 2005)}.

Mobility is concerned with how easy it is to travel. focuses on travel
speeds. Improvements in mobility alone are not sufficient to ensure
improvements in accessibility. (Proffitt et al. 2019)

Planning for accessibility, Land use policies designed to bring
destinations within walking distance of residential areas. may not even
require retrofitting neighborhoods. For example, transit services that
link specific groups of users to their desired destinations, such as
reverse commute programs and other client-based transport services,
Efforts like these reduce the need to drive,.25

it facilitates the evaluation of tradeoffs between land use,
transportation and social needs. By combining aspects of land use and
transportation into a single measure, accessibility focuses attention on
the performance of the system as a whole rather than on just segments of
the transportation network.

the clear distinction between mobility and accessibility indicators.
Interestingly, accessibility and mobility are included as two distinct
goals with different criteria and methodologies,

Having shorter travel times does not necessarily equate to having access
to a larger number of destinations. Furthermore, as discussed by Litman
(2013), strategies aiming at increasing traffic speed may in some cases
lead to an overall reduction in accessibility. In sum, increased
mobility does not always result in increased accessibility (Halden,
2011, Levine et al., 2012).

`traditional transport planning, which tends to focus on improvements to
the transport system that facilitate mobility, without considering the
access needs that drive travel behaviour' (Chapman and Weir, 2008: 7).

Accessibility is increasingly seen as an alternative to mobility
oriented planning paradigm (Geurs et al., 2012), as it allows capturing
the complex interactions between land use and transportation systems
(Hansen, 1959) and provides a social perspective on transportation
planning (Banister, 2008, Lucas, 2012). While mobility merely reflects
the ease of moving, accessibility addresses the ease of reaching desired
destinations, which is in fact the reason why people undertake trips
(Preston and Rajé, 2007).

Planning for mobility has taken on the meaning of making it easier to
get around. This focus on the ease of traveling along the transport
network itself (rather than focusing on the ease of reaching
destinations) has aligned well with modern planning paradigms; road
building has been the most popular solution to congestion. These
paradigms prize the planning-for-mobility perspective because it
accommodates growing levels of travel and increases the potential for
movement.

conventional practice in transportation planning has employed a `predict
and provide' model that focuses the majority of funding and planning
attention on expanding roadway capacity. (Levinson and Krizek 2018, 22)

performance metrics that focus on roadway congestion as experienced by
automobile drivers \emph{(Ewing, 1993; Handy, 2005; Krizek and Levinson,
2010)}, with higher vehicle speeds the `fundamental criterion for
success' \emph{(Levine et al., 2012: 158)}. Speed-based metrics include
roadway level of service (LOS), peak-period delay, traffic volume/road
capacity, travel time/speed, vehicle hours of travel, the duration of
peak-period congestion, and others \emph{(Ewing, 1996; Transportation
Research Board, 2010)}. Even high-occupancy toll lanes, the most common
demand-management strategy used in the USA, are typically added as new
capacity, only rarely replacing existing highway travel lanes
\emph{(Ewing and Proffitt, 2016)}. Planning for higher travel speeds
that facilitate longer and often more frequent trips is the mobility
paradigm.

improving mobility -- via automobile, transit, or any other travel mode
-- means facilitating faster travel speeds so individuals can reach more
destinations in a given travel time, improving proximity means
shortening distances between trip origins and destinations so
individuals can reach a satisfactory number of exchange opportunities
even if they travel more slowly. In other words, neighbourhoods, cities,
and metro areas can be made more accessible by reducing travel distances
as well as by facilitating faster travel. The advantage of planning for
accessibility versus planning solely for (auto)mobility is that the
former allows for a comparison of the tradeoffs among financial,
environmental and human health and wellbeing concerns when making
decisions about land use and transportation.

Accessibility does not depend on good mobility.Some places such as San
Francisco downtown have good accessibility despite having poor mobility
(severe traffic congestion). The Residents live within a short distance
of all needed and desired destinations. the travel times between
destinations are relatively short, even if travel speeds are low.

In the suburban areas of Bay Area, transit service is relatively sparse
and destinations are generally beyond walking distance, leaving
residents with no option but to drive. the practice of planning is
largely mobility-dependent, and car-dependent and has deteriorated
levels of accessibility. As traffic levels invariably increase in these
areas, accessibility ultimately declines for all modes.

(Proffitt et al. 2019) increasingly tight budgets and a growing
awareness that it simply is not possible for regions to pave their way
out of congestion do seem to be pushing many MPOs to look for
alternatives to expanding roadway capacity. Improving accessibility by
coordinating land-use and transportation rather than an exclusive focus
on automobility is one such alternative.

\hypertarget{multi-criteria-analyses}{%
\section{multi-criteria analyses}\label{multi-criteria-analyses}}

accessibility indicators should systematically be included in
multi-criteria analyses.

it offers an alternative to mobility-based decisions and potentially
provide greater transparency in the decision-making process (Halden,
2011). Furthermore, national and regional authorities can require local
authorities to address accessibility in their project analysis. One
especially effective way of doing so is by including accessibility
criteria in the selection process of projects,

One of the most systematic and transparent way to inform decision-making
is by including accessibility indicators into multi-criteria analyses. a
multi-criteria analysis was conducted to compare the projects submitted
by local jurisdictions and to select the ones to be included in the RTP.
to conduct a prioritization of the projects to assess the effectiveness
of various modelling scenarios.

The accessibility indicators included in the multi-criteria analyses
range from broad questions to specific quantified metrics, which
influence the flexibility of the analysis. For example, Transport for
London defines clear specific accessibility metrics, for example the
change in the number of jobs accessible by public transport within 45
minutes travel time (see Table 1). These access to jobs metrics are
relatively easy to generate and to interpret. Accordingly, they foster
the inclusion of accessibility indicators that adequately reflect the
ease of reaching destinations. Furthermore, given their specific nature,
they are easy to communicate as exemplified in the plan: ``Implementing
the schemes will increase the employment catchment area of central
London (the number of people within 45 minutes of central London
employment) by almost 25 percent.'' (p.74).

An intermediate way of defining accessibility indicators is by
attributing scores (from 1 to 3 for example) based on specific
guidelines. This approach has the advantage of defining clear weights
associated with accessibility criteria, thus providing greater
transparency.

quantified metrics provide more specific guidelines that directly
reflect the ease of reaching destinations. However, they provide lower
flexibility and might not adequately reflect the outcomes of the
different investments.

\hypertarget{notes}{%
\section{Notes}\label{notes}}

Decrease the share of lower-income residents' household income consumed
by transportation and housing by 10\%.

Increase the share of affordable housing in PDAs, TPAs, or
high-opportunity areas by 15\%.

Do not increase the share of low- and moderate-income renter households
in PDAs, TPAs, or high-opportunity areas that are at risk of
displacement.

\hypertarget{references}{%
\section*{References}\label{references}}
\addcontentsline{toc}{section}{References}

\hypertarget{refs}{}
\leavevmode\hypertarget{ref-boisjoly2017get}{}%
Boisjoly, Geneviève, and Ahmed M El-Geneidy. 2017a. ``How to Get There?
A Critical Assessment of Accessibility Objectives and Indicators in
Metropolitan Transportation Plans.'' \emph{Transport Policy} 55.
Elsevier: 38--50.

\leavevmode\hypertarget{ref-boisjoly2017insider}{}%
---------. 2017b. ``The Insider: A Planners' Perspective on
Accessibility.'' \emph{Journal of Transport Geography} 64. Elsevier:
33--43.

\leavevmode\hypertarget{ref-handy2018enough}{}%
Handy, Susan. 2018. ``Enough with the `Ds' Already---Let's Get Back to
`a'.'' Transfers Magazine.
\url{https://transfersmagazine.org/enough-with-the-ds-already-lets-get-back-to-a/}.

\leavevmode\hypertarget{ref-hansen1959accessibility}{}%
Hansen, Walter G. 1959. ``How Accessibility Shapes Land Use.''
\emph{Journal of the American Institute of Planners} 25 (2). Taylor \&
Francis: 73--76.

\leavevmode\hypertarget{ref-levinson2018metropolitan}{}%
Levinson, David M, and Kevin J Krizek. 2018. \emph{Metropolitan Land Use
and Transport: Planning for Place and Plexus}. Routledge.
\url{https://doi.org/10.4324/9781315684482}.

\leavevmode\hypertarget{ref-litman2017evaluating}{}%
Litman, Todd. 2017. \emph{Evaluating Accessibility for Transport
Planning}. Victoria Transport Policy Institute.
\url{http://www.vtpi.org/access.pdf}.

\leavevmode\hypertarget{ref-lyons2016guidance}{}%
Lyons, Glenn, and Cody Davidson. 2016. ``Guidance for Transport Planning
and Policymaking in the Face of an Uncertain Future.''
\emph{Transportation Research Part A: Policy and Practice} 88. Elsevier:
104--16.

\leavevmode\hypertarget{ref-proffitt2019accessibility}{}%
Proffitt, David G, Keith Bartholomew, Reid Ewing, and Harvey J Miller.
2019. ``Accessibility Planning in American Metropolitan Areas: Are We
There yet?'' \emph{Urban Studies} 56 (1). SAGE Publications Sage UK:
London, England: 167--92.


\end{document}
