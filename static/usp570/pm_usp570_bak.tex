\documentclass[12pt,]{article}
\usepackage{lmodern}
\usepackage{amssymb,amsmath}
\usepackage{ifxetex,ifluatex}
\usepackage{fixltx2e} % provides \textsubscript
\ifnum 0\ifxetex 1\fi\ifluatex 1\fi=0 % if pdftex
  \usepackage[T1]{fontenc}
  \usepackage[utf8]{inputenc}
\else % if luatex or xelatex
  \ifxetex
    \usepackage{mathspec}
  \else
    \usepackage{fontspec}
  \fi
  \defaultfontfeatures{Ligatures=TeX,Scale=MatchLowercase}
    \setmainfont[]{Times New Roman}
\fi
% use upquote if available, for straight quotes in verbatim environments
\IfFileExists{upquote.sty}{\usepackage{upquote}}{}
% use microtype if available
\IfFileExists{microtype.sty}{%
\usepackage{microtype}
\UseMicrotypeSet[protrusion]{basicmath} % disable protrusion for tt fonts
}{}
\usepackage[margin=1in]{geometry}
\usepackage{hyperref}
\hypersetup{unicode=true,
            pdftitle={Policy Memo},
            pdfborder={0 0 0},
            breaklinks=true}
\urlstyle{same}  % don't use monospace font for urls
\usepackage{graphicx,grffile}
\makeatletter
\def\maxwidth{\ifdim\Gin@nat@width>\linewidth\linewidth\else\Gin@nat@width\fi}
\def\maxheight{\ifdim\Gin@nat@height>\textheight\textheight\else\Gin@nat@height\fi}
\makeatother
% Scale images if necessary, so that they will not overflow the page
% margins by default, and it is still possible to overwrite the defaults
% using explicit options in \includegraphics[width, height, ...]{}
\setkeys{Gin}{width=\maxwidth,height=\maxheight,keepaspectratio}
\IfFileExists{parskip.sty}{%
\usepackage{parskip}
}{% else
\setlength{\parindent}{0pt}
\setlength{\parskip}{6pt plus 2pt minus 1pt}
}
\setlength{\emergencystretch}{3em}  % prevent overfull lines
\providecommand{\tightlist}{%
  \setlength{\itemsep}{0pt}\setlength{\parskip}{0pt}}
\setcounter{secnumdepth}{0}
% Redefines (sub)paragraphs to behave more like sections
\ifx\paragraph\undefined\else
\let\oldparagraph\paragraph
\renewcommand{\paragraph}[1]{\oldparagraph{#1}\mbox{}}
\fi
\ifx\subparagraph\undefined\else
\let\oldsubparagraph\subparagraph
\renewcommand{\subparagraph}[1]{\oldsubparagraph{#1}\mbox{}}
\fi

%%% Use protect on footnotes to avoid problems with footnotes in titles
\let\rmarkdownfootnote\footnote%
\def\footnote{\protect\rmarkdownfootnote}

%%% Change title format to be more compact
\usepackage{titling}

% Create subtitle command for use in maketitle
\providecommand{\subtitle}[1]{
  \posttitle{
    \begin{center}\large#1\end{center}
    }
}

\setlength{\droptitle}{-2em}

  \title{Policy Memo}
    \pretitle{\vspace{\droptitle}\centering\huge}
  \posttitle{\par}
  \subtitle{Accessibility measures for regional transportation planning}
  \author{}
    \preauthor{}\postauthor{}
      \predate{\centering\large\emph}
  \postdate{\par}
    \date{May 2 - June 11}


\begin{document}
\maketitle

To: Scott Haggerty, chair of governing Commission of Bay Area
Metropolitan Transportation Commission

From: Shen Qu, Policy Advisor

Date: 5/2/2019

RE: Define and measure accessibility in Plan Bay are 2050

\hypertarget{summary}{%
\section{Summary}\label{summary}}

Clarifying accessibility definition and measures is necessary in Plan
Bay Area 2050. Accessibility, the ease of reaching destinations, is a
key land use and transportation performance measure(Boisjoly and
El-Geneidy 2017b)

accessibility need to be have clearly defined in plans

including clear accessibility indicators in planning documents is key to
promoting the use of metrics in policy and practice, as it was stated as
a main reason motivating the generation of accessibility metrics.

Furthermore, multi-criteria analysis approaches including accessibility
indicators need to guide the decision-making process. (Boisjoly and
El-Geneidy 2017a)

ultimatey, informing decision-making and influencing land use and
transportation planning

Accessibility-oriented RTPs:

\begin{itemize}
\item
  explicitly define accessibility,
\item
  set goals and objectives in terms of accessibility (not just
  congestion reduction), and
\item
  use accessibility measures to define regional transportation
  priorities, including the evaluation of individual projects in the MPO
  service area.
\end{itemize}

\hypertarget{background}{%
\section{Background}\label{background}}

\begin{itemize}
\tightlist
\item
  Plan Bay Area 2050: they are going to update their regional
  transportation plan, along with their processes for selecting projects
  for funding, at all scales.
\end{itemize}

Plan Bay Area 2050 is a state-mandated, integrated long-range
transportation and land use plan developed and adopt by MTC and ABAG.
Work on Plan Bay Area 2050 is expected to begin in August 2019 and
focused update that builds upon the growth pattern and strategies
developed in the original Plan Bay Area 2040 (adopted in July 2017) but
with updated planning assumptions that incorporate key economic,
demographic and financial trends from the last four years.

The Bay Area can meet its transportation needs through 2050 while
preserving the character of its diverse communities and adapting to
future population growth. The Bay Area can make progress toward the
region's long-range transportation and land use goals. Plan Bay Area
2050 does set a roadmap for future transportation investments and will
identify what it would take to accommodate expected growth.

The most common accessibility-based objective is to increase access to
jobs, both as a way to foster economic development and to reduce social
inequities. For example, Transport for London identified two access to
jobs indicators, one to support economic development (through improved
employment accessibility) and one to improve social inclusion (through
increased access to employment for deprived areas). With respect to
social inclusion or social equity, a broader range of destinations is
generally included (libraries, health care facilities, greeneries,
supermarkets, etc.)

Overall most plans including accessibility objectives focus on economic
development and social inclusion, mostly through access to jobs. Access
to the transportation system (see Houston-Galveston Area Table 1) or to
mobility (see Ile-de-France Table 1) are also often stated as goals or
objectives. However, these goals do not directly address access to
destinations.

In many plans, however, accessibility or access is used in a way that
does not reflect the ease of reaching various destinations and does not
translate into accessibility indicators.

the term accessibility (or access) is often not defined, and is used as
a vague term that does not translate into clear accessibility
objectives.

Accessibility is rather used as a buzzword, together with mobility, and
does not refer to a distinct concept.

Even when specific accessibility objectives are stated in the plans,
they are often translated into indicators that do not reflect
accessibility.

Bay Area (San Francisco)

Objective of Equitable access: Decrease by 10 percentage points (to 56
percent from 66 percent) the share of low-income and lower-middle income
residents' household income consumed by transportation and housing

Objective of Access to Jobs: Average travel time in minutes for commute
trips.

Although travel time is a component of accessibility, it does not fully
reflect access to destinations. It is an indicator of mobility and does
not capture the potential of interaction for opportunities, as defined
by Hansen (1959). Having shorter travel times does not necessarily
equate to having access to a larger number of destinations. Furthermore,
as discussed by Litman (2013), strategies aiming at increasing traffic
speed may in some cases lead to an overall reduction in accessibility.
In sum, increased mobility does not always result in increased
accessibility (Halden, 2011, Levine et al., 2012).

While the concept of accessibility in formal measurement has been around
for more than fifty years, and has been a common element in the goals
and objectives of transport plans, its actual adoption in transport
planning has been limited until recently. Hansen (1959)

The lack of clarity on accessibility leaves vehicle speed as the
fundamental criterion for success in previous plans.
({\textbf{???}}\{proffitt2019accessibility)

`traditional transport planning, which tends to focus on improvements to
the transport system that facilitate mobility, without considering the
access needs that drive travel behaviour' (Chapman and Weir, 2008: 7).

travel is a derived demand \emph{(Goldman and Gorham, 2006; Grengs et
al., 2010; Halden, 2002; Handy and Niemeier, 1997; Levine et al.,
2012)}. people travel to places where they can meet their daily needs,
not simply to move about. the purpose of most travel is about the
destination, not the journey.

in those plans accessibility is often not clearly defined and thus often
used as a buzzword.

given the broad and flexible guidelines, accessibility is often
``misused'' and ``abused in practice'' (Halden, 2011)

\hypertarget{definition}{%
\section{Definition:}\label{definition}}

\begin{itemize}
\tightlist
\item
  Explains and defines accessibility, including why it should be used in
  transportation decision making
\end{itemize}

accessibility can be understood as the ease of reaching services and
activities (Litman, 2013).

The most recent federal transportation bills -- the Fixing America's
Surface Transportation Act (FAST Act) lay out guidelines for how MPOs
must develop regional transportation plans. The guidelines require every
plan to address eight planning factors, including `accessibility and
mobility of people and freight'. None of the guidelines define
accessibility explicitly, and all use it exclusively in the phrase
`accessibility and mobility'. The result is that the terms accessibility
and mobility often are conflated in practice, with both terms describing
attributes of mobility. Explicitly defining accessibility as the end
goal of the transportation network would clarify the concept for all
MPOs and local governments.

the US federal government defines eight planning factors that guide the
development of the Transportation Plans by the MPOs, one of which is to
``increase the accessibility and mobility of people and for freight''
(U.S. Department of Transportation, 2014). As accessibility is not
clearly defined, access to destinations is often not reflected in the
plans. Accessibility goals should hence be clearly defined to encourage
the establishment of accessibility-based performance indicators.

Accessibility, defined as the ease of reaching destinations
\emph{(Preston and Rajé, 2007)}, is one of the most comprehensive
performance measures of land use and transportation systems
\emph{(El-Geneidy and Levinson, 2006)}. As such, accessibility reflects
the multiple benefits provided by land use and transportation systems
\emph{(Ben-Akiva and Lerman, 1979)}. For example, greater accessibility
is associated with higher land values \emph{(Koenig, 1980, El-Geneidy
and Levinson, 2006, Du and Mulley, 2012)} and employment rates
\emph{(Ornati et al., 1969, Pignatar and Falcocch, 1969, Sanchez, 1999,
Blumenberg and Ong, 2001, Sari, 2015, Tyndall, 2015)}, as it provides
residents with greater access to a variety of opportunities. In the same
way, increased accessibility contributes to reducing the risks of social
exclusion for vulnerable individuals \emph{(Preston and Rajé, 2007,
Lucas, 2012)}. Furthermore, accessibility by transit is associated with
greater transit use \emph{(Chen et al., 2008, Owen and Levinson,
2015b)}, and can thus help in reducing car use and the resulting
greenhouse gas emissions \emph{(Levinson, 1998, Handy, 2002)}.

``Access is the fundamental force for understanding cities,''\ldots{}"
is a concept that helps understand and conceptualize the complex
relationship between transport and land use in a city and their impacts
on city organization, development, and planning to achieve more
sustainable outcomes."(Levinson and Krizek 2018, 22)

Accessibility is increasingly seen as an alternative to mobility
oriented planning paradigm (Geurs et al., 2012), as it allows capturing
the complex interactions between land use and transportation systems
(Hansen, 1959) and provides a social perspective on transportation
planning (Banister, 2008, Lucas, 2012). While mobility merely reflects
the ease of moving, accessibility addresses the ease of reaching desired
destinations, which is in fact the reason why people undertake trips
(Preston and Rajé, 2007). Accessibility is one of the most comprehensive
measures to assess the complex performance of land use and
transportation systems in a region.(Boisjoly and El-Geneidy 2017a)

\begin{itemize}
\tightlist
\item
  Importance
\end{itemize}

The accessibility paradigm frames accessibility as the end goal of a
transportation system.

Accessibility is instrumental in explaining the spatial form of
metropolitan areas. Past writings have claimed accessibility is
``perhaps the most important concept in defining and explaining regional
form and function.''8 Consistent with the standard model of urban
economics, it has been observed that living in an area with relatively
high accessibility to jobs is associated with shorter trips, as is
working in an area of relatively high housing accessibility.9

(Levinson and Krizek 2018, 111) factors that affect land use and rate of
development associated with different types of accessibility:
e.g.~access to suppliers, workforce, customers, desirable environment
(aesthetically pleasing surroundings, clean air and water),amenities
(access to recreational and other non-work destinations), friendly
government, etc.

(Handy 2018) the level of accessibility from a given place reflects the
distribution of destinations around it, the ease with which those
destinations can be reached by various modes, and the amount and
character of activity found there. It tells us something about the
choices that the built environment offers to travelers. What matters to
people is how easy it is for them to get to where they need to be, and
how easy it is to access the services they need or want. Using
accessibility as the performance measure by which we assess current
conditions and proposed policies It's a goal that almost everyone can
agree on, and it opens doors to a host of strategies that could reduce
auto dependence and improve quality of life.

\begin{itemize}
\tightlist
\item
  Accessibility versus Mobility
\end{itemize}

a clear distinction should be made between access to mobility, access to
destinations (Levine et al., 2012) and universal accessibility.

Mobility is concerned with how difficult it is to travel. accessibility
how easier it is to reach destinations.

In contrast, the concept of mobility is concerned only with how easy it
is to get from one place to another. Mobility is only part of the
picture.

accessibility focuses on reaching destinations, the end goal of a
transportation network, while mobility focuses on a single means of
achieving this goal, travel speeds. Improvements in mobility alone are
not sufficient to ensure improvements in accessibility.
({\textbf{???}}\{proffitt2019accessibility)

Accessibility does not depend on good mobility.Some places such as San
Francisco downtown have good accessibility despite having poor mobility
(severe traffic congestion). The Residents live within a short distance
of all needed and desired destinations. the travel times between
destinations are relatively short, even if travel speeds are low.

A place can also have good mobility but poor accessibility. Think of
Manitoba in Canada. Some rural area has ample roads, low levels of
congestion, and high speeds of travel, but relatively few destinations
for shopping. Thus, good mobility is neither a sufficient nor a
necessary condition for good accessibility.

Planning for mobility has taken on the meaning of making it easier to
get around. This focus on the ease of traveling along the transport
network itself (rather than focusing on the ease of reaching
destinations) has aligned well with modern planning paradigms;

road building has been the most popular solution to congestion. These
paradigms prize the planning-for-mobility perspective because it
accommodates growing levels of travel and increases the potential for
movement.

In the suburban areas of Bay Area, transit service is relatively sparse
and destinations are generally beyond walking distance, leaving
residents with no option but to drive. the practice of planning is
largely mobility-dependent, and car-dependent and has deteriorated
levels of accessibility. As traffic levels invariably increase in these
areas, accessibility ultimately declines for all modes.

Planning for accessibility, in contrast, means making it easier for
people to get where they want to go. Land use policies designed to bring
destinations within walking distance of residential areas are one
example of this paradigm. But planning for access may not even require
retrofitting neighborhoods. For example, transit services that link
specific groups of users to their desired destinations, such as reverse
commute programs and other client-based transport services, are examples
of planning for accessibility. Efforts like these reduce the need to
drive, although they don't necessarily reduce actual driving.25

({\textbf{???}}\{proffitt2019accessibility)

conventional practice in transportation planning has employed a `predict
and provide' model that focuses the majority of funding and planning
attention on expanding roadway capacity. (Levinson and Krizek 2018, 22)

performance metrics that focus on roadway congestion as experienced by
automobile drivers \emph{(Ewing, 1993; Handy, 2005; Krizek and Levinson,
2010)}, with higher vehicle speeds the `fundamental criterion for
success' \emph{(Levine et al., 2012: 158)}. Speed-based metrics include
roadway level of service (LOS), peak-period delay, traffic volume/road
capacity, travel time/speed, vehicle hours of travel, the duration of
peak-period congestion, and others \emph{(Ewing, 1996; Transportation
Research Board, 2010)}. Even high-occupancy toll lanes, the most common
demand-management strategy used in the USA, are typically added as new
capacity, only rarely replacing existing highway travel lanes
\emph{(Ewing and Proffitt, 2016)}. Planning for higher travel speeds
that facilitate longer and often more frequent trips is the mobility
paradigm.

Accessibility is focused on making it easier for individuals to reach
destinations where they can meet daily needs such as work, recreation,
socialising, shopping, and other forms of social exchange
\emph{(Martens, 2015; Miller, 2005)}.

improving mobility -- via automobile, transit, or any other travel mode
-- means facilitating faster travel speeds so individuals can reach more
destinations in a given travel time, improving proximity means
shortening distances between trip origins and destinations so
individuals can reach a satisfactory number of exchange opportunities
even if they travel more slowly. In other words, neighbourhoods, cities,
and metro areas can be made more accessible by reducing travel distances
as well as by facilitating faster travel. The advantage of planning for
accessibility versus planning solely for (auto)mobility is that the
former allows for a comparison of the tradeoffs among financial,
environmental and human health and wellbeing concerns when making
decisions about land use and transportation.

it facilitates the evaluation of tradeoffs between land use,
transportation and social needs. By combining aspects of land use and
transportation into a single measure, accessibility focuses attention on
the performance of the system as a whole rather than on just segments of
the transportation network.

the clear distinction between mobility and accessibility indicators. In
the Baltimore plan, the multi-criteria analysis includes the following
goals: safety, accessibility, mobility, environmental conservation,
security and economic prosperity. Interestingly, accessibility and
mobility are included as two distinct goals with different criteria and
methodologies,

\hypertarget{evaluates-the-accessibility-measures}{%
\section{Evaluates the accessibility
measures}\label{evaluates-the-accessibility-measures}}

\begin{itemize}
\tightlist
\item
  You evaluation should use clear criteria and draw on existing sources.
\end{itemize}

Lyons and Davidson (2016) argue for a focus upon the \emph{Triple Access
System} of spatial proximity in land use system, physical mobility in
the transport system and digital connectivity in the telecommunications
system as a framework for policy and investment decisions that can
harness flexibility and resilience.

Levinson and Krizek (2018) introduce four measure methods for
accessibility to employment\footnote{\(A_j=\sum_jE_jf(C_ij)\) where:
  \(A_i\) represents accessibility to employment from zone \(i\).
  \(E_j\) is employment at destination \(j\). \(f(C_{ij})\) is a
  function of the travel cost (time and money) between \(i\) and \(j\).
  The higher the cost, the less the weight given to the employment
  location.}, overall accessibility\footnote{an overall accessibility
  measure is a summation of the measures of all origins:
  \(A =\sum_jW_i A_i\) where \(A\) is overall accessibility for region,
  \(W_i\) represents workers living at origin \(i\)}, and gravity
model\footnote{Isaac Newton (1687) first found the relationship between
  the gravitational force, distance, and mass. Ernest Ravenstein
  (1876-1889) developed a similar idea in the context of the social
  sciences. William J. Reilly developed a ``Law of Retail Gravitation''
  (1931) John Q. Stewart developed the notion of demographic force (F)
  between places, and this demographic force equation forms the basis of
  the gravity model used in many transport planning models. Alan
  Voorhees (1956) first applied the gravity model to address problems of
  urban transport planning. \[T_{ij}=K_iK_j\frac{T_iT_j}{f(C_{ij})}\]
  \(T_i=\sum_jT_{ij},\ T_j=\sum_iT_{ij},\ K_i=\frac1{\sum_jK_jT_jf(C_{ij})},\ K_j=\frac1{\sum_iK_iT_if(C_{ij})}\)
  where: \(T_{ij}\): trips between origin \(i\) and destination \(j\)
  \(T_i\): trips originating at \(i\) (for example, workers) \(T_j\):
  trips destined for \(j\) (for example, jobs) \(f\): distance decay
  factor, as in the accessibility model \(C_{ij}\): generalized travel
  cost between \(i\) and \(j\) \(K_i, K_j\): balancing factors solved
  iteratively This gravity model suggests several things. First, as city
  size increases, mean commuting time increases. The structure of
  gravity models implies diminishing marginal returns to job
  opportunities at the edge, since each additional job is less and less
  likely to be taken and thus less likely to increase travel time.
  Second, these models are largely independent of density---except to
  the extent that density changes network speed. A uniform increase in
  density increases the opportunities within each time band
  proportionately, and thus does not change the distribution of travel
  times. Third, if preferences shift, mean travel time will change
  inward or outward. Fourth, if congestion rises, more opportunities
  will be farther away in terms of travel time, and fewer
  nearby---implying that average commuting time will rise.}. Network
size indicates an attribute of the built environment. cumulative
opportunities measures Accessibility can describe the ``interaction by a
function of the travel cost, such that distant interactions have less
weight than nearby interactions.'' gravity models also consider distance
or travel time and disclose that the interaction between places is
inversely proportional to travel cost.

\begin{itemize}
\tightlist
\item
  activity component, the land use system, goes by the presence of
  destinations, such as jobs, restaurants, daycares, health care
  facilities, and households.
\end{itemize}

Firstly, access to destinations is largely influenced by the
distribution of residential, economic, cultural and social activities
(the land use component).

location-based metrics are most commonly used in planning as they
provide a comprehensive measure of regional accessibility (Boisjoly and
El-Geneidy, 2017). These metrics indicate the ease of accessing
destinations from a specific location and accounts for the spatial
distribution of opportunities (for example, jobs or healthcare services)
and the ability to move from one place to another (Geurs and van Wee,
2004).

accessibility is largely contingent on the spatial distribution of
destinations, the land use component,

The land use component is related to the spatial distribution of
opportunities. Urban opportunities can include, but are not limited to,
jobs, health services and retail stores.

\begin{itemize}
\tightlist
\item
  the transport component. This is some measure of how easy or difficult
  it is to move along the network.
\end{itemize}

Accessibility further depends on the transportation network which
determines the travel time, costs and convenience from a place (for
example, home) to another (for example, work) (the transport component).

The transport component, the ability to move from one place to the
other, is generally mode specific and based on travel time or distance
(Hansen, 1959, Vickerman, 1974, Handy, 1994, Geurs and van Wee, 2004,
Owen and Levinson, 2015a).

and the ability to move from one place to another, the transport
component (Geurs and van Wee, 2004).

The transport component refers to the transport infrastructure specific
to each mode.

\begin{itemize}
\tightlist
\item
  utility measures (which quantify the benefit individuals get from
  destinations), A second type of measures is the utility-based
  measures, which capture the economic benefits provided by changes in
  the network. Utility-based measures account for most components of
  accessibility and can be included in traditional cost-benefit analysis
  (van Wee, 2016).
\end{itemize}

\hypertarget{access-measures-location-level-place-based.-the-availability-of-opportunities-close-by}{%
\subsection{Access measures: location level (place-based). the
availability of opportunities close
by}\label{access-measures-location-level-place-based.-the-availability-of-opportunities-close-by}}

Location-based accessibility is most commonly used by policy-makers as
it provides a comprehensive measure of the land use and transport system
at the regional level (Dodson et al., 2007). Location-based metrics
typically accounts for the number of opportunities that can be reached
from a specific location, based on the travel costs to destinations
using a specific mode (Handy and Niemeier, 1997).

Travel costs are generally measured based on travel time or distance

Two location-based measures are commonly used in accessibility research.

\begin{verbatim}
* cumulative opportunity indices (counting exchange opportunities within a defined geographic parameter), 
\end{verbatim}

A common location-based metric is a measure of cumulative-opportunities,
which counts all opportunities that can be reached within a travel costs
threshold. For example, the number of jobs that are within 45 min of
travel times by transit from a specific place is used to assess the
access to jobs by public transit.

only counts the opportunities that are within a specific travel costs
threshold.

cumulative-opportunity measures are easy to generate and interpret.
Furthermore, these measures are highly correlated with gravity-based
measures (El-Geneidy and Levinson, 2006), and hence represent an
adequate measure of regional accessibility (Boisjoly and El-Geneidy,
2016).

cumulative-opportunity metrics provide indicators that typically reflect
the ease of reaching destinations and is thus encouraged. More
specifically, the use of cumulative-opportunities measure of
accessibility to jobs by public transport and car is suggested. These
measures provide adequate indicators of the regional patterns of
accessibility, and are easy to generate, to interpret, and to
communicate (Boisjoly and El-Geneidy, 2016, Geurs and van Wee, 2004).

Accessibility metrics are typically location-based and focus on the
transport and land use components of accessibility. In all cases,
measures are based on cumulative opportunities, using a travel time or
distance threshold, mainly for public transport and driving (Fig. 1,
right). Cumulative-opportunity measures are easy to communicate and
interpret, and thus better suited for planning documents (Geurs and van
Wee, 2004).

access to destinations, and access to or from public transport station.
Access to public transport is the most common measure used in the plans.
This measure is generally presented as the percentage of people or jobs
that are within 0.5 mile of a public transport station. it does not
directly addresses the ease of reaching urban opportunities.

The second type of metrics (access to urban opportunities) directly
measures the ease of reaching various destinations, generally jobs,
using a specific mode. This measure is however more complex to generate,
as the locations of the destinations is needed.

In terms of modes and thresholds (Fig. 1), accessibility to jobs is
generally generated for transit or automobile, using travel time
thresholds varying from 30 minutes to 60 minutes.

\hypertarget{access-measures-the-ease-and-worth-of-travel-to-destinations-far-away}{%
\subsection{Access measures: the ease and worth of travel to
destinations far
away}\label{access-measures-the-ease-and-worth-of-travel-to-destinations-far-away}}

\begin{verbatim}
* gravity-type models (pitting the importance of given exchange opportunities against travel time impedances), 
\end{verbatim}

Another common metric is the gravity-based measure, which discounts
opportunities based on a distance-decay function. Accordingly,
opportunities that are located farther (by distance or time) receive
less weight than closer opportunities. While this measure is more
reflective of travel behavior, cumulative-opportunities are simpler to
generate, interpret and communicate.

discounts all opportunities based on their travel costs.

Gravity-based measures better reflect travel behavior as it accounts for
the travelers' perceptions of time (Ben-Akiva and Lerman, 1979). This
measure is, however, more complex to generate, as a distance-decay
function must be calculated, and more difficult to interpret and
communicate, as it is not directly expressed in terms of the number of
opportunities (Geurs and van Wee, 2004; A. Owen and Levinson, 2014).

Overall, the cumulative-opportunities accessibility metrics are
generated for access to transport, and to a lesser extent, for access to
destinations, mainly jobs. Ideally, plans would integrate both types of
metrics. Access to transport provides a good indication of transport
coverage, whereas access to destinations captures the performance of the
land use and transportation systems, which better reflect the social and
economic benefits (Banister, 2008, Koenig, 1980, Wachs and Kumagai,
1973).

\hypertarget{options}{%
\section{Options}\label{options}}

\begin{itemize}
\item
  Defines different options for measuring accessibility for use in the
  regional transportation plan, as well as at the project level
  (e.g.~new roadway capacity, transit infrastructure, bike/pedestrian
  infrastructure, demand and system management projects, etc.).
\item
  the individual level (person-based). Person-based measures of
  accessibility are generated at the individual level, and are concerned
  with the level of accessibility experienced by a specific person
  (Geurs and van Wee, 2004, Miller, 2005, Owen and Levinson, 2015b).
\end{itemize}

In addition to the exogenous factors, individual characteristics such as
income, level of education, gender and vehicle ownership affect one's
abilities and needs to access destinations (the individual component).

whereas personal characteristics such as income and car ownership
reflect the individual component.

With respect to the individual components, many areas segment the
accessibility analysis by socio-economic groups. However, only few of
them (Atlanta Regional Commission, Boston Region Metropolitan Planning
Organization) do address destination segmentation. This is an important
improvement as the accessibility to all jobs may not represent the
opportunities that are available to different groups of populations

\begin{itemize}
\item
  space-time prisms (three-dimensional constructs measuring individuals'
  range of possible geographic movement within specified time
  constraints)

  Time restrictions also play an important role in determining
  accessibility. These include land use, transport and individual
  constraints such as the availability of opportunities (i.e., opening
  hours), personal schedules, and the schedule of public transport
  services.
\end{itemize}

the attractiveness of destinations based on the number of opportunities
available there and an impedance factor based on travel cost/time.

The availability of opportunities for example (opening hours of shops
and services, job starting time) represents temporal elements,

More detailed analysis can include other types of destinations, or
segmentation by job types, to address specific social issues, all
depending on the context of analysis. Temporal fluctuations in
accessibility can also be addressed to improve the quality of the
accessibility analysis. Furthermore, while most plans focus on car
accessibility, and to a lesser extent on accessibility by transit, all
modes should be included in the accessibility objectives and indicators.
Increasing accessibility by transit, cycling and walking can contribute
to achieving broader environmental, economic and social goals.

accessibility indicators should systematically be included in
multi-criteria analyses.

it offers an alternative to mobility-based decisions and potentially
provide greater transparency in the decision-making process (Halden,
2011). Furthermore, national and regional authorities can require local
authorities to address accessibility in their project analysis. One
especially effective way of doing so is by including accessibility
criteria in the selection process of projects,

\begin{itemize}
\item
  be measured by destination
\item
  Absolute vs.~Relative Accessibility\footnote{Absolute accessibility is
    the total measure of accessibility within a particular area. A
    transport improvement increases overall accessibility---analogous to
    increasing the size of the pie. Relative accessibility is the share
    of total accessibility associated with a particular place. A new
    transport facility increases the relative accessibility of those
    points that can directly use the facility---analogous to increasing
    the percentage of the pie that a particular slice comprises.}
\end{itemize}

Absolute accessibility is the total measure of accessibility within a
particular area. A transport improvement increases overall
accessibility---analogous to increasing the size of the pie. Relative
accessibility is the share of total accessibility associated with a
particular place. A new transport facility increases the relative
accessibility of those points that can directly use the
facility---analogous to increasing the percentage of the pie that a
particular slice comprises. For a new infrastructure improvement (a
faster bus, a new link, etc.), while society overall receives greater
accessibility, the markets served by the improvement gain in both
absolute and relative accessibility.

\begin{itemize}
\item
  be measured by mode
\item
  Regional and Local/neighborhood Accessibility\footnote{Regional
    accessibility is determined by the regional structure of a
    metropolitan area and incorporated variables such as location, type
    of activities, and size of activities that affect shopping behavior.
    Local accessibility, is primarily determined by nearby activity
    (where ``nearby'' is used to refer to the neighborhood unit,
    approximately one-half to one mile (800 to 1,600 m) in residential
    areas). Areas with higher local accessibility would be oriented to
    convenience goods, such as supermarkets and drug stores, and located
    in small centers.} be measured by scales
\end{itemize}

Regional accessibility is determined by the regional structure of a
metropolitan area and incorporated variables such as location, type of
activities, and size of activities that affect shopping behavior. Local
accessibility is primarily determined by nearby activity (where
``nearby'' is used to refer to the neighborhood unit, approximately
one-half to one mile (800 to 1,600 m) in residential areas). Areas with
higher local accessibility would be oriented to convenience goods, such
as supermarkets and drug stores, and located in small centers.

Many policy initiatives speak to increasing accessibility on both
regional and local scales; and, while the two scales are intricately
related, each calls for different policies. regional transport--land use
policies may speak to issues of urban growth boundaries, increasing
densification, and diversifying the geographical distribution of
employment centers.

Access to jobs provides an adequate indicator of regional accessibility,
as many people commute across the region for work. Access to jobs can
also be a reflection of the level of services available around a certain
location, as the delivery of services often equates a certain number of
employees.

Neighborhood accessibility policy initiatives speak more to issues of
mixing uses on a parcel or neighborhood scale, site design, and more
directly, facilitating circulation patterns that enhance walking,
bicycling, and transit use.

Other types of destinations include libraries, schools, grocery stores,
hospitals, public parks, educational services as exemplified by the
Atlanta Regional Commission. Many of these destinations reflect local
accessibility and are thus often associated with cycling and walking.

The most common metrics for cycling and walking are measures of local
accessibility (to grocery stores, schools, parks or public transport
station for example),

With respect to local accessibility walking and cycling distance
thresholds are used instead of travel time thresholds (0.5 miles for
walking). These appear to be appropriate measures of accessibility, as
time is generally proportional to the distance travelled by bicycle or
foot.

(Proffitt et al. 2019) increasingly tight budgets and a growing
awareness that it simply is not possible for regions to pave their way
out of congestion do seem to be pushing many MPOs to look for
alternatives to expanding roadway capacity. Improving accessibility by
coordinating land-use and transportation rather than an exclusive focus
on automobility is one such alternative.

Analysing future development scenarios in terms of their levels of
accessibility could leverage MPOs' influence on municipal-level land-use
decisions even without explicit authority over them. Highlighting the
connections between land use and transportation infrastructure
facilitates cost-benefit analysis of providing accessibility rather than
relieving congestion. Such comparisons could help MPOs provide better
information about the performance and the costs of different
transportation-infrastructure and land-use scenarios. Better information
about the tradeoffs inherent in different development scnearios can help
regions choose projects more objectively.

OECD (2017) provide some latest research findings, methodologies and
data sources on urban accessibility.

Another good practice to address accessibility in transport plan is the
use of visualization tools such as maps. Accessibility maps provide a
clear way to communicate gaps and benefits of a transportation and land
use network, and thus helps decision-makers, planners and the general
population to better grasp the impacts of transportation investments.

All Transit: \url{https://alltransit.cnt.org/}

WalkScore: \url{https://www.walkscore.com/}

Bicycle Network Analysis: \url{https://bna.peopleforbikes.org/\#/}
(city-level only)

Mobility Score: \url{https://transitscreen.com/mobilityscore/}

EPA Access to Jobs \& Workers via Transit Tool: Overview:
\url{https://www.epa.gov/smartgrowth/smart-location-mapping\#Trans45}
Link to tool:
\url{https://epa.maps.arcgis.com/home/webmap/viewer.html?webmap=3bffc086a9b34928a632ab6c8530ebcf}

EPA Walkability Index: Overview:
\url{https://www.epa.gov/smartgrowth/smart-location-mapping\#walkability}\\
Link to tool:
\url{http://www.arcgis.com/home/webmap/viewer.html?url=https\%3A\%2F\%2Fgeodata.epa.gov\%2Farcgis\%2Frest\%2Fservices\%2FOA\%2FWalkabilityIndex\%2FMapServer\&source=sd}

One of the most systematic and transparent way to inform decision-making
is by including accessibility indicators into multi-criteria analyses,
as done by the Baltimore Regional Transportation Board, the Puget Sound
Regional Council, Transport for London and the Greater Manchester
Combined Authority. For example, in Baltimore, a multi-criteria analysis
was conducted to compare the projects submitted by local jurisdictions
and to select the ones to be included in the Regional Transportation
Plan (Maximize 2040). Similarly, the Puget Sound Regional Council
included accessibility in their multi-criteria analysis used to conduct
a prioritization of the projects. With respect to scenario assessments,
Transport for London used a multi-criteria analysis including
accessibility indicators to assess the effectiveness of various
modelling scenarios.

The accessibility indicators included in the multi-criteria analyses
range from broad questions to specific quantified metrics, which
influence the flexibility of the analysis. For example, Transport for
London defines clear specific accessibility metrics, for example the
change in the number of jobs accessible by public transport within 45
minutes travel time (see Table 1). These access to jobs metrics are
relatively easy to generate and to interpret. Accordingly, they foster
the inclusion of accessibility indicators that adequately reflect the
ease of reaching destinations. Furthermore, given their specific nature,
they are easy to communicate as exemplified in the plan: ``Implementing
the schemes will increase the employment catchment area of central
London (the number of people within 45 minutes of central London
employment) by almost 25 percent.'' (p.74).

In contrast, accessibility criteria in the Greater Manchester plan are
defined with broad questions such as ``Will the LTP help improve
accessibility through integrated spatial planning?'' and ``Will the LTP
improve access to jobs, particularly for people who suffer income or
employment deprivation?'' (see Table 1). These questions provide greater
flexibility in the assessment of the plan, which can be beneficial as
quantified metrics do not always reflect the benefits provided by
improvements in accessibility (Curl et al., 2011). However, as
emphasized by Halden (2011), it can also lead to the use and misuse of
the concept of accessibility.

An intermediate way of defining accessibility indicators is by
attributing scores (from 1 to 3 for example) based on specific
guidelines. This approach has the advantage of defining clear weights
associated with accessibility criteria, thus providing greater
transparency.

quantified metrics provide more specific guidelines that directly
reflect the ease of reaching destinations. However, they provide lower
flexibility and might not adequately reflect the outcomes of the
different investments.

including accessibility indicators in multi-criteria analysis provides a
systematic alternative to mobility-focused decision-making.

Accessibility maps and metrics are useful tools to provide an overview
of the land use and transportation network and they illustrate an
underlying accessibility analysis.

other dimensions of accessibility might currently be neglected in
metropolitan transportation plans. For example, affordability, transfer
and multimodal connectivity, as well as travel information did not come
up as main aspects of accessibility objectives.

\hypertarget{equity-analysis}{%
\section{equity analysis}\label{equity-analysis}}

Discusses how equity could be incorporated in the accessibility
measures. In doing so, how are you defining equity?

Equity analysis based on accessibility indicators generally assess the
level of accessibility of specific vulnerable groups relatively the
general population, using detailed accessibility metrics.

However, in most cases the use of the generated accessibility metrics is
limited to the environmental justice assessment, although accessibility
is also stated as a main planning factor by the federal government.

Accessibility is mainly perceived as an equity indicator, while it has
the potential to address multiple aspects of a land use and
transportation system.

It is also important to note that accessibility indicators should be
used as general performance indicators and should not be limited to
social equity analyses.

Many plans from American metropolitan areas generate accessibility
measures to address the environmental justice federal requirement.

Yet, accessibility allows tackling multiple objectives, including
environmental and economic benefits (Handy, 2002, Koenig, 1980), and
should hence also be used to assess the overall benefits of potential
investments.

Measures of generalized costs (including the costs and time of travel)
have been developed in the literature (Bocarejo and Oviedo, 2012,
El-Geneidy et al., 2016). These measures better reflect the total costs
of travel as they include both financial and time burdens. They are
however very challenging to generate due to complex fare structures and
availability of data. Yet, excluding the financial costs of travel
results in an overestimation of accessibility (El-Geneidy et al., 2016),
especially for low-income individuals. In this regard, accessibility
based on financial and time costs is closer to reality and can also
provide an insight on fare structures and trip affordability. From a
planning perspective, travel time measures of accessibility adequately
represent accessibility patterns with respect to the transportation
networks and locations of activities, but do not address the financial
constraints that vulnerable individuals may face.

\hypertarget{conclusion}{%
\section{Conclusion}\label{conclusion}}

Neccessary define and measure accessibility in updating Plan Bay area
2050.

Choose proper method and criteria

incorporate measures of accessibility into decision making and needs the
MPO board to agree.

Integrating RTP goals with accessibility-focused performance measures
could help MPOs make better decisions about the selection and
prioritisation of transportation infrastructure. For instance,
prioritising transit improvements to connect key origins and
destinations can increase ridership \emph{(Badoe and Miller, 2000;
Chakraborty and Mishra, 2013)} Accessibility tools that take into
account benefits from both transportation and land-use decisions provide
a more complete picture than mobility measures alone.

National and regional governments and organizations can play a key role
in setting clear accessibility requirements for transportation planning
processes and planning documents. clear guidelines must be provided and
a clear distinction between mobility and accessibility must be
made.(Boisjoly and El-Geneidy 2017b)

Accessibility, the ease of reaching destinations, is increasingly seen
as a complimentary and in some cases alternative to the mobility
oriented planning paradigm, as it allows capturing the complex
interactions between land use and transportation systems while providing
a social perspective on transportation planning. how accessibility is
incorporated into metropolitan transportation plans and translated into
performance indicators around the world, to ultimately derive policy
recommendations. few plans have accessibility-based indicators that can
guide their decision-making processes. plans need to have clearly
defined accessibility goals with a distinction between accessibility and
mobility. Furthermore, multi-criteria analysis approaches including
accessibility indicators need to guide the decision-making process.

More efforts are needed to effectively implement accessibility-based
approaches.

Table 3. Best Practices for a Greater Inclusion of Accessibility
Planning and Metrics.

\textbar{}Recommendation \textbar{}Description\textbar{} Key examples
Accessibility goals and objectives \textbar{}Clearly defined goals and
objectives are included in the plan. The plan is structured around the
goals and objectives.\textbar{}London \textbar{} --- \textbar{} ---
\textbar{} --- \textbar{} Distinction between accessibility and mobility
\textbar{}Distinct accessibility and mobility objectives and indicators
are defined. \textbar{}Baltimore \textbar{}Multi-criteria analysis
including accessibility indicators \textbar{}Accessibility indicators
are systematically included in the performance analyses. Accessibility
metrics are used to assess the general performance of the land use and
transportation system, in addition to social equity.\textbar{}London,
Baltimore, Puget Sound (Seattle), Manchester, Melbourne \textbar{}Access
to destinations metrics \textbar{}The accessibility indicators are based
on access to destinations (e.g.: jobs), rather than to transport
amenities (e.g.: public transport stop) \textbar{}Boston
\textbar{}Multiple modes \textbar{}Accessibility is measured for various
modes of transport \textbar{}North Central Texas, Atlanta
\textbar{}Visualization tools \textbar{}Accessibility maps are included
in the plan. \textbar{}London, Sydney

Overall, clear multi-criteria analysis, using clearly defined
indicators, provide greater transparency and typically foster the
inclusion of accessibility aspect in the decision-making process.

\hypertarget{notes}{%
\section{Notes}\label{notes}}

\hypertarget{references}{%
\section*{References}\label{references}}
\addcontentsline{toc}{section}{References}

\hypertarget{refs}{}
\leavevmode\hypertarget{ref-boisjoly2017get}{}%
Boisjoly, Geneviève, and Ahmed M El-Geneidy. 2017a. ``How to Get There?
A Critical Assessment of Accessibility Objectives and Indicators in
Metropolitan Transportation Plans.'' \emph{Transport Policy} 55.
Elsevier: 38--50.

\leavevmode\hypertarget{ref-boisjoly2017insider}{}%
---------. 2017b. ``The Insider: A Planners' Perspective on
Accessibility.'' \emph{Journal of Transport Geography} 64. Elsevier:
33--43.

\leavevmode\hypertarget{ref-handy2018enough}{}%
Handy, Susan. 2018. ``Enough with the `Ds' Already---Let's Get Back to
`a'.'' Transfers Magazine.
\url{https://transfersmagazine.org/enough-with-the-ds-already-lets-get-back-to-a/}.

\leavevmode\hypertarget{ref-hansen1959accessibility}{}%
Hansen, Walter G. 1959. ``How Accessibility Shapes Land Use.''
\emph{Journal of the American Institute of Planners} 25 (2). Taylor \&
Francis: 73--76.

\leavevmode\hypertarget{ref-levinson2018metropolitan}{}%
Levinson, David M, and Kevin J Krizek. 2018. \emph{Metropolitan Land Use
and Transport: Planning for Place and Plexus}. Routledge.
\url{https://doi.org/10.4324/9781315684482}.

\leavevmode\hypertarget{ref-litman2017evaluating}{}%
Litman, Todd. 2017. \emph{Evaluating Accessibility for Transport
Planning}. Victoria Transport Policy Institute.
\url{http://www.vtpi.org/access.pdf}.

\leavevmode\hypertarget{ref-lyons2016guidance}{}%
Lyons, Glenn, and Cody Davidson. 2016. ``Guidance for Transport Planning
and Policymaking in the Face of an Uncertain Future.''
\emph{Transportation Research Part A: Policy and Practice} 88. Elsevier:
104--16.

\leavevmode\hypertarget{ref-OECD2017linking}{}%
OECD, International Transport Forum. 2017. ``Linking People and
Places.'' ITF. \url{https://www.itf-oecd.org/linking-people-and-places}.

\leavevmode\hypertarget{ref-proffitt2019accessibility}{}%
Proffitt, David G, Keith Bartholomew, Reid Ewing, and Harvey J Miller.
2019. ``Accessibility Planning in American Metropolitan Areas: Are We
There yet?'' \emph{Urban Studies} 56 (1). SAGE Publications Sage UK:
London, England: 167--92.


\end{document}
