\documentclass[12pt,]{article}
\usepackage{lmodern}
\usepackage{amssymb,amsmath}
\usepackage{ifxetex,ifluatex}
\usepackage{fixltx2e} % provides \textsubscript
\ifnum 0\ifxetex 1\fi\ifluatex 1\fi=0 % if pdftex
  \usepackage[T1]{fontenc}
  \usepackage[utf8]{inputenc}
\else % if luatex or xelatex
  \ifxetex
    \usepackage{mathspec}
  \else
    \usepackage{fontspec}
  \fi
  \defaultfontfeatures{Ligatures=TeX,Scale=MatchLowercase}
    \setmainfont[]{Times New Roman}
\fi
% use upquote if available, for straight quotes in verbatim environments
\IfFileExists{upquote.sty}{\usepackage{upquote}}{}
% use microtype if available
\IfFileExists{microtype.sty}{%
\usepackage{microtype}
\UseMicrotypeSet[protrusion]{basicmath} % disable protrusion for tt fonts
}{}
\usepackage[margin=1in]{geometry}
\usepackage{hyperref}
\hypersetup{unicode=true,
            pdftitle={Reading reflections},
            pdfauthor={Shen Qu},
            pdfborder={0 0 0},
            breaklinks=true}
\urlstyle{same}  % don't use monospace font for urls
\usepackage{graphicx,grffile}
\makeatletter
\def\maxwidth{\ifdim\Gin@nat@width>\linewidth\linewidth\else\Gin@nat@width\fi}
\def\maxheight{\ifdim\Gin@nat@height>\textheight\textheight\else\Gin@nat@height\fi}
\makeatother
% Scale images if necessary, so that they will not overflow the page
% margins by default, and it is still possible to overwrite the defaults
% using explicit options in \includegraphics[width, height, ...]{}
\setkeys{Gin}{width=\maxwidth,height=\maxheight,keepaspectratio}
\IfFileExists{parskip.sty}{%
\usepackage{parskip}
}{% else
\setlength{\parindent}{0pt}
\setlength{\parskip}{6pt plus 2pt minus 1pt}
}
\setlength{\emergencystretch}{3em}  % prevent overfull lines
\providecommand{\tightlist}{%
  \setlength{\itemsep}{0pt}\setlength{\parskip}{0pt}}
\setcounter{secnumdepth}{0}
% Redefines (sub)paragraphs to behave more like sections
\ifx\paragraph\undefined\else
\let\oldparagraph\paragraph
\renewcommand{\paragraph}[1]{\oldparagraph{#1}\mbox{}}
\fi
\ifx\subparagraph\undefined\else
\let\oldsubparagraph\subparagraph
\renewcommand{\subparagraph}[1]{\oldsubparagraph{#1}\mbox{}}
\fi

%%% Use protect on footnotes to avoid problems with footnotes in titles
\let\rmarkdownfootnote\footnote%
\def\footnote{\protect\rmarkdownfootnote}

%%% Change title format to be more compact
\usepackage{titling}

% Create subtitle command for use in maketitle
\providecommand{\subtitle}[1]{
  \posttitle{
    \begin{center}\large#1\end{center}
    }
}

\setlength{\droptitle}{-2em}

  \title{Reading reflections}
    \pretitle{\vspace{\droptitle}\centering\huge}
  \posttitle{\par}
  \subtitle{USP 570}
  \author{Shen Qu}
    \preauthor{\centering\large\emph}
  \postauthor{\par}
      \predate{\centering\large\emph}
  \postdate{\par}
    \date{Week 4}


\begin{document}
\maketitle

\begin{itemize}
\tightlist
\item
  balance and imbalance, the theories of location selection
\end{itemize}

Levinson and Krizek (2018) introduces the how the networks within
economies dictate the current lay of land use. the networks include
public and non-public entities. The firms as the non-public entities,
including developers and locators (non-retailers {[}Chapter.8{]} and
retialers {[}Chapter.9{]}), always try to reduce transaction costs
taking place within firms while market saving transaction costs by the
gains from specialization and economies of scale. Developers provide
space for poducing/cunsuming, exchaning, and connecting.

Suppliers, competitors, complementors, and customers are four basic
factors of firms' location selection in metropolitan areas Levinson and
Krizek (2018 Chapter.8). The four factors also form an economic network,
which dominates a business's location related to the proximity of both
labor and material, which is called supply chain. Supply chain is ``a
network of facilities and distribution options that procures materials,
transforms the materials into intermediate and finished products, and
distributes the finished products to customers.''(Ganeshan and Harrison
2005)

Most of location related theories based on the assumption of minimizing
transportation cost and maximizing externalities. From Alfred Weber's
Industiral Location Theory to Alonso (1960)`s bid-rent curve in a
monocentric city, Christaller (1966)'s central place theory, Zipf
(1949)'s Law about the hierarchy of places, and Lösch (1967)'s theory
about firms' location decisions and the spatial competition between
them, explain how locators compete for the sites with higher
accessibility.

As a model of New Economic Geography (NEG), agglomeration economies
(Fujita, Krugman, and Venables 2001) provide another type of
explaination. Some positive inter-firm externalities, information
spillovers, local non-traded inputs, and a locally skilled labor pool,
explain clusters of employment such as edge cities, subruban activity
centers, secondary business districts, and polynucleated city.

Accessibility works for both types of theory. Spatial proximity to the
workforce, supplier, or cunsumer can reduce transaction costs. Levinson
and Krizek (2018) list ten factors\footnote{The ten factors that affect
  the location and rate of development: 1. market velocity (the general
  level of activity in a specified market); 2. price of land; 3.
  availability of hard infrastructure (capabilities related to roads,
  water, sewers); 4. access choices (intersections, frequency of
  existing transit services, parking); 5. human infrastructure
  (education of workforce, nearby school quality, housing, daycare); 6.
  physical character (quality surrounding district, vitality, views and
  vistas); 7. environmental quality (healthy air and water); 8.
  predictability (no dramatic changes in zoning or character,
  appropriate capital improvement plan); 9. amenities (parks,
  restaurants); 10. available financing.} which affect the location
decision and rate of development. In authors' words, all relate to
accessibility in some ways.

Economies of agglomeration are also the driving force to break
job-housing balance. Residences and firms are competing for the same
land. Althogh job and workers are largely in balance at the metropolitan
level, they are always imbalnaced inside metropolitan. Even jobs-worker
are balanced in number by some policy or design, the workfore may not
compatible with local jobs, which is called spatial mismatch.

\begin{itemize}
\tightlist
\item
  Gentrification and displacement
\end{itemize}

Gentrification and displacement are consequnces of this imbalance. Zuk
et al. (2015)'s literature reveiw analyses the definitions of
gentrification and replacement. The authors also examine the approaches
of measuring gentrification and displacement. At the end, the authors
emphasis the role of public investments in transportation infrastructure
on neighborhood change.

In hamonic models, increasing housing price is a positive signal
responsing the infrastructure improvment. In another perspective, some
schorlars find higher housing price is a causal factor associated with
the gentrification. The distinction in three type of displacment
pressure can explain the phenomenon in some way. Disinvestment,
Reinvestment and enhanced market competition result the involuntary
displacement, which can occur even in the absence of gentrification.
Previous studies cannot establish the relationship between
gentrification and displacement. The authors suggest that future
research should examine more aspects except home price increases, and
should explore the impact of public investment on commercial change,
employment paterns, affordability of goods and services, and change in
clientele.

Since violence has plummeted dramatically since the 1990s, Sharkey
(2018) argues that gentrification has brought unrecognised benefits to
the poor in many cities. Sharkey also says that ``the relative lack of
income mobility at the neighbourhood level across the USA challenges the
narrative of rampant gentrification.'' Sampson (2019)'s study shows
that, rather than public investment, neighbourhood inequality is an
important driver and mediator of urban transformation. The discussion
should focus on neighbourhood structure which is ``a persistent feature
of urban systems that exert causal effects on a wide variety of everyday
life.'' The research of transportation and land use should not only
observe where people live, but also observe where they travel throughout
a city and to whom they are exposed by visits from others.

\hypertarget{notes}{%
\section{Notes}\label{notes}}

\hypertarget{references}{%
\section*{References}\label{references}}
\addcontentsline{toc}{section}{References}

\hypertarget{refs}{}
\leavevmode\hypertarget{ref-alonso1960theory}{}%
Alonso, William. 1960. ``A Theory of the Urban Land Market.''
\emph{Papers in Regional Science} 6 (1). Wiley Online Library: 149--57.

\leavevmode\hypertarget{ref-christaller1966central}{}%
Christaller, Walter. 1966. \emph{Central Places in Southern Germany}.
Prentice Hall.

\leavevmode\hypertarget{ref-fujita2001spatial}{}%
Fujita, Masahisa, Paul R Krugman, and Anthony J Venables. 2001.
\emph{The Spatial Economy: Cities, Regions, and International Trade}.
MIT press.

\leavevmode\hypertarget{ref-ganeshan2005introduction}{}%
Ganeshan, R, and TP Harrison. 2005. ``An Introduction to Supply Chain
Management. Http.''

\leavevmode\hypertarget{ref-levinson2018metropolitan}{}%
Levinson, David M, and Kevin J Krizek. 2018. \emph{Metropolitan Land Use
and Transport: Planning for Place and Plexus}. Routledge.
\url{https://doi.org/10.4324/9781315684482}.

\leavevmode\hypertarget{ref-litman2017evaluating}{}%
Litman, Todd. 2017. \emph{Evaluating Accessibility for Transport
Planning}. Victoria Transport Policy Institute.
\url{http://www.vtpi.org/access.pdf}.

\leavevmode\hypertarget{ref-losch1967economics}{}%
Lösch, August. 1967. \emph{The Economics of Location}. New York, John
Wiley.

\leavevmode\hypertarget{ref-sampson2019neighbourhood}{}%
Sampson, Robert J. 2019. ``Neighbourhood Effects and Beyond: Explaining
the Paradoxes of Inequality in the Changing American Metropolis.''
\emph{Urban Studies} 56 (1). SAGE Publications Sage UK: London, England:
3--32.

\leavevmode\hypertarget{ref-sharkey2018uneasy}{}%
Sharkey, Patrick. 2018. \emph{Uneasy Peace: The Great Crime Decline, the
Renewal of City Life, and the Next War on Violence}. WW Norton \&
Company.

\leavevmode\hypertarget{ref-zipf1949human}{}%
Zipf, George Kingsley. 1949. ``Human Behavior and the Principle of Least
Effort.'' addison-wesley press.

\leavevmode\hypertarget{ref-zuk2015gentrification}{}%
Zuk, Miriam, Ariel H Bierbaum, Karen Chapple, Karolina Gorska, Anastasia
Loukaitou-Sideris, Paul Ong, and Trevor Thomas. 2015. ``Gentrification,
Displacement and the Role of Public Investment: A Literature Review.''
In \emph{Federal Reserve Bank of San Francisco}. Vol. 32.


\end{document}
