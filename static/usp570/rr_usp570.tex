\documentclass[12pt,]{article}
\usepackage{lmodern}
\usepackage{amssymb,amsmath}
\usepackage{ifxetex,ifluatex}
\usepackage{fixltx2e} % provides \textsubscript
\ifnum 0\ifxetex 1\fi\ifluatex 1\fi=0 % if pdftex
  \usepackage[T1]{fontenc}
  \usepackage[utf8]{inputenc}
\else % if luatex or xelatex
  \ifxetex
    \usepackage{mathspec}
  \else
    \usepackage{fontspec}
  \fi
  \defaultfontfeatures{Ligatures=TeX,Scale=MatchLowercase}
    \setmainfont[]{Times New Roman}
\fi
% use upquote if available, for straight quotes in verbatim environments
\IfFileExists{upquote.sty}{\usepackage{upquote}}{}
% use microtype if available
\IfFileExists{microtype.sty}{%
\usepackage{microtype}
\UseMicrotypeSet[protrusion]{basicmath} % disable protrusion for tt fonts
}{}
\usepackage[margin=1in]{geometry}
\usepackage{hyperref}
\hypersetup{unicode=true,
            pdftitle={Reading reflections},
            pdfauthor={Shen Qu},
            pdfborder={0 0 0},
            breaklinks=true}
\urlstyle{same}  % don't use monospace font for urls
\usepackage{graphicx,grffile}
\makeatletter
\def\maxwidth{\ifdim\Gin@nat@width>\linewidth\linewidth\else\Gin@nat@width\fi}
\def\maxheight{\ifdim\Gin@nat@height>\textheight\textheight\else\Gin@nat@height\fi}
\makeatother
% Scale images if necessary, so that they will not overflow the page
% margins by default, and it is still possible to overwrite the defaults
% using explicit options in \includegraphics[width, height, ...]{}
\setkeys{Gin}{width=\maxwidth,height=\maxheight,keepaspectratio}
\IfFileExists{parskip.sty}{%
\usepackage{parskip}
}{% else
\setlength{\parindent}{0pt}
\setlength{\parskip}{6pt plus 2pt minus 1pt}
}
\setlength{\emergencystretch}{3em}  % prevent overfull lines
\providecommand{\tightlist}{%
  \setlength{\itemsep}{0pt}\setlength{\parskip}{0pt}}
\setcounter{secnumdepth}{0}
% Redefines (sub)paragraphs to behave more like sections
\ifx\paragraph\undefined\else
\let\oldparagraph\paragraph
\renewcommand{\paragraph}[1]{\oldparagraph{#1}\mbox{}}
\fi
\ifx\subparagraph\undefined\else
\let\oldsubparagraph\subparagraph
\renewcommand{\subparagraph}[1]{\oldsubparagraph{#1}\mbox{}}
\fi

%%% Use protect on footnotes to avoid problems with footnotes in titles
\let\rmarkdownfootnote\footnote%
\def\footnote{\protect\rmarkdownfootnote}

%%% Change title format to be more compact
\usepackage{titling}

% Create subtitle command for use in maketitle
\providecommand{\subtitle}[1]{
  \posttitle{
    \begin{center}\large#1\end{center}
    }
}

\setlength{\droptitle}{-2em}

  \title{Reading reflections}
    \pretitle{\vspace{\droptitle}\centering\huge}
  \posttitle{\par}
  \subtitle{USP 570}
  \author{Shen Qu}
    \preauthor{\centering\large\emph}
  \postauthor{\par}
      \predate{\centering\large\emph}
  \postdate{\par}
    \date{Week 5}


\begin{document}
\maketitle

\begin{itemize}
\tightlist
\item
  virtuous circle or vicious circle in mode split
\end{itemize}

Levinson and Krizek (2018 Chapter.5) introduce some concepts about
individual demand, modal competition, and network effects explain the
mode choice decisions. The discipline of psychology defines travel
behavior as habitual behavior which means ``learned sequences of acts
that have become automatic responses to specific cues, and are
functional in obtaining certain goals or end-states.''

travelers are generally being individually rational. They select
automobile for transit's weakness in terms of door-to-door travel time.
\textbf{Wardrop's Principle of User Equilibrium}10 states that users are
minimizing their own time rather than minimizing society's overall
travel time. \textbf{arms race} is another reason of consumer prefer
SUVs than compact car, choosing car instead of bicycle in some case. The
competition between users leads to overuse ``common pool'' resources
that have limited supply and free access, bids up the cost for everyone.
The competition between modes may result in socially sub-optimal results
and lead to a vicious circle.

To avoid \textbf{Prisoner's Dilemma}, a mechanism should let players
consider their effects on others, care about the future payoffs, and
play a Pareto efficient strategy. The investment and subsidies in
transit show a collective rationality for lower total social travel
costs by a high transit mode share. The \textbf{Mohring Effect}7 states
that When bus frequency increases on a given route, users benefit from
reduced waiting times, an increasing returns property of networks The
positive feedback loop.

Transit supply and demand have two stable states: One at high wait time
yields zero ridership, which returns high wait time. Another at low wait
time yields high rider ship, which returns low wait times. The interim
states are not stable and need large subsidies to prop them up.

\begin{itemize}
\tightlist
\item
  Describing people's travel
\end{itemize}

In Chapter.6, Levinson and Krizek (2018) describe and explain people's
time spent in activities and transportation. The destination
characteristics strongly influence trip characteristics. These
discribtion only let us realize the complex of travel behaviors. It's
still hard to understand and measure the whole prossess of trips. The
analysts use several strategies to understand time spent in travel. Work
versus non-work trips, simple versus complex tours, there are many
dimensions of travel. The commom methods include calculate the distance,
number of trips, and the mode of travel. Measures can be averaged at
different levels from a single individual to a household, to
transportation analysis zones, or even to entire metropolitan regions.
Hagerstrand's Space--Time Prism provides a powerful method for
representing the relation among activities, individual trips, and total
travel.

``Travel is a derived demand, and therefore we only do it when we want
to get somewhere else or when necessary.'' In other word, how complex
people's activities are determines how complex people's travels are.
Hamilton suggests that people's actual commutes were eight times longer
than model-predicted values for shortest commutes. It is not strange
that ``even the most robust models that predict travel distance rarely
explain more than 30 percent of the observed variation.'' The concoction
of behaviors is difficult to predict.

\hypertarget{references}{%
\section*{References}\label{references}}
\addcontentsline{toc}{section}{References}

\hypertarget{refs}{}
\leavevmode\hypertarget{ref-levinson2018metropolitan}{}%
Levinson, David M, and Kevin J Krizek. 2018. \emph{Metropolitan Land Use
and Transport: Planning for Place and Plexus}. Routledge.
\url{https://doi.org/10.4324/9781315684482}.


\end{document}
