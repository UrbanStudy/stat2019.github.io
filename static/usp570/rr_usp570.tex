\documentclass[12pt,]{article}
\usepackage{lmodern}
\usepackage{amssymb,amsmath}
\usepackage{ifxetex,ifluatex}
\usepackage{fixltx2e} % provides \textsubscript
\ifnum 0\ifxetex 1\fi\ifluatex 1\fi=0 % if pdftex
  \usepackage[T1]{fontenc}
  \usepackage[utf8]{inputenc}
\else % if luatex or xelatex
  \ifxetex
    \usepackage{mathspec}
  \else
    \usepackage{fontspec}
  \fi
  \defaultfontfeatures{Ligatures=TeX,Scale=MatchLowercase}
    \setmainfont[]{Times New Roman}
\fi
% use upquote if available, for straight quotes in verbatim environments
\IfFileExists{upquote.sty}{\usepackage{upquote}}{}
% use microtype if available
\IfFileExists{microtype.sty}{%
\usepackage{microtype}
\UseMicrotypeSet[protrusion]{basicmath} % disable protrusion for tt fonts
}{}
\usepackage[margin=1in]{geometry}
\usepackage{hyperref}
\hypersetup{unicode=true,
            pdftitle={Reading reflections},
            pdfauthor={Shen Qu},
            pdfborder={0 0 0},
            breaklinks=true}
\urlstyle{same}  % don't use monospace font for urls
\usepackage{graphicx,grffile}
\makeatletter
\def\maxwidth{\ifdim\Gin@nat@width>\linewidth\linewidth\else\Gin@nat@width\fi}
\def\maxheight{\ifdim\Gin@nat@height>\textheight\textheight\else\Gin@nat@height\fi}
\makeatother
% Scale images if necessary, so that they will not overflow the page
% margins by default, and it is still possible to overwrite the defaults
% using explicit options in \includegraphics[width, height, ...]{}
\setkeys{Gin}{width=\maxwidth,height=\maxheight,keepaspectratio}
\IfFileExists{parskip.sty}{%
\usepackage{parskip}
}{% else
\setlength{\parindent}{0pt}
\setlength{\parskip}{6pt plus 2pt minus 1pt}
}
\setlength{\emergencystretch}{3em}  % prevent overfull lines
\providecommand{\tightlist}{%
  \setlength{\itemsep}{0pt}\setlength{\parskip}{0pt}}
\setcounter{secnumdepth}{0}
% Redefines (sub)paragraphs to behave more like sections
\ifx\paragraph\undefined\else
\let\oldparagraph\paragraph
\renewcommand{\paragraph}[1]{\oldparagraph{#1}\mbox{}}
\fi
\ifx\subparagraph\undefined\else
\let\oldsubparagraph\subparagraph
\renewcommand{\subparagraph}[1]{\oldsubparagraph{#1}\mbox{}}
\fi

%%% Use protect on footnotes to avoid problems with footnotes in titles
\let\rmarkdownfootnote\footnote%
\def\footnote{\protect\rmarkdownfootnote}

%%% Change title format to be more compact
\usepackage{titling}

% Create subtitle command for use in maketitle
\providecommand{\subtitle}[1]{
  \posttitle{
    \begin{center}\large#1\end{center}
    }
}

\setlength{\droptitle}{-2em}

  \title{Reading reflections}
    \pretitle{\vspace{\droptitle}\centering\huge}
  \posttitle{\par}
  \subtitle{USP 570}
  \author{Shen Qu}
    \preauthor{\centering\large\emph}
  \postauthor{\par}
      \predate{\centering\large\emph}
  \postdate{\par}
    \date{Week 8}


\begin{document}
\maketitle

\begin{itemize}
\tightlist
\item
  The `Diamond of Assenbly'
\end{itemize}

Levinson and Krizek (2018 Chapter.12) introduced some dynamic
interaction mechanisms of transportation. This system is not static.
From motivation to implementation, polictical facotors features
prominently but is hard to capture. Economic factors are helpful for
understanding the dynamic process. The anaylises based on supply and
demand are classical approaches. The supply and demand curves of
microeconomics illustrate that demand drops as the price increase.
Elasticity, how steep or flat of the slop of demand curve, decides the
relationship between quantities and price such as auto tirp is inelastic
in the price of fuel, which means large changes in fuel prices have
small effects on total driving. The existence of consumer suplus proves
that the subsidies on transit are necessary for maximizing the overall
socail benefits.

Except moving along the demand curve, the curve itself could move too.
Income level, service quality, or complement/substitue goods could shift
the location. Any change of the demand and supply could induce traffic
and development. As shown by the `diamond of assenbly', The effects of
transportation investment are cyclical. Building infrastructure improves
accessibility by reducing travel times, that increase land value and
economic productivity. It induce more travel demand and lead to
congestion and next round of investment.

\begin{itemize}
\tightlist
\item
  Metro Designing Guide
\end{itemize}

The Draft of ``Designing Livable Streets \& Trails'' is an encyclopedic
guide, which covered all major points of street design: The
expectations, funcetions, elements, integration, design decisions, and
implementation strategyies. It is intresting that the guide consider
some emerging technologies like automated vehicles. The current draft
mentioned that automated vhicles could provide a reliable level of
mobility and street designs should envolve these technologies and
changing demands. But the draft doesn't give more about the proposed
solution or plan. The potential may include how to utilize the curve
space, pick-up and drop-off area, etc. Although automated vehicle still
not be offered, it worth exploring now.

\begin{itemize}
\tightlist
\item
  Discussion: dynamics and causality
\end{itemize}

The chaper 12 (Levinson and Krizek 2018) involved two difficult conepts
of dyanamics and cauality. As the author said, moving along or shifting
a demand curve, we can calculate the change of quantity and price. The
reality is the change only represents one moment of the whole proccess.
For incomplete and asymmetric information, each moment has some
residuals and unknown variables. When come to the cycle of effects in
the `Diamond of Assembly', the accumulated error make the precise
prediction impossible. If the cycle keep replicating, we can keep
observing and find the rule. However, there are always some new events
happened or new variables added in each round. In many circumstances,
even the direction of factor effects, positive or negative, is a
question. In case of automated/connected vehicles, we are interested in
their potential impacts on current transportation system. These
technology should change the travel cost and demand, then induce more
traffic and development. They can increase the efficiency of the system
while produce more VMT. It can saving many parking lots but may lead to
wider spawl. There are so many factor change at the same time. It is
even hard to say what will happened at the second step of the cycle.

Causalyity is a more challenging topic. John Stuart Mill gives the three
conditions: ``Concomitant variation is the extent to which a cause X and
an effect Y occur together or vary together in the way predicted by the
hypothesis under consideration; The time order of occurrence condition
states that the causing event must occur either before or simultaneously
with the effect; it cannot occur after- wards; The absence of other
possible causal factors means that the factor or variable being
investigated should be the only possible causal explanation.'' The last
condition represents the `sufficiency' principle, which is the most
difficult to satisfy and not feasible for most research on
transprotation. The second one only can reject the effects of Y on X but
cannot prove the causality bewteen them. The first condition reflect the
equivariance principle, which includes `measurement equivariance' and
`formal invariance'. Researcher often use former approach, use previous
models and modify some assumptions or parameters in simulation like
cheaper travel cost hasing longer VMT. The part of `formal invariance'
is critical for applicability but is still hard to comfirm. Many scholar
still keen on this field because they believe AVs and ridesharing could
fill the gap between the `Mount Transit' and `Mount Auto.' The
transcendental empiricism by Deleuze might give us some inspiration.

\hypertarget{references}{%
\section*{References}\label{references}}
\addcontentsline{toc}{section}{References}

\hypertarget{refs}{}
\leavevmode\hypertarget{ref-levinson2018metropolitan}{}%
Levinson, David M, and Kevin J Krizek. 2018. \emph{Metropolitan Land Use
and Transport: Planning for Place and Plexus}. Routledge.
\url{https://doi.org/10.4324/9781315684482}.


\end{document}
