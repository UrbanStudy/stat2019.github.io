\documentclass[12pt,]{article}
\usepackage{lmodern}
\usepackage{amssymb,amsmath}
\usepackage{ifxetex,ifluatex}
\usepackage{fixltx2e} % provides \textsubscript
\ifnum 0\ifxetex 1\fi\ifluatex 1\fi=0 % if pdftex
  \usepackage[T1]{fontenc}
  \usepackage[utf8]{inputenc}
\else % if luatex or xelatex
  \ifxetex
    \usepackage{mathspec}
  \else
    \usepackage{fontspec}
  \fi
  \defaultfontfeatures{Ligatures=TeX,Scale=MatchLowercase}
    \setmainfont[]{Times New Roman}
\fi
% use upquote if available, for straight quotes in verbatim environments
\IfFileExists{upquote.sty}{\usepackage{upquote}}{}
% use microtype if available
\IfFileExists{microtype.sty}{%
\usepackage{microtype}
\UseMicrotypeSet[protrusion]{basicmath} % disable protrusion for tt fonts
}{}
\usepackage[margin=1in]{geometry}
\usepackage{hyperref}
\hypersetup{unicode=true,
            pdftitle={Reading reflections},
            pdfauthor={Shen Qu},
            pdfborder={0 0 0},
            breaklinks=true}
\urlstyle{same}  % don't use monospace font for urls
\usepackage{graphicx,grffile}
\makeatletter
\def\maxwidth{\ifdim\Gin@nat@width>\linewidth\linewidth\else\Gin@nat@width\fi}
\def\maxheight{\ifdim\Gin@nat@height>\textheight\textheight\else\Gin@nat@height\fi}
\makeatother
% Scale images if necessary, so that they will not overflow the page
% margins by default, and it is still possible to overwrite the defaults
% using explicit options in \includegraphics[width, height, ...]{}
\setkeys{Gin}{width=\maxwidth,height=\maxheight,keepaspectratio}
\IfFileExists{parskip.sty}{%
\usepackage{parskip}
}{% else
\setlength{\parindent}{0pt}
\setlength{\parskip}{6pt plus 2pt minus 1pt}
}
\setlength{\emergencystretch}{3em}  % prevent overfull lines
\providecommand{\tightlist}{%
  \setlength{\itemsep}{0pt}\setlength{\parskip}{0pt}}
\setcounter{secnumdepth}{0}
% Redefines (sub)paragraphs to behave more like sections
\ifx\paragraph\undefined\else
\let\oldparagraph\paragraph
\renewcommand{\paragraph}[1]{\oldparagraph{#1}\mbox{}}
\fi
\ifx\subparagraph\undefined\else
\let\oldsubparagraph\subparagraph
\renewcommand{\subparagraph}[1]{\oldsubparagraph{#1}\mbox{}}
\fi

%%% Use protect on footnotes to avoid problems with footnotes in titles
\let\rmarkdownfootnote\footnote%
\def\footnote{\protect\rmarkdownfootnote}

%%% Change title format to be more compact
\usepackage{titling}

% Create subtitle command for use in maketitle
\providecommand{\subtitle}[1]{
  \posttitle{
    \begin{center}\large#1\end{center}
    }
}

\setlength{\droptitle}{-2em}

  \title{Reading reflections}
    \pretitle{\vspace{\droptitle}\centering\huge}
  \posttitle{\par}
  \subtitle{USP 570}
  \author{Shen Qu}
    \preauthor{\centering\large\emph}
  \postauthor{\par}
      \predate{\centering\large\emph}
  \postdate{\par}
    \date{Week 9}


\begin{document}
\maketitle

\begin{itemize}
\tightlist
\item
  Present and future
\end{itemize}

At the end of this book (Levinson and Krizek 2018 Chapter.14), the
authors give the reader some advice. Firstly, ``Do not harm'' point out
the several typical mistakes of the past. Some policies ``interfere with
the healthy functioning of networks and neighborhoods. Many have equally
detrimental effects.'' Such as low-density development, zoning
monoculture, minimum parking requirements, and etc. Secondly,
``Evidence-based practices'' emphasize positivism and empiricism, which
``base decisions on facts and logic, not ideology, hunches, fads, or
poorly understood the experience.'' In the third section, the authors
also notice the counter-argument to positivism. Because of the lag time,
confounding effects, and the complexity of society, the evidence-based
research about transportation and land use have many limitations. In the
fourth caution, the authors repeat their value of incrementalism for the
``path dependency resulting from political, institutional, and physical
systems.'' At the end of this book, the authors state their thoughts
about how the future life, work, and travel could influence the
transportation and land use field.

\begin{itemize}
\tightlist
\item
  The future with AVs
\end{itemize}

In the APA report, Crute et al. (2018) introduced the concept of
Autonomous Vehicles (AVs) and how they work. This emerging technology
may bring both benefits and challenges to urban and communities. The
potential impacts on land use include parking, sprawl, and other
redevelopments. The report suggests keeping eyes on the parking
standards and requirements, to bolster transportation demand management,
to reconsider the right-of-way, and to continue to develop transit.

\begin{itemize}
\tightlist
\item
  Discussion:
\end{itemize}

Levinson and Krizek (2018) summarized their values and expectations in a
prose style. This chapter reflects the contradictions between
incrementalism and strategic planning in the transportation and land use
field. The incremental, empirical methodology and epistemology are good
at given conditions, are good in the middle section of one historical
stage. For example, although the public often votes against high density
in many cases, at least scholars reach a consensus on compact
development. Why ``land use regulations that reduce development
densities'' is a mistake? Because today we realized and anticipated some
bad results of the sprawl that based on the previous observation and
evidence. Supposing now there is only one-tenth of the population as
before, or we successfully achieve immigration Mars, will we still
insist these opinions? The human's rationality is limited and expedient,
is subject to the context constraints. The human instinct of getting
higher mobility and occupying more living space is stronger and
permanent. Why the land use density and transit share are still not
ideal despite many efforts for the past few decades? Because we still
live in the auto era. The travel cost is significantly less than that in
the last era. That is the ``confines of mature Systems'' mentioned by
Levinson and Krizek (2018).

The authors believe ``while the coefficients within the equations will
change, the equations will remain the same.'' I believe the changes are
not merely about quantities or parameters, are about the distributions.
One hundred years ago, the automobile dominated the street because it
released human instinct rather than restricting it. Many troubles at the
old age gone with the wheel and new troubles emerged. Now when walking
to the node of history, we need to put down some granted experiences and
mindset.

Crute et al. (2018) give a comprehensive prospect for the AVs era,
including rethinking the parking standard, right-of-way, and improving
transit. But these approaches are hoping the new technology to solve the
old problems belong to the current age, rather than propose some
strategies towards a new age. It is similar to say, in one hundred years
ago, ``By the newly invented automobile, we have the opportunity to
completely solve the problem of horse dung everywhere on the street.''
It was correct for the faithful empiricist, but it only answered a small
part of the questions. We need some revolutionary thinking to match a
revolutionary technology, even if most of them proved to be wrong one
hundred years later.

\hypertarget{references}{%
\section*{References}\label{references}}
\addcontentsline{toc}{section}{References}

\hypertarget{refs}{}
\leavevmode\hypertarget{ref-APA2018autonomous}{}%
Crute, Jeremy, William Riggs, Timothy Stewart Chapin, and Lindsay
Stevens. 2018. ``Planning for Autonomous Mobility.'' PAS Report 592.
American Planning Association.

\leavevmode\hypertarget{ref-levinson2018metropolitan}{}%
Levinson, David M, and Kevin J Krizek. 2018. \emph{Metropolitan Land Use
and Transport: Planning for Place and Plexus}. Routledge.
\url{https://doi.org/10.4324/9781315684482}.


\end{document}
