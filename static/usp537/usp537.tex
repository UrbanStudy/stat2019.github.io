\documentclass[10pt,portrait]{article}
\usepackage{multicol}
\usepackage{calc}
\usepackage{ifthen}
\usepackage[portrait]{geometry}
\usepackage{hyperref}

% This sets page margins to .5 inch if using letter paper, and to 1cm
% if using A4 paper. (This probably isn't strictly necessary.)
% If using another size paper, use default 1cm margins.
\ifthenelse{\lengthtest { \paperwidth = 8.5in}}
	{ \geometry{top=.2in,left=.1in,right=.1in,bottom=.2in} }

% Turn off header and footer
\pagestyle{empty}

% Redefine section commands to use less space
\makeatletter
\renewcommand{\section}{\@startsection{section}{1}{0mm}%
                                {-1ex plus -.5ex minus -.2ex}%
                                {0.5ex plus .2ex}%x
                                {\normalfont\normalsize\bfseries}}
\renewcommand{\subsection}{\@startsection{subsection}{2}{0mm}%
                                {-1explus -.5ex minus -.2ex}%
                                {0.5ex plus .2ex}%
                                {\normalfont\small\bfseries}}
\renewcommand{\subsubsection}{\@startsection{subsubsection}{3}{0mm}%
                                {-1ex plus -.5ex minus -.2ex}%
                                {1ex plus .2ex}%
                                {\normalfont\small\bfseries}}
\makeatother

% Define BibTeX command
\def\BibTeX{{\rm B\kern-.05em{\sc i\kern-.025em b}\kern-.08em
    T\kern-.1667em\lower.7ex\hbox{E}\kern-.125emX}}

% Don't print section numbers
\setcounter{secnumdepth}{0}


\setlength{\parindent}{0pt}
\setlength{\parskip}{0pt plus 0.1ex}


% -----------------------------------------------------------------------

\begin{document}

\raggedright
\footnotesize
\begin{multicols}{2}


% multicol parameters
% These lengths are set only within the two main columns
%\setlength{\columnseprule}{0.25pt}
\setlength{\premulticols}{.5pt}
\setlength{\postmulticols}{.5pt}
\setlength{\multicolsep}{.5pt}
\setlength{\columnsep}{1pt}

%\begin{center}
%     \Large{\textbf{STAT564}} \\
%\end{center}

\section{537 mid-term}
\subsection{Demand}
Law of demand: the quantity demanded of a good falls when the price of the good rises, other things equal; downward sloping.
Derived from consumer utility maximization theory. Demand curve represents MB (marginal benefit)

\subsubsection{Elasticities}
Price elasticity of demand: percentage change of quantity demanded in response to a 1\% change in price. 

Inelastic indicates that the quantity demanded is not very sensitive to price changes.

the fare elasticity of demand for transit service varied across income groups. The elasticity of demand was greatest for the low income group, and lowest for the high income group.

$\eta=\frac{\Delta Q/Q}{\Delta P/P}$

Income elasticity of demand

$\eta=\frac{{\Delta Q}/Q}{{\Delta In}/In}=\beta_i\frac{\bar{In}}{\bar{Q}}$

Cross-elasticity of demand for auto travel with respect to fare for transit trips, for peak-period rail travel with respect to the off-peak rail fare. The positive sign indicates that peak travel and off-peak travel tend to be substitutes.

$\eta_{Cross}=\frac{\Delta Q_{Auto}/Q_{Auto}}{\Delta P_{Rail}/P_{Rail}}=\frac{\Delta Q_{Peak}/Q_{Peak}}{\Delta P_{Off}/P_{Off}}$

A substitution good may make the price elasticity of demand for transit more elastic, because people now have more alternatives when price of transit changes. 

A complementary good may or may not change the price elasticity of demand for transit itself (although the cross-price elasticity will be negative in this case).

\subsubsection{long-run price elasticity}

the long-run price elasticity of demand to be larger than in the short-run. This is because consumers have more flexibility to change their behaviors in the long-run. With a fare increase, consumers may be able to make larger changes such as move to a different location, change jobs, switch modes (e.g., get a car or a bike), negotiate work-from-home days, etc.

\subsection{Supply}
Law of supply: an increase in price results in an increase in quantity supplied, all else equal.
Derived from producer profit maximization theory. Supply curve represents MC (marginal cost)

set the price to maximize efficiency $P^*=MC$, there is no producer surplus or deadweight loss.

maximize profit by setting the marginal cost equal to marginal revenue MC=MR. This corresponds to an efficiency (deadweight) loss. Profit=TR-TC Monopolist profit-maximization; 

\subsubsection{Fares - revenue}
TR=PQ, $\%\Delta_{Rev}=\frac{P_2Q_2-P_1Q_1}{P_1Q_1}$

Fares would need to be lowered by about 67\% in order to achieve a 20\% increase in transit usage. Combined, this would mean TriMet would take in about 60.4\% less in farebox revenue.

\subsubsection{TC=VC+FC}
If the extra riderships don’t exceed current transit capacity, fixed cost would not change. The variable cost will rise for the addtional operation cost such as energy consumption and labor cost. Thus, the total cost would increase. If current transit is underload, the change would be tiny.

\subsubsection{subsidy}
set the price to maximize efficiency P=MC, there are benefits accruing to society in the form of consumer surplus that outweigh the subsidy needed to provide the train service. Thus, there is still a net economic benefit to society as a result of the train service. The subsidy could be eliminated if enough of the consumer surplus could be translated into fare payments that fund the train service. examples could be a varying pricing scheme based on willingness-to-pay or a higher tax directed toward those who benefit the most from the flat price. Even the monopolist is providing a consumer surplus that is greater than their required subsidy. It would not be a mistake to build the train if sufficient value-capture mechanisms were in place to offset the subsidy required. In addition, there may be additional positive externalities or equity considerations that are not evident in the economic efficiency calculations.

When raising the price TC=TR, no subsidies would now be needed , but the consumer surplus would be decreased.

New riders attracted by the new lower fare received some consumer surplus resulting from their travel mode change. The additional consumer surplus for the existing rail riders essentially translate into benefits for the consumer without any benefits for the transit agency.

\subsubsection{externality}

Charging road users the full social cost
The principle of road pricing is to balance the cost and externality. Because the road using costs do not reflect true costs to society and environment, users are not paying the full costs of their purchases and even unaware of it. They will tend to use too much that is subsidized by society. The markets are distorted and inefficient.

\subsubsection{demand-led and supply-led}

Supply-led view suggests that improving transportation infrastructure will lead to widening of markets, increased production and multiplier effects. There may also be indirect effects in construction and operation. Demand-led view suggests that transport provision should be a response to a basic demand, which could be revealed demand (current journeys or trips that are made using existing infrastructure which may be congested) or latent demand (demand for trips that cannot be satisfied due to current inadequacies in the existing infrastructure). Demand-led transportation initiatives are typically responses to high congestion and demand for transportation in developed areas, whereas supply-led initiatives are aimed at connecting markets and spurring economic development.

Causal processes can be identified by their ability to transmit an alteration over space and time.  in the early and late stages, the project is a relatively strong factor to the economy. Supply-led transportation stimulates economic development. During the middle stages, the project is a relatively weak factor. When the slope is approaching 1, which means the project may leverage the economic development; when the slope is away from 1 to the inflection point, the project may Lags behind the economy and need ahead investment. After the inflection point, the project will have less and less stimulating effects. When the slope is away from 1 again and convergent to 0, the policy decision for a megaproject should be more conservative. 

\subsubsection{Tempo-spatial}
Cross-sectional data is a type of data collected by observing many subjects (such as individuals, firms, countries, or regions) at the same point of time, or without regard to differences in time. Analysis of cross-sectional data usually consists of comparing the differences among the subjects.

A time series is a series of data points indexed (or listed or graphed) in time order. Time series data have a natural temporal ordering. cross-sectional studies is no natural ordering of the observations 

\subsubsection{aggregate demand model}

$\widehat{\Delta G}=\beta1*\Delta Pr+\beta2*\Delta In$


\subsubsection{Price change}

The price effect is the sum of substitution and income effects. Every price change can be decomposed into an income effect and a substitution effect. 

The substitution effect reflects the effect of the changed relative prices of the two goods, which alters the slope of the budget constraint but leaves the consumer on the same indifference curve.  

The income effect is the phenomenon observed through changes in purchasing power.It reveals the change in quantity demanded brought by a change in real income. Graphically, as long as the prices remain constant, changing income will create a parallel shift of the budget constraint. Increasing the income will shift the budget constraint right since more of both can be bought, and decreasing income will shift it left.

For a normal goods, the income effect from the rise in purchasing power from a price fall reinforces the substitution effect. For an inferior good, then the income effect will offset in some degree the substitution effect.

\end{multicols}

\begin{tabular}{l|l|l|l|l|l|l|l|l|l|l}
P(\$)&Rider&Fixed Cost&Variable Cost&Total Cost&TR&subsidies/profit&CS&PS&Total Surplus& DWL\\
P  & Q &FC &$P^*$Q &FC+VC  &PQ &(P-$P^*$)Q-FC &(10-P)Q/2&(P-$P^*$)Q &(10+P-2$P^*$)Q/2&$\frac12\Delta P\Delta Q$
\end{tabular}

\end{document}
