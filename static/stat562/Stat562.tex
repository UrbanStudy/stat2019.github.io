\documentclass[10pt,landscape]{article}
\usepackage{multicol}
\usepackage{calc}
\usepackage{ifthen}
\usepackage[landscape]{geometry}
\usepackage{hyperref}
\usepackage{amsmath, amssymb, amsthm}
%\usepackage{bbm}




\usepackage{graphicx}

\usepackage{latexsym, marvosym}
\usepackage{pifont}
\usepackage{lscape}
\usepackage{graphicx}
\usepackage{array}
\usepackage{booktabs}
\usepackage[bottom]{footmisc}
\usepackage{tikz}
\usetikzlibrary{shapes}
\usepackage{pdfpages}
\usepackage{wrapfig}
\usepackage{enumitem}
\setlist[description]{leftmargin=0pt}
\usepackage{xfrac}

\usepackage{relsize}
\usepackage{rotating}

 \newcommand\independent{\protect\mathpalette{\protect\independenT}{\perp}}
    \def\independenT#1#2{\mathrel{\setbox0\hbox{$#1#2$}%
    \copy0\kern-\wd0\mkern4mu\box0}} 

%\newenvironment{proof}[1][Proof]{\begin{trivlist}
%\item[\hskip \labelsep {\bfseries #1}]}{\end{trivlist}}
\newenvironment{definition}[1][Definition]{\begin{trivlist}
\item[\hskip \labelsep {\bfseries #1}]}{\end{trivlist}}
\newenvironment{example}[1][Example]{\begin{trivlist}
\item[\hskip \labelsep {\bfseries #1}]}{\end{trivlist}}
\newenvironment{remark}[1][Remark]{\begin{trivlist}
\item[\hskip \labelsep {\bfseries #1}]}{\end{trivlist}}

% To make this come out properly in landscape mode, do one of the following
% 1.
%  pdflatex Statisticscheat.tex
%
% 2.
%  latex Statisticscheat.tex
%  dvips -P pdf  -t landscape Statisticscheat.dvi
%  ps2pdf Statisticscheat.ps

\newcommand{\noin}{\noindent}    
\newcommand{\logit}{\textrm{logit}} 
\newcommand{\var}{\textrm{Var}}
\newcommand{\cov}{\textrm{Cov}} 
\newcommand{\corr}{\textrm{Corr}} 
\newcommand{\N}{\mathcal{N}}
\newcommand{\Bern}{\textrm{Bern}}
\newcommand{\Bin}{\textrm{Bin}}
\newcommand{\Beta}{\textrm{Beta}}
\newcommand{\Gam}{\textrm{Gamma}}
\newcommand{\Expo}{\textrm{Expo}}
\newcommand{\Pois}{\textrm{Pois}}
\newcommand{\Unif}{\textrm{Unif}}
\newcommand{\Geom}{\textrm{Geom}}
\newcommand{\NBin}{\textrm{NBin}}
\newcommand{\Hypergeometric}{\textrm{HGeom}}
\newcommand{\HGeom}{\textrm{HGeom}}
\newcommand{\Mult}{\textrm{Mult}}

% This sets page margins to .5 inch if using letter paper, and to 1cm
% if using A4 paper. (This probably isn't strictly necessary.)
% If using another size paper, use default 1cm margins.
\ifthenelse{\lengthtest { \paperwidth = 11in}}
	{ \geometry{top=.2in,left=.2in,right=.2in,bottom=.2in} }
	{\ifthenelse{ \lengthtest{ \paperwidth = 297mm}}
		{\geometry{top=1cm,left=1cm,right=1cm,bottom=1cm} }
		{\geometry{top=1cm,left=1cm,right=1cm,bottom=1cm} }
	}

% Turn off header and footer
\pagestyle{empty}
 

% Redefine section commands to use less space
\makeatletter
\renewcommand{\section}{\@startsection{section}{1}{0mm}%
                                {-.2ex plus -.5ex minus -.2ex}%
                                {0.5ex plus .2ex}%x
                                {\normalfont\large\bfseries}}
\renewcommand{\subsection}{\@startsection{subsection}{2}{0mm}%
                                {-0.2ex plus -.2ex minus -.2ex}%
                                {0.2ex plus .2ex}%
                                {\normalfont\normalsize\bfseries}}
\renewcommand{\subsubsection}{\@startsection{subsubsection}{3}{0mm}%
                                {-.2ex plus -.2ex minus -.2ex}%
                                {.2ex plus .2ex}%
                                {\normalfont\small\bfseries}}
\makeatother

% Define BibTeX command
\def\BibTeX{{\rm B\kern-.05em{\sc i\kern-.025em b}\kern-.08em
    T\kern-.1667em\lower.7ex\hbox{E}\kern-.125emX}}

% Don't print section numbers
\setcounter{secnumdepth}{0}


\setlength{\parindent}{0pt}
\setlength{\parskip}{0pt plus 0.5ex}


% -----------------------------------------------------------------------

\begin{document}

\raggedright
\footnotesize
\begin{multicols}{1}


% multicol parameters
% These lengths are set only within the two main columns
%\setlength{\columnseprule}{0.25pt}
\setlength{\premulticols}{1pt}
\setlength{\postmulticols}{1pt}
\setlength{\multicolsep}{1pt}
\setlength{\columnsep}{2pt}



\subsubsection{4.1 Joint and Marginal}

$f_X(x)=\sum_{y\in\mathbf{R}}f_{X,Y}(x,y)\ \text{and}\ f_Y(y)=\sum_{x\in\mathbf{R}}f_{X,Y}(x,y)$
$f_X(x)=\int_{-\infty}^{\infty}f(x,y)dy, -\infty<x<\infty\quad f_Y(y)=\int_{-\infty}^{\infty}f(x,y)dx, -\infty<x<\infty$


$E[g(\vec X)]=\begin{cases}\sum\cdots\sum_{all\vec x} g(\vec x)p(\vec x) \\ \idotsint_{\mathbf{R^n}}g(\vec x)f(\vec x)d\vec x\end{cases}$

$Eg(X,Y)=\int_{-\infty}^{\infty}\int_{-\infty}^{\infty} g(x,y)f(x,y)dxdy$

$F(x,y) = P(X\le x, Y\le y)$

$F(x,y) =\int_{-\infty}^{y}\int_{-\infty}^{x}f(s,t)dsdt$

\subsubsection{4.2 conditional and independent}

$f(x_2|x_1)=\frac{f(x_1,x_2)}{f_1(x_1)}$;$p(x_2|x_1)=\frac{p(x_1,x_2)}{p_1(x_1)}$

4.2.1 $f(y|x)=P(Y=y|X=x)=\frac{f(x,y)}{f_X(x)}$


\textbf{Independence}: 

\begin{tabular}{l|l|l|l}
4.2.7   & $f(x,y)=g(x)h(y)$ & & $F_{X,Y}(x,y)=F_X(x)F_Y(y)$\\
4.2.10a & $P(X\le x,Y\le y)=P(X\le x)P(Y\le y)$ & 4.2.10b & $E(XY)=E(X)E(Y)$\\
4.2.12 & $M_Z(t)=M_X(t)M_Y(t)$ & 4.6.7 & $M_Z(t)=(M_X(t))^n$ \\
4.3.5 & $X,Y$ indep r.v. $g(X)$,$h(Y)$ indep & 4.5.6 & $V(X\pm Y) =VX+VY$ 
\end{tabular}


4.2.10 $E(g(X)h(Y))=\int_{-\infty}^{\infty}\int_{-\infty}^{\infty}g(x)h(y)f(x,y)dxdy=\int_{-\infty}^{\infty}h(y)f_Y(y)\left[\int_{-\infty}^{\infty}g(x)f_X(x)dx\right]dy=E[g(x)]E[h(y)]=(Eg(X))(Eh(Y))$

Use 4.2.12 proof 4.3.2 $M_W(t)=M_X(t)M_Y(t)=e^{\mu_1(e^t-1)}e^{\mu_2(e^t-1)}=e^{(\mu_1+\mu_2)(e^t-1)}$

4.6.6 $X_1,.., X_n$ indep, $E(g_1(X_1)\cdots g_n(X_n))=(E(g_1(X_1))\cdots(E(g_n(X_n))$

$U\sim Geom(p=\frac12),u=1,2..$ the number of trials needed to get the first head.

$V\sim N Bin(p=\frac12,r=2),v=2,3..$ the number of trials needed to get two heads in repeated tosses of a fair coin.

the distribution of (U,V) is $\{(u,v):u=1,2,..;v=u+1,u+2,..\}$ is not a cross-product set. U and V are not independent.

\begin{tabular}{l|l|l}
4.2.14 & $X\sim n(\mu,\sigma^2)$ & $Y\sim n(\gamma,\tau^2)$\\
$X,Y$ indep &$Z=X+Y$ & $Z\sim n(\mu+\gamma,\sigma^2+\tau^2)$
\end{tabular}

\subsubsection{4.3 Transform}

\begin{tabular}{l|l|l}
4.3.2 & $X\sim Poisson(\theta)$ & $Y\sim Poisson(\lambda)$\\
$X,Y$ indep &$Z=X+Y$ & $Z\sim Poisson(\theta+\lambda)$
\end{tabular}

\begin{tabular}{l|l|l}
4.3.3 & $X\sim Beta(\alpha,\beta)$ & $Y\sim Beta(\alpha+\beta,\gamma)$\\
$X,Y$ indep &$U=XY$ & $f_U(u)\sim n(\alpha,\beta+\gamma)$
\end{tabular}

\begin{tabular}{l|l|l}
4.3.4 & $X\sim n(0,1)$ & $Y\sim n(0,1)$\\
$X,Y$ indep &$X-Y\sim n(0,2)$ & $X-Y\sim n(0,2)$
\end{tabular}

4.3.2 $f_{Y_1,Y_2}(y_1,y_2)=\sum_{i=1}^kf_{X_1,X_2}(h_1(y_1,y_2), h_2(y_1,y_2))|J|$

4.3.5 $f_{U,V}(u,v)=\sum_{i=1}^kf_{X,Y}(h_{1i}(u,v),\ h_{2i}(u,v))|J_i|$

$J_{1,2}=\begin{vmatrix}\frac{\partial h_1}{\partial y_1} & \frac{\partial h_2}{\partial y_2} \\ \frac{\partial h_1}{\partial y_1} & \frac{\partial h_2}{\partial y_2} \end{vmatrix}=\frac{\partial h_1}{\partial y_1}\frac{\partial h_2}{\partial y_2}-\frac{\partial h_2}{\partial y_1}\frac{\partial h_1}{\partial y_2}$

\begin{tabular}{l|l|l}
4.3.6 & $X\sim n(0,1)$ & $Y\sim n(0,1)$\\
$X,Y$ indep & $U=X/Y$ & $U\sim cauchy(0,1)$
\end{tabular}

\subsubsection{4.4 Mixture Distributions}


4.2.3 $E(g(x_2)|x_1)=\sum_{all\ x_2}g(x_2)p(x_2|x_1)$,$E(g(x_2)|x_1)=\int_{all\ x_2}g(x_2)f(x_2|x_1)dx_2$

\textbf{Expectations}

$E(aX+b) = aE(X) + b$ 

$E(X)=\int_{-\infty}^{0}F_X(t)dt+\int_{0}^{\infty}F_X(t)dt$

$E[g(x)]=\mu=\sum_{x\in D}h(x)p(x)=\int_{-\infty}^{\infty}g(x)f(x)dx$ 

$E[(X-\mu)^n]=\mu_n=\sum{(x-\mu)^n}p(x)=\int{(x-\mu)^n}f(x)dx$

4.4.3 $EX=E[E(X|Y)]$

\textbf{Variances}

$V[X|Y]=E[(X-E[X|Y])^2|Y]$

$V[aX+b]=a^2\sigma^2$
$\sigma_{ax+b} =|a|\sigma_{x}$

$V[X]=\sigma_{x}^2=E(X^2)-[E(X)]^2$

4.2.4 $V[y|x]=E[Y^2|x]-(E[Y|x])^2$

4.4.7 $V[X]=E[V(X|Y)]+V[E(X|Y)]$

$V(X\pm Y) =VX+VY\pm 2Cov(X,Y)$

\begin{tabular}{ l|l }
\end{tabular}

\subsubsection{4.5 Covariance and Correlation}

$Cov(x,y)=E(XY)-E(X)E(Y)$

4.5.1/3 $Cov(X,Y)=E[(X-\mu_X)(Y-\mu_Y)]=E[XY]-\mu_X\mu_Y=\sigma_{XY}$

4.5.2 $\rho_{XY}=\frac{\sigma_{XY}}{\sigma_X\sigma_Y}$ $Corr(X,Y)=\frac{Cov(X,Y)}{\sqrt{Var(X)Var(Y)}}$

$Cov(aX,bY)=abCov(X,Y)$

$Cov(X,Y+Z)=Cov(X,Y)+Cov(X,Z)$

$Cov(X,c)=0$

4.5.6 $Var(aX+bY)=a^2VarX+b^2VarY+2abCov(X,Y)$

4.5.7 $E[Y|X]=a+bx$, $E[Y]=E[E[Y|X]]=E[a+bX]=a+bE[X]$ by 4.4.3, $E[XE[Y|X]]=E[X(a+bX)]=aE[X]+bE[X^2]$, $E[XE[Y|X]]=\int_{-\infty}^{\infty}xE[Y|x]f_X(x)dx$ by 2.2.1, $=\int_{-\infty}^{\infty}x\left[\int_{-\infty}^{\infty}yf_Y(y|x)dy\right]f_X(x)dx$ by 4.2.3,$=\int_{-\infty}^{\infty}\int_{-\infty}^{\infty}xyf(x,y)dydx=E[XY]$ by 4.1.10, $\sigma_{XY}=E[XY]-\mu_X\mu_Y=a\mu_X+bE[X^2]-\mu_X\mu_Y=a\mu_X+b(\sigma_X^2+\mu_X^2)-\mu_X(a+b\mu_X)=b\sigma_X^2$

$\rho_{XY}=\frac{\sigma_{XY}}{\sigma_{X}\sigma_{Y}}=\frac{b\sigma_X^2}{\sigma_{X}\sigma_{Y}}=b\frac{\sigma_X}{\sigma_Y}$

4.5.10 bivarialte normal pdf with $\mu_X,\mu_Y,\sigma_X^2,\sigma_Y^2, \rho$

$f_X(x)\sim n(\mu_X,\sigma^2_X)$ $f_Y(y)\sim n(\mu_Y,\sigma^2_Y)$

$f_{Y|X}(y|x) \sim n(\mu_Y+\rho\frac{\sigma_Y}{\sigma_X})(x−\mu_X),\sigma_Y^2(1-\rho^2)$

$aX+bY\sim n(a\mu_X+b\mu_Y,a^2\mu_X^2+b^2\mu_Y^2+2ab\rho\sigma_X\sigma_Y$

\begin{tabular}{l|l|l}
4.6.8 & $X_1,.., X_n$  & $X_i\sim gamma(\alpha,\beta)$\\
indep & $Z=X_1+..+X_n$ & $Z\sim gamma(\alpha_1+..+\alpha_n,\beta)$
\end{tabular}



\subsection{5.2 Sum of r.s.}

5.2.2 $\bar X=\frac{X_1+..+X_n}{n}=\frac1n\sum_{i=1}^nX_i$

5.2.3 $S^2=\frac1{n-1}\sum_{i=1}^n(X_i-\bar X)^2=\frac1{n-1}\sum_{i=1}^n(X_i^2-n\bar X^2)$

5.2.5 $E\left(\sum_{i=1}^ng(X_i) \right)=nE(g(X_1))$ $Var\left(\sum_{i=1}^ng(X_i) \right)=nVar(g(X_1))$

5.2.6 $E\bar X=\mu$, $Var\bar X=\frac{\sigma^2}n$, $ES^2=\sigma^2$

5.2.7 $M_{\bar X}(t)=[M_X(\frac{t}n)]^n$

\begin{tabular}{l|l|l}
5.2.8 & $X_1,.., X_n\sim N(\mu,\sigma^2)$ & $\bar X\sim N(\mu,\sigma^2/n)$\\
indep & $X_1,.., X_n\sim gamma(\alpha,\beta)$ & $\bar X\sim gamma(n\alpha,\beta/n)$
\end{tabular}

5.2.9 Convolution formula $Z=X+Y$ $f_Z(z) =\int_{-\infty}^{\infty}f_X(w)f_Y(z-w)dw$

\begin{tabular}{l|l|l|l}
5.2.10 & $X\sim cauchy(0,\sigma)$ & $Y\sim cauchy(0,\tau)$ & $X_1,.., X_n\sim cauchy(0,\sigma)$\\
indep & $Z=X/Y$ & $g_Z(z)\sim cauchy(0,\sigma+\tau)$ & $bar X\sim cauchy(0,\sigma),\sum_{1}^{n} X\sim cauchy(0,n\sigma)$
\end{tabular}


\subsection{5.3 Sampling from N}

$X_1..X_n\sim iid N(\mu,\sigma^2)$. 

\textbf{Properties of $\bar X$ and $S^2$}

5.3.1 when $X_1,..,X_n$ be iid $n(\mu,\sigma^2)$
\begin{tabular}{l|lll}
W  Normal& $\bar X\sim N(\mu,\frac{\sigma^2}n)$ & $\bar X,S^2$ indep & $\frac{(n-1)S^2}{\sigma^2}\sim\chi^2_{n-1}$\\
W/o Normal & $E[\bar X]=\mu$ & $V[\bar X]=\frac{\sigma^2}n)$ & $E[S^2]=\sigma^2$\\\hline
\end{tabular}

$f_{\chi^2}(x)=\frac{x^{\frac{p}2-1}}{\Gamma \frac{p}2 2^{\frac{p}2}}e^{-\frac{x}2}, x>0$

\begin{tabular}{ l|l|l }
5.3.2a & $Z\sim n(0,1)$ & $Z^2\sim \chi^2_1$ \\ \hline
5.3.2b & $X_1..X_n$     & $X_i\sim \chi^2_{p_i}$\\
indep  & $Z=X_1+..+X_n$ & $Z\sim \chi^2_{p_1+..p_n}$
\end{tabular}

For $\chi^2_{p}\sim gamma(\frac{p}2,2)$ see 4.6.8

5.3.1 proof $\bar X_{n+1}=\frac{\sum_{i=1}^{n+1}X_{i}}{n+1}=\frac{\sum_{i=1}^{n}X_{i}+X_{n+1}}{n+1}=\frac{n\bar X_{n}+X_{n+1}}{n+1}$

$nS^2_{n+1}=(n-1)S_n^2+(\frac{n}{n+1})(X_{n+1}-\bar X_n)^2$

,$Var\chi^2_{n-1}=2(n-1)$

$Var[\frac{(n-1)S^2}{\sigma^2}]=\frac{(n-1)^2}{\sigma^4}Var[S^2]=2(n-1)\implies Var[S^2]=\frac{2\sigma^4}{n-1}$

\textbf{Connection between $N,\chi^2,t,F$}

5.3.4 $X_1,..X_n\sim n(\mu,\sigma^2)$, $\frac{\bar X-\mu}{{\sigma}/\sqrt{n}}\sim t_{n-1}$

$f_{T}(x)=\frac{\Gamma(\frac{p+1}2)}{\Gamma(\frac{p}2)\sqrt{p\pi}}\left (1+\frac{x^2}{p}\right)^{-\frac{p+1}2},-\infty<x<\infty$

$t_1=Cauchy(0,1)$

5.3.5  $X_1..X_n$,$Y_1..Y_m$ indep. $X_i\sim n(\mu_X,\sigma_X^2)$,$Y_j\sim n(\mu_Y,\sigma_Y^2)$,$\frac{S_X^2}{\sigma_X^2}\sim \chi^2$

$\frac{S_X^2/\sigma_X^2}{S_Y^2/\sigma_Y^2} \sim F$


5.3.6 $f_{F}(x)=\frac{\Gamma(\frac{p+q}2)}{\Gamma(\frac{p}2)\Gamma (\frac{q}2)}(\frac{p}q)^{\frac{p}2}{x^{\frac{p}2-1}\over{(1+\frac{p}qx)^{\frac{p+q}2}}}, x>0$

5.3.8 

a. $X\sim F_(p,q)$ $\frac{1}X \sim F_(q,p)$

b. $X\sim T_(q)$ $X^2 \sim F_(1,q)$

c. $X\sim F_(p,q)$ $\frac{\frac{p}{q}X}{1+\frac{p}{q}X} \sim Beta(\frac{p}2,\frac{q}2)$


\subsection{5.4 Order statistics}

5.4.4 $f_{K}(x)=\frac{n!}{(j-1)!(n-j)!}K[F_X(x)]^{k-1}[1-F_X(x)]^{n-k}f(x)$ 1-29p9

5.4.6 $f_{X_{(i)},X_{(j)}}(u,v)=\frac{n!}{(i-1)!(j-1-i)!(n-j)!}f_X(u)f_X(v)[F_X(u)]^{i-1}[F_X(v)-F_X(u)]^{j-1-i}[1-F_X(v)]^{n-j}$

$f_{X_{(1)},..,X_{(n)}}(x_1,..,x_n)=\begin{cases}n!f_X(x_1)\cdot...\cdot f_X(x_n) &-\infty<x_1<..<x_n<\infty\\0&\text{otherwise}\end{cases}$

\subsection{1 from N to T to Chi to F}
Given some function of these, find the distribution.


\subsection{2 Transformation of pairs of r.v.s}

Given 2 random variables and their joint function, and given a function of them, find its distribution.

1-10p1

\begin{tabular}{ l|l }
$f(x,y)$ & $0<x<\infty,0<y<\infty$ \\
$\frac14e^{-\frac{x+y}2}$ & $u=\frac{X-Y}2$
\end{tabular}

1. $V=Y \to X=2u+v, Y=v$

2. $J=\begin{vmatrix}\frac{\partial x}{\partial u} & \frac{\partial x}{\partial v} \\ \frac{\partial y}{\partial u} & \frac{\partial y}{\partial v} \end{vmatrix}=\begin{vmatrix}2 & 1 \\ 0 & 1 \end{vmatrix}=2$

3. $g(u,v)=f(x,y)|J|=\frac14e^{-\frac{x+y}2}2=\frac12e^{-\frac{2u+v+v}2}=\frac12e^{-(u+v)}$

4. $0<x<\infty,0<y<\infty\implies0<2u+v<\infty,0<v<\infty\implies v>-2u$

5.
\begin{tabular}{ l|l|l }
$g_U(u)=$ & $\int_{-2u}^\infty\frac12e^{-(u+v)}dv=\frac12e^{-u}\int_{-2u}^\infty e^{-v}dv=\frac12e^{-u}\left[-e^{-v}\right]_{-2u}^\infty=\frac12e^{-u}\left[0+e^{2u}\right]$ & $u<0$ \\
$=\frac12e^{|u|}$ & $\int_{0}^\infty\frac12e^{-(u+v)}dv=\frac12e^{-u}\int_{0}^\infty e^{-v}dv=\frac12e^{-u}\left[-e^{-v}\right]_{0}^\infty=\frac12e^{-u}\left[0+1\right]$ & $u\ge0$
\end{tabular}

Distribution: Double Exponential(Laplace)


\subsection{3 }

Given a joint pdf, find the covariance, correlation, conditional expectation,conditional variance.

1-10p10-13 4.5.4/8/8





\subsection{4 Order statistics}

* Find the distribution of X(k) or find the joint distribution of $X_{(j)},X_{(k)}$

 joint pmf 1-31p2-3

5.4.5 uniform order pdf 1-31p4-8




  \subsubsection{pdf/pmf}
CDF: $F(x) = P(X \leq x) = \sum_{y: y \leq x} p(y)= \int_{-\infty}^{x} f(y) dy$ \\
	\textbf{probabilities}: $a \leq b$:\\
PMF: $p(x)= P(X=x) = P(\forall w \in \mathcal{W}: X(w)=x)$ \\
$P(a \leq X \leq b) = F(b) - F(a^{-}) $;
$P(a < X \leq b) = F(b) - F(a) $;
$P(a \leq X \leq a) =p(a)$;
$P(a < X < b) = F(b^{-}) - F(a) $;
(where $a^-$ is the largest possible X value strictly less than $a$);
Taking $a=b$ yields $P(X=a) = F(a) - F(a-1)$ as desired.\\

PDF: $P(\forall w \in \mathcal{W}: a \leq X(w)\leq b) = \int_{a}^{b}f(x) dx$\\
$P(a \leq X \leq b) = P(a < X \leq b) = P(a < X < b) $;
$P(X>a) = 1-F(a) $;
$ P(a \leq X \leq b) = F(b) - F(a) $

	\subsubsection{CDF}
Condition:$f(x)\ge 0 \forall x$
pmf $\sum_xf_X(x)=1, pdf \int_{-\infty}^{\infty}f(x)dx=1$


\subsubsection{mgf}
$E(e^{tx})=\int e^{tx}f(x)dx=\sum e^{tx}f(x)$; $M_{aX+b}(t)=e^{tb}M_{aX}(t)$

\begin{tabular}{ l|l }
$M_X(t)=E(e^{tx})$   & $M_X(0)=1$ \\
$M_X'(t)=E(xe^{tx})$  & $M_X'(0)=E(X)$  \\ 
$M_X''(t)=E(x^2e^{tx})$ & $M_X''(0)=E(X^2)$ \\
$M_X^n(t)=E(x^ne^{tx})$ & $M_X^n(0)=E(X^n)$
\end{tabular}


  \subsubsection{transform}
\begin{tabular}{ l|l }
$g(x)\uparrow$   & $F_Y(y)=F_X(g^{-1}(y))$ \\
$g(x)\downarrow$ & $F_Y(y)=1-F_X(g^{-1}(y))$ 
\end{tabular}

	\textbf{not monotone}
$\because X\le 0\ is\ \emptyset \therefore P(X\le-\sqrt y)=0$\\
$F_Y(y)=P(Y\le y)=P(X^2\le y)=P(-\sqrt y\le X\le\sqrt y)=P(X\le\sqrt y)-P(X\le-\sqrt y)=F_X(\sqrt y)=\sqrt y,\ 0<\sqrt y<1$

	\textbf{monotone}:$f_Y(y)=f_X(g^{-1}(y))|\frac{d(g^{-1}(y))}{dy}|$


  \subsubsection{Series}

\begin{tabular}{ l|l }
Finite                                    & Binomial \\
$\sum_{k=1}^n k=\frac{n(n+1)}{2}$         & $\sum_{k=0}^n \binom{n}{k} = 2^n$ \\
$\sum_{k=1}^n (2k-1)=n^2$                 & $\sum_{k=0}^n \binom{r+k}{k}=\binom{r+n+1}{n}$ \\
$\sum_{k=1}^n k^2=\frac{n(n+1)(2n+1)}{6}$ & $\sum_{k=0}^n \binom{k}{m}=\binom{n+1}{m+1}$ \\
$\sum_{k=1}^n k^3=(\frac{n(n+1)}{2})^2$   & $\sum_{k=0}^r \binom{m}{k}\binom{n}{r-k}=\binom{m+n}{r}$ \\
$\sum_{k=0}^n c^k=\frac{c^{n+1}-1}{c-1}$  & $\sum_{k=0}^n \binom{n}{k}a^{n-k}b^k = (a+b)^n$
\end{tabular}

\begin{tabular}{ l|l }
Infinite $|p|<1$                        & $\sum_{k=0}^\infty kp^{k-1}=\frac{d}{dp}\left(\sum_{k=0}^\infty p^k\right)=\frac{1}{(1-p)^2}$ \\
$\sum_{k=0}^\infty p^k = \frac{1}{1-p}$ & $\sum_{k=0}^\infty \binom{r+k-1}{k} x^k = (1-x)^{-r} r\in\mathbb N^+$ \\
$\sum_{k=1}^\infty p^k = \frac{p}{1-p}$ & $\sum_{k=0}^\infty \binom{\alpha}{k} p^k
    = (1+p)^\alpha \quad |p|<1\,,\,\alpha \in \mathbb C$
\end{tabular}

\begin{tabular}{ l|l }
$\Gamma(n)=(n-1)!$           & $\sum_{n=0}^{\infty}\frac{x^n}{n!}=e^x$ \\
$\Gamma(a+1)=a\Gamma(a)$     & $\sum_{n=0}^{\infty}{ar^n}=\frac{a}{1-r}, |r|<1$ \\
$\Gamma(1/2)=\sqrt\pi$       & $\sum_{k=0}^{n-1}{ar^n}=\frac{a(1-r^n)}{1-r}$ \\
$\Gamma(0)=\Gamma(-1)=\infty$& $\int_0^1 x^{a - 1}(1-x)^{b-1}, dx = \frac{\Gamma(a)\Gamma(b)}{\Gamma(a+b)}$ \\
$\Gamma(-1/2)=-2\Gamma(1/2)$ & $\int_0^\infty x^{t-1}e^{-x}dx = \Gamma(t)$ 
\end{tabular}


  \subsubsection{Integrals+c}

\begin{tabular}{ l|l|l }
$\int{k}dx=kx$ & $\int_a^b{c}dx=c(b-a)$   & $|\int_a^b{f(x)}dx|\le \int_a^b|{f(x)}|dx$ \\
$\int{e^u}du=e^u$ & $\int{lnu}du=uln(u)-u$& $\int\frac1{a^2+u^2}du=\frac1{a}tan^{-1}(\frac{u}a)$ \\
$\int{x^n}=\frac{x^{n+1}}{n+1}$ & $\int\frac1{ax+b}=\frac{1}aln|ax+b|$ & $\int\frac1{\sqrt{a^2-u^2}}du=\frac1{sin(\frac{u}a)}$ \\
$\int{\ln{a}a^x}=a^x$ &
\end{tabular}

\textbf{Substitution}
\begin{tabular}{ l|l|l }
$u=g(x)$ & $du=g'(x)dx$ & $\int_a^b{f(g(x))}g'(x)dx=\int_{g(a)}^{g(b)}{f(u)}du $ \\
$x^3$    & $du=3x^2dx$  & $\int_1^2{5x^2}cos(x^3)dx=\int_1^8\frac53{cos(u)}du $ \\
\end{tabular}

\textbf{Integreation by parts}
\begin{tabular}{ l|l|l|l|l }
$u$ & $du$ & $dv$     & $v$        & $\int_a^b{u}dv$ \\
$x$ & $dx$ & $e^{-x}$ & $-e^{-x}$  & $\int{x}e^{-x}dx$ \\ 
$lnx$ & $\frac1x{dx}$ & $dx$ & $x$ & $\int_3^5{x}lnxdx$ 
\end{tabular}

$=\left.uv\right|_a^b-\int_a^b{v}du $ \\
$=-xe^{-x}+\int{x}e^{-x}dx=-xe^{-x}-e^{-x}+c $ \\
$=\left.xlnx\right|_3^5-\int_3^5dx=\left.(xlnx-x)\right|_3^5=5ln5-3ln3-2 $

$\int_a^b{f(x)}dx=F(b)-F(a)=-\int_b^a{f(x)}dx=\int_a^c{f(x)}dx+\int_c^b{f(x)}dx$;
$\frac{d}{dx}\int_{v(x)}^{u(x)}{f(t)}dt=u'(x)f[u(x)]-v'(x)f[v(x)]$ 

  \subsubsection{Derivatives}
\begin{tabular}{ l|l|l }
$(cf)'=cf'(x)$               & $(fg)'=f'g+fg'$                      & $(f\pm g)'=f'(x)\pm g'(x)$ \\
$\frac{dx}{dx}=1$            & $(\frac{f}g)'=\frac{(f'g-fg')}{g^2}$ & $(f(g(x)))'=f'(g(x))g'(x)$ \\
$\frac{de^x}{dx}=e^x$        & $\frac{d(x^n)}{dx}=nx^{n-1}$         & $\frac{d\ln(x)}{dx}=\frac1x,x>0$ \\
$\frac{da^x}{dx}=a^xln(a)$   & $\frac{d\tan^{-1}x}{dx}=\frac1{1+x^2}$  & $\frac{d\log_a(x)}{dx}=\frac1{xlna},x>0$ \\
$\frac{d\cos{x}}{dx}=-\sin{x}$& $\frac{d\tan{x}}{dx}=\sec^2{x}$  & $\frac{dx(\ln{x}-1)}{dx}=\ln{x}$ 
\end{tabular} 




\end{multicols}

\begin{tabular}{llllllllll}
\textbf{Distribution} & \textbf{CDF} & \textbf{P(X=x),f(x)} & \textbf{$\mu$} & \textbf{$EX^2$} & \textbf{Var} & \textbf{MGF} & \textbf{M'(t)} & \textbf{M''(t)} & \textbf{$M^n(t)$}\\
\hline 

$\Bern(p) $ & $ $ & $p^xq^{1-x},x\in\{1,0\}$ & $p$ & $p$ & $pq$ & $pe^t+q$ \\
\hline

$\Bin(n,p)$ & $I_{1-p}(n-x,x+1)$ & $ \binom{n}{x}p^x q^{n-x}; x \in \{0,1..n\}$ & $np$ & $\mu(\mu+q)$ & $q\mu$ & $(pe^t+q)^n$ \\
\hline

$\Geom(p)$ & $1-q^x    $ & $pq^{x-1},x\in 1,2,..$ & $\frac1p  $ & $\frac{p+2q}{p^2} $ & $\frac{q}{p^2}$ & $\frac{pe^t}{1-qe^t},t<-\ln{q}$ \\
           & $1-q^{x+1}$ & $pq^x,x\in 0,1,..    $ & $\frac{q}p$ & $\frac{q^2+q}{p^2}$ & $\frac{q}{p^2}$ & $\frac{p}{1-qe^t}, qe^t<1     $ & $\frac{pqe^t}{(1-qe^t)^2}$ & $\frac{2pqe^t}{(1-qe^t)^3}-M'(t)$ \\
\hline

$\NBin(r,p)$ & $ $ & $\binom{x-1}{r-1}p^rq^{x-r},x\in r,r+1..$ & $\frac{r}p $ & $ $ & $\frac{rq}{p^2}$ & $(\frac{pe^t}{1-qe^t})^r$\\
             & $ $ & $\binom{x+r-1}{r-1}p^rq^x, x \in 0,1..  $ & $\frac{rq}p$ & $ $ & $\frac{rq}{p^2}$ & $(\frac{p}{1-qe^t})^r, qe^t<1$\\
\hline

$\Hypergeometric(N,m,k)$ & $ $ & $\frac{\binom{m}{x}\binom{N-m}{k-x}}{\binom{N}{k}}$ & $\frac{km}{N}     $ & $ $ & $\mu\frac{(N-m)(N-k)}{N(N-1)}$ \\
$\Hypergeometric(w,b,k)$ & $ $ & $\frac{\binom{w}{x}\binom{b}{k-x}}{\binom{w+b}{k}}$ & $\frac{kw}{w+b}$ & $ $ & $\mu\frac{b(w+b-k)}{(w+b)(w+b-1)}$   \\
\hline

$\Pois(\mu)$ & $e^{-\mu}\sum_{i=0}^x\frac{\mu^i}{i!}$ & $\frac{\mu^x}{x!}e^{-\mu},x \in 0,1..$ & $\mu$ & $\mu^2+\mu$ & $\mu$ & $e^{\mu(e^t-1)}$ & $\mu e^tM(t)$ & $\mu e^t(1+\mu e^t)M(t)$ \\
\hline

$\Unif(n)  $ & $               $ & $ \frac{1}n,x \in 1,2..n   $ & $\frac{n+1}2  $ & $\frac{(n+1)(2n+1)}{6}$ & $\frac{(n^2-1)}{12}$ &  $\frac{\sum_{i=1}^n{e^{ti}}}n$\\
\hline
\hline

$\Unif(a,b)$ & $\frac{x-a}{b-a}$ & $ \frac{1}{b-a},x \in(a,b) $ & $\frac{a+b}{2}$ & $ $& $\frac{(b-a)^2}{12}$ &  $\frac{e^{tb}-e^{ta}}{t(b-a)}$\\
\hline

$\N(\mu,\sigma^2)$ & $ $ & $\frac{1}{\sigma\sqrt{2\pi}} e^{-\frac{(x-\mu)^2}{2\sigma^2}}$ & $\mu$ & $\mu^2+\sigma^2$ & $\sigma^2$ & $e^{\mu t +\frac{\sigma^2t^2}2}$ & $(\mu+\sigma^2t)M(t)$ & $[(\mu+\sigma^2t)^2+\sigma^2]M(t) $\\
\hline

$\N(0, 1)        $ & $ $ & $\frac{1}{\sqrt{2\pi}}e^{-\frac{x^2}2}$ & $0$ & $1$ & $1$ & $e^{\frac{t^2}2}$ \\
\hline

$\mathcal{LN}(\mu,\sigma^2)$ & $ $ & $\frac{1}{x\sigma \sqrt{2\pi}}e^{\frac{-(\ln x-\mu)^2}{2\sigma^2}}$ & $e^{\mu+\frac{\sigma^2}2}$ & $e^{2\mu+2\sigma^2}$ & $\theta^2(e^{\sigma^2}-1)$ & $\times$\\
\hline

$Cauchy(\theta,\sigma^2)$ & $ $ & $\frac{1}{\pi\sigma}\frac1{1+(\frac{x-\theta}{\sigma})^2}$ & $\times$ & $\times$ & $\times$ & $ $ \\
\hline

$D\Expo(\mu,\sigma^2)$ & $ $ & $\frac{1}{2\pi\sigma} e^{-|\frac{x-\mu}{\sigma}|}$ & $\mu$ & $\mu^2+2\sigma^2$ & $2\sigma^2$ & $\frac{e^{\mu t}}{1-\sigma^2t^2}$ \\
\hline

$\Expo(\lambda)$ & $1-e^{-\lambda x}$ & $\lambda e^{-\lambda x},x \in (0,\infty)$ & $\frac{1}{\lambda}$ & $ $ & $\frac{1}{\lambda^2}$ & $\frac{\lambda}{\lambda - t}, t < \lambda$\\
$\Expo(\beta)  $ & $                $ & $\frac1{\beta} e^{-\frac{x}\beta}$ & $\beta$ & $ $ & $\beta^2$ & $\frac{1}{1-\beta t}$ & $\beta(1-\beta t)^{-2}$ & $2\beta^2(1-\beta t)^{-3}$\\
\hline

$\Gam(a, \lambda)$ & $ $ & $\frac{\lambda^a}{\Gamma(a)}x^{a-1}e^{-\lambda x},x \in (0,\infty)$ & $\frac{a}{\lambda}$  & $\frac{a}{\lambda^2}$ & $\left(\frac{\lambda}{\lambda - t}\right)^a, t < \lambda$\\
$\Gam(\alpha,\beta)$ & $ $ & $\frac{1}{\Gamma(a)\beta^{\alpha}}x^{a-1}e^{-x/\beta}$ & $\alpha\beta$  & $\alpha\beta^2$ & $\left(\frac{1}{1-\beta t}\right)^a, t <\frac1\beta$\\
\hline

$\Beta(a, b)$ & $ $ & $\frac{\Gamma(a+b)}{\Gamma(a)\Gamma(b)}x^{a-1}(1-x)^{b-1} $ & $\frac{a}{a+b}$ & $ $  & $\frac{\mu(1-\mu)}{(a+b+1)}$ & $ $ & $ $ & $ $ & $\frac{\Gamma(\alpha+n)\Gamma(\alpha+\beta)}{\Gamma(\alpha+\beta+n)\Gamma(\alpha)}$  \\
$B(\alpha,\beta)=$ & $ $ & $\frac{\Gamma(\alpha)\Gamma(\beta)}{\Gamma(\alpha+\beta)},x\in(0,1)$ & $ $  & $\frac{a(a+1)}{(a+b)(a+b+1)}$ & $\frac{ab}{(a+b)^2(a+b+1)}$ \\
\hline

$\chi_p^2$ & $ $ & $\frac{1}{2^{p/2}\Gamma(p/2)}x^{p/2-1}e^{-x/2}$ & $p$ & $2p+p^2$ & $2p$ & $(1-2t)^{-p/2}, t<\frac12$\\
\hline

$t_n$ & $ $ & $\frac{\Gamma(\frac{n+1}2)}{\sqrt{n\pi} \Gamma(\frac{n}2)} (1+\frac{x^2}n)^{-\frac{n+1}2}$ & $0,n>1$ & $ $ & $\frac{n}{n-2},n>2$ & $\times$\\
\hline
\end{tabular}



\end{document}